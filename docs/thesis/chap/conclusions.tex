% \TODO{Write an experiment for which static instruction counts fall
% down. For example, two programs with similar instruction counts, one
% with a huge loop, the other with straight line code.}

\section{Critical Analysis}

% TODO: Review against project plan.

% Achievements:
%
% Understanding a large code base (SkelCL)

% Things I could have done
%
% How many features are required?
% How many synthetic benchmarks are required?
% Compare against hand-crafted stencils


\section{Future Work}

\begin{algorithm}
\begin{algorithmic}[1]
\Require kernel features $k$, hardware features $h$, dataset features $d$.
\Ensure workgroup size $w$

\State $r \leftarrow \underset{w \in W_{legal}(s)}{\min} f(k,h,d,w)$
\Comment Predict minimum runtime.
\State $w \leftarrow \underset{w \in W_{legal}(s)}{\argmin} f(k,h,d,w)$
\Comment Workgroup size for $r$.
\State $t_r \leftarrow$ measure runtime of program with $w$
\If{$t_r \approx r$}
  \State \textbf{return} $w$
\Comment Predicted runtime is accurate.
\Else
   \State $W \leftarrow W_{legal}(s)$
   \State converged $\leftarrow$ false
   \State $w_b \leftarrow$ baseline value
   \State $t_b \leftarrow$ measure runtime of runtime of program with $w_b$
   \While{not converged}
     \State $s \leftarrow \underset{w \in W}{\max} g(k,h,d,w)$
     \Comment Predict best speedup.
     \State $w \leftarrow \underset{w \in W}{\argmax} g(k,h,d,w)$
     \Comment Workgroup size for $s$.
     \State $t \leftarrow$ measure runtime of program with $s$
     \If{$t_b / t \approx s$}
       \State converged = true
       \Comment Predicted speedup is accurate.
     \Else
       \State $W = W - \{w\}$
     \EndIf
   \EndWhile
   \State \textbf{return} $w$
\EndIf
\end{algorithmic}

\caption{%
  Selecting workgroup size using a combination of classifiers and
  regressors.%
}
\label{alg:autotune-hybrid}
\end{algorithm}

% TODO: Auto-discovery of errors & bug reports

% TODO: Extensions to SkelCL: Cluster parallelism, multi-dimensional
% stencils, stencil specialisations for specific memory access
% patterns.

% TODO: Additional optimisation parameters, e.g. which device to
% execute on.

% TODO: Apply to additional skeleton libraries.

% TODO: energy as an optimisation cotarget.


% Leather, H., O’Boyle, M., & Worton, B. (2009). Raced Profiles:
% Efficient Selection of Competing Compiler Optimizations. In LCTES
% ’09: Proceedings of the ACM SIGPLAN/SIGBED 2009 Conference on
% Languages, Compilers, and Tools for Embedded Systems
% (pp. 1–10). Dublin.
\TODO{%
  Consider using adaptive sampling plans and some sort of global
  mechanism for performing cooperative exploration across multiple
  devices, while reducing the number of samples required to
  distinguish good from bad workgroup sizes~\cite{Leather2009}.%
}

\TODO{%
  Make the remote active. Rather than simply fetching from remote SQL
  tables, the remote could be a webserver with a REST api for HTTP
  GET/PUT of training data, and could analyse the data and build
  models asynchronously.%
}
