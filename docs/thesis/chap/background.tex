\TODO{Challenges of parallel programming.}

\section{Automatic Parallelisation}

\TODO{Brief summary of auto-parallelising compilers. Highlight the
  shortcomings which lay the scene for Alg Skel to succeed where
  compilers haven't.}

\section{High-Level Parallel Programming with Algorithmic Skeletons}

\subsection{Abstracting Task and Data Parallelism}

\TODO{Short examples of task parallel skeletons. Describe the
  shortcomings of GPUs for task parallel operations. Don't repeat
  myself from the data parallel operations defined in the SkelCL
  intro.}

\subsection{Algorithmic Skeleton Frameworks}

\TODO{Case studies of skeleton-esque products in industry: Intel
  Thread Building Blocks, and Google's MapReduce and the various
  implementations. Comment on their lack of ``Skeleton'' branding.}


\section{Heterogeneous parallelism with GPUs}

\TODO{Start with a description of GPU hardware, i.e. SIMD EUs, many
  simple cores, lockstep execution. Describe the historical reasoning
  for this and how it influences the types of programs which are well
  suited for GPUs. GPUs offer huge throughput, but are even more
  challenging to program and debug. Give concrete examples.}

% S. Ryoo, C. I. Rodrigues, S. S. Baghsorkhi, S. S. Stone, D. B. Kirk,
% and W. W. Hwu, “Optimization principles and application performance
% evaluation of a multithreaded GPU using CUDA,” Proc. 13th ACM
% SIGPLAN Symp. Princ. Pract. parallel Program. - PPoPP ’08, p. 73,
% 2008.
\TODO{The challenges and approaches to optimising GPU programs. CUDA
programming requires the developer to explicitly manage data layout in
DRAM, caching, thread communication and other resources. Performance
of such programs depends heavily on fully utilizing zero-overhead
thread scheduling, memory bandwidth, thread grouping, shared control
flow, and intra-block thread communication. The paper gives an example
of optimising matrix multiplication by utilising shared memory through
tiling, and loop unrolling. \cite{Ryoo2008a}}

Prior research has shown that the performance of a GPGPU program is
heavily influenced by proper exploitation of local shared memory and
synchronisation costs~\cite{Ryoo2008a, Lee2010} \TODO{``Where's the
  data?'' paper}

% S. Ryoo, C. I. Rodrigues, S. S. Stone, S. S. Baghsorkhi, S.-Z. Ueng,
% J. a. Stratton, and W. W. Hwu, “Program optimization space pruning
% for a multithreaded GPU,” in Proceedings of the 6th annual IEEE/ACM
% international symposium on Code generation and optimization, 2008,
% pp. 195–204.
\TODO{Background info about optimizing for GPUs, including a checklist
of optimization metrics: \cite{Ryoo2008}}

% D. Grewe, Z. Wang, and M. F. P. M. O’Boyle, “Portable mapping of
% data parallel programs to OpenCL for heterogeneous systems,” in Code
% Generation and Optimization (CGO), 2013 IEEE/ACM International
% Symposium on, 2013, pp. 1–10.
\TODO{Vaguely ``autopar''-esque: automatic translation from OpenMP to
  OpenCL~\cite{Grewe2013}.}


\subsubsection{GPU vs CPU performance analysis}

% V. W. Lee, P. Hammarlund, R. Singhal, P. Dubey, C. Kim, J. Chhugani,
% M. Deisher, D. Kim, A. D. Nguyen, N. Satish, M. Smelyanskiy, and
% S. Chennupaty, “Debunking the 100X GPU vs. CPU myth,” ACM SIGARCH
% Comput. Archit. News, vol. 38, p. 451, 2010.
In~\cite{Lee2010}, \citeauthor{Lee2010} present a performance analysis
of optimised throughput computing applications for GPUs and CPUs. Of
the 14 applications tested, they found GPU performance to be
$0.7\times$-$14.9\times$ that of multi-threaded CPU code, with an
average of only 2.5$\times$. This is much lower than the
$100\times$-$1000\times$ values reported by other studies,
% TODO: ^^ Citation needed!
a fact that they attribute to uneven comparison of optimised GPU code
to unoptimised CPU code, or vice versa. \citeauthor{Lee2010} found
that multithreading, cache blocking, reordering of memory accesses and
use of SIMD instructions to contribute most to CPU performance. For
GPUs, the most effective optimisations are reducing synchronization
costs, and exploiting local shared memory. In all cases, the programs
were optimised and hand-tuned by programmers with expert knowledge of
the target architectures. It is unclear whether their performance
results still hold for subsequent generations of devices.


\subsection{GPGPU Programming}

\TODO{Introduction to popular GPU programming models: OpenCL and
  CUDA.}

% K. Komatsu, K. Sato, Y. Arai, K. Koyama, H. Takizawa, and
% H. Kobayashi, “Evaluating performance and portability of OpenCL
% programs,” in The fifth international workshop on automatic
% performance tuning, 2010, p. 7.
\citeauthor{Komatsu2010} provide quantitative comparisons between the
performance of CUDA and OpenCL kernels
in~\cite{Komatsu2010}. \TODO{This paper is deserving of close
  scrutiny, as they perform an in-depth evaluation of the effect of
  workgroup size on the performance of kernels across devices and
  programs.}

% J. Fang, A. L. Varbanescu, and H. Sips, “A Comprehensive Performance
% Comparison of CUDA and OpenCL,” in Parallel Processing (ICPP), 2011
% International Conference on, 2011, pp. 216–225.
\TODO{Comparing OpenCL and CUDA performance: \cite{Karimi2010}}

% Enmyren, J., & Kessler, C. (2010). SkePU: a multi-backend skeleton
% programming library for multi-GPU systems. In Proceedings of the
% fourth international workshop on High-level parallel programming and
% applications (pp. 5–14). ACM. Retrieved from
% http://dl.acm.org/citation.cfm?id=1863487
\paragraph{SkePU} \TODO{Similar to SkelCL in scope and ambition,
  alternative implementation with C++ macros and
  CUDA~\cite{Enmyren2010}.}

\subsection{The OpenCL Programming Model}

\TODO{Technical details of OpenCL model. Describe the memory model.}


% Rul, S., Vandierendonck, H., Haene, J. D., & Bosschere,
% K. De. (2010). An Experimental Study on Performance Portability of
% OpenCL Kernels. In 2010 Symposium on Application Accelerators in
% High Performance Computing (SAAHPC’10) (pp. 4–6).
\TODO{Portability of OpenCL kernels: \cite{Rul2010}}

\subsubsection{Problem decomposition and workgroups}

In OpenCL, kernels are mapped to work-items for execution on the
processing units of GPUs and CPUs. These work-items are then grouped
into one or more workgroups. The choice of workgroup size is left to
the developer, but with two sets of constraints. The first set of
constraints is the maximum workgroup size which an execution device
supports. This value cannot be exceeded, irrespective of the kernel
being executed. The second constraint is the maximum workgroup size a
kernel supports, and can only be queried at runtime once a kernel has
been compiled. The selection of workgroup size for stencil skeletons
is particularly relevant to the performance of the stencil as it
affects utilisation of fast local memory. In a stencil code, each
work-item reads the values of multiple neighbouring elements. To
facilitate this, the value of all elements within a workgroup are
stored in fast local memory, which greatly reduces the read latency
for GPUs. Changing the workgroup size affects the amount of local
memory required for each workgroup, which in turn affects the number
of workgroups which may be simultaneously active.


\section{Iterative Compilation and Autotuning}

% O. Jeffrey and M. David, “The Vision of Autonomic Computing,”
%Computer (Long. Beach. Calif)., vol. 36, no. 1, pp. 41–50, 2003.
\TODO{The ideal vision of self-governing agents in complex
environments. We can draw a parallel to the use of self-optimizing
agents: \cite{Jeffrey2003}}

% J. Bilmes, K. Asanovic, C.-W. Chin, and J. Demmel, “Optimizing
% Matrix Multiply Using PHiPAC: A Portable, High-performance, ANSI C
% Coding Methodology,” in Proceedings of the 11th International
% Conference on Supercomputing, 1997, pp. 340–347.
\TODO{Guidelines for optimising, using an automatic code generator and
a search engine to optimise parameters: \cite{Bilmes1997}}

% SUPEROPTIMISERS:
%
% H. Massalin, “Superoptimizer -- A Look at the Smallest Program,” ACM
% SIGPLAN Not., vol. 22, no. 10, pp. 122–126, 1987.
%
% R. Joshi, G. Nelson, and K. Randall, “Denali: a goal-directed
% superoptimizer,” ACM SIGPLAN Not., vol. 37, no. 5, p. 304, 2002.
\TODO{Super-optimisers: provably~\cite{Joshi2002} or
  demonstrably~\cite{Massalin1987} optimal, but unbelievably
  costly. Can ML help?}

\paragraph{PETABRICKS}

% Ansel, J. (2009). PetaBricks: a language and compiler for
% algorithmic choice. MIT.
\TODO{\cite{Ansel2009}}

% Ansel, J., & Chan, C. (2010). PetaBricks. XRDS: Crossroads, The ACM
% Magazine for Students, 17(1), 32. doi:10.1145/1836543.1836554
\TODO{\cite{Ansel2010}}

% Chan, C., Ansel, J., Wong, Y. L., Amarasinghe, S., & Edelman,
% A. (2009). Autotuning multigrid with PetaBricks. In ACM/IEEE
% Conference on Supercomputing. New York, New York, USA: ACM
% Press. doi:10.1145/1654059.1654065
\TODO{\cite{Chan2009}}

% Ansel, J. (2014). Autotuning Programs with Algorithmic
% Choice. Massachusetts Institute of Technology.
\TODO{\cite{Ansel2014}}

\subsection{Machine Learning in Compilers}

% M. Stephenson, M. Martin, and U. O. Reilly, “Meta Optimization:
% Improving Compiler Heuristics with Machine Learning,” ACM SIGPLAN
% Not., vol. 38, no. 5, pp. 77–90, 2003.
\TODO{Using ML to improve compiler heuristics~\cite{Stephenson2003}}

% E. K. Burke, M. Gendreau, M. Hyde, G. Kendall, G. Ochoa, E. Özcan,
% and R. Qu, “Hyper-heuristics: a survey of the state of the art,”
% J. Oper. Res. Soc., vol. 64, pp. 1695–1724, 2013.
\TODO{Hyper-heuristics survey paper: \cite{Burke2013}}

%
% SOME EXAMPLES OF ML IN THE WILD:
%

% R. Bitirgen, E. Ipek, and J. F. Martinez, “Coordinated Management of
% Multiple Interacting Resources in Chip Multiprocessors: A Machine
% Learning Approach,” in 2008 41st IEEE/ACM International Symposium on
% Microarchitecture, 2008, pp. 318–329.
\TODO{Artificial Neural Networks for resource allocation of CMPS:
\cite{Bitirgen2008}}

% G. Fursin, Y. Kashnikov, A. W. Memon, Z. Chamski, O. Temam,
% M. Namolaru, E. Yom-Tov, B. Mendelson, A. Zaks, E. Courtois,
% F. Bodin, P. Barnard, E. Ashton, E. Bonilla, J. Thomson,
% C. K. I. Williams, and M. O’Boyle, “Milepost GCC: Machine Learning
% Enabled Self-tuning Compiler,” Int. J. Parallel Program., vol. 39,
% no. 3, pp. 296–327, Jan. 2011.
\TODO{Milepost GCC is a machine learning enabled self-tuning
  compiler~\cite{Fursin2011}.}

% Z. Wang and M. F. P. O. Boyle, “Partitioning Streaming Parallelism
% for Multi-cores: A Machine Learning Based Approach,” in Proceedings
% of the 19th international conference on Parallel architectures and
% compilation techniques, 2010, pp. 307–318.
\TODO{Offline ML for partitioning streaming applications:
\cite{Wang2010}}

%
% TOWARDS ONLINE TUNING:
%

% Ansel, J., & Reilly, U. O. (2012). SiblingRivalry: Online Autotuning
% Through Local Competitions. In Proceedings of the 2012 International
% Conference on Compilers, Architectures and Synthesis for Embedded
% Systems (pp. 91–100). ACM. doi:10.1145/2380403.2380425
\TODO{\cite{Ansel2012}}

% Eastep, J., Wingate, D., & Agarwal, A. (2011). Smart Data
% Structures: An Online Machine Learning Approach to Multicore Data
% Structures. In Proceedings of the 8th ACM International Conference
% on Autonomic Computing (pp. 11–20). New York, NY, USA:
% ACM. doi:10.1145/1998582.1998587
%
% Tesauro, G. (2005). Online Resource Allocation Using Decompositional
% Reinforcement Learning. In AAAI (pp. 886–891).
\paragraph{Reinforcement learning} \TODO{\cite{Eastep2011},
  \cite{Tesauro2005}}

% W. F. Ogilvie, P. Petoumenos, Z. Wang, and H. Leather, “Intelligent
% Heuristic Construction with Active Learning,” in 18th International
% Workshop on Compilers for Parallel Computing, 2015.
\paragraph{Active Learning} \TODO{Using Active Learning to speed up
  the learning of compiler heuristics~\cite{Ogilvie2015}. Towards
  online auto-tuning, albeit only with binary optimisation parameter.}

\subsection{Autotuning for GPUs}

% Ryoo, S., Rodrigues, C. I., Stone, S. S., Baghsorkhi, S. S., Ueng,
% S.-Z., Stratton, J. a., & Hwu, W. W. (2008). Program optimization
% space pruning for a multithreaded GPU. In Proceedings of the 6th
% annual IEEE/ACM international symposium on Code generation and
% optimization (pp. 195–204). New York, New York, USA: ACM
% Press. doi:10.1145/1356058.1356084
\TODO{\cite{Ryoo2008}}

% G. Chen and B. Wu, “PORPLE: An Extensible Optimizer for Portable
% Data Placement on GPU,” in Microarchitecture (MICRO), 2014 47th
% Annual IEEE/ACM International Symposium on, 2014, pp. 88–100.
\TODO{Optimising data placement on GPUs using a description of the
hardware and a memory-placement-agnostic compiler. Shows impressive
speedups (2.08x max, 1.59x avg) from just optimising data
placement. \cite{Chen2014}}

% Lee, H., Brown, K. J., Sujeeth, A. K., Rompf, T., & Olukotun,
% K. (2014). Locality-Aware Mapping of Nested Parallel Patterns on
% GPUs. In Microarchitecture (MICRO), 2014 47th Annual IEEE/ACM
% International Symposium on
% (pp. 63–74). IEEE. doi:10.1109/MICRO.2014.23
\TODO{GPU benchmarking: \cite{Lee}}

% A. Magni, C. Dubach, and M. O’Boyle, “Automatic optimization of
% thread-coarsening for graphics processors,” in International
% Conference on Parallel Architectures and Compilation, 2014,
% pp. 455–466.
\citeauthor{Magni2014} explored the effect of thread coarsening
in~\cite{Magni2014}, achieving average speedups between 1.11$\times$
and 1.33$\times$ using a machine-learning model based on the static
features of kernels.

% Steuwer, M., Fensch, C., & Dubach, C. (2015). Patterns and Rewrite
% Rules for Systematic Code Generation From High-Level Functional
% Patterns to High-Performance OpenCL Code. arXiv Preprint
% arXiv:1502.02389.
\TODO{Using re-write rules to translate high-level programs to OpenCL:
\cite{Steuwer2015}}


\subsection{Autotuning Algorithmic Skeletons}

% Contreras, G., & Martonosi, M. (2008). Characterizing and improving
% the performance of Intel Threading Building Blocks. In Workload
% Characterization, 2008. IISWC 2008. IEEE International Symposium on
% (pp. 57–66). IEEE. doi:10.1109/IISWC.2008.4636091
\TODO{INTEL TBB \cite{Contreras2008}}

% Collins, A., Fensch, C., & Leather, H. (2012). Auto-Tuning Parallel
% Skeletons. Parallel Processing Letters, 22(02),
% 1240005. doi:10.1142/S0129626412400051
%
% Collins, A., Fensch, C., Leather, H., & Cole, M. (2013). MaSiF:
% Machine Learning Guided Auto-tuning of Parallel Skeletons. 20th
% Annual International Conference on High Performance Computing -
% HiPC, 186–195. doi:10.1109/HiPC.2013.6799098
\paragraph{MaSiF} \TODO{\cite{Collins2012}, \cite{Collins2013}}


% Dastgeer, U., Enmyren, J., & Kessler, C. W. (2011). Auto-tuning
% SkePU: a multi-backend skeleton programming framework for multi-GPU
% systems. In Proceedings of the 4th International Workshop on
% Multicore Software Engineering (pp. 25–32). ACM. Retrieved from
% http://dl.acm.org/citation.cfm?id=1984697
\citeauthor{Dastgeer2011} developed a machine learning based autotuner
for the SkePU skeleton library in~\cite{Dastgeer2011}. Training data
is used to select the optimal backend (i.e.\ CPU, GPU) for a given
program by estimating execution time and memory copy overhead based on
problem size. There is limited cross-architecture evaluation, and the
autotuner only supports vector operations.

% Dastgeer, U., & Kessler, C. (2015). Smart Containers and Skeleton
% Programming for GPU-Based Systems. International Journal of Parallel
% Programming, 1–25. doi:10.1007/s10766-015-0357-6
\TODO{Smart containers for SkePU:~\cite{Dastgeer2015a}}


\subsection{Autotuning Stencil Codes}

% Y. Zhang and F. Mueller, “Auto-generation and Auto-tuning of 3D
% Stencil Codes on GPU clusters,” in Proceedings of the Tenth
% International Symposium on Code Generation and Optimization, 2012,
% pp. 155–164.
In~\cite{Zhang2013a}, \citeauthor{Zhang2013a} present a code generator
and autotuner for 3D Jacobi stencil codes. Using a DSL to express
kernel functions, the code generator performs substitution from one of
two CUDA templates to create GPU programs. These programs are
parameterised and tuned for block size, block dimensions, and whether
input data is stored in read only texture memory. This creates a
search space of up to 200 configurations.

The authors report impressive performance, comparable with previous
implementations of iterative Jacobi stencils on
GPUs~\cite{Holewinski2012,Phillips2010}. The dominating parameter is
shown to be block dimensions, followed by block size, then read only
memory. An unfortunate limitation of their system is that they do not
support multi-GPU execution within a single node, instead using
message passing in a cluster. This rules out the possible benefits of
performing loop compaction for iterative stencils, an optimisation
which has shown to be beneficial for multi-GPU performance
in~\cite{Lutz2013}.

The DSL presented in the paper is limited to expressing only Jacobi
Stencils applications. Critically, their autotuner requires a full
enumeration of the parameter space for each program, with no element
of search space reduction or cross-program learning. Since there is no
evidence of the compute time required to perform this enumeration, it
gives the impression that the system would be impractical for the
needs of general purpose Stencil
computing.
% The majority of applications tested are memory bound. Does this
% transfer to computer bound?

In~\cite{Ganapathi2009}, \citeauthor{Ganapathi2009} argues for
applying statistical machine learning (SML) techniques to develop
autotuners for multicore software. They present an autotuner for
Stencil codes which can achieve performance within 1\% or up to 18\%
better than that of a human expert after 2 hours of running. They
evaluate the performance of a randomly selected 1500 configurations
(from a posible 10 million configs), and use Kernel Canonical
Correlation Analysis (KCCA) to build correlations between tunable
parameter values and measured performance values. Performance is
measured using hardware counters (L1 cache misses, TLB misses, cycles
per thread) and power consumtion in Watts/sec. KCCA seems like a
strange choice: it scales exponentially with the feature vector sizes,
and it takes 2 hours (!!!) to build the ML model for 400 sec worth of
benchmark data. They present an interesting argument that enegy
efficiency should be used as an autotuning target as well as just run
time, since it was the power wall that lead to the multicore
revolution in the first place. They explain the motivation and results
well. I like that they compare their results with human expert and
hardware upper bound. It is a solid paper which makes a compelling
argument, but their choice of only 2 benchmarks and 2 platforms makes
the evaluation of their autotuner a little limited.

\citeauthor{Kamil2010} presents an auto-tuning framework which accepts as input a
Fortran 95 stencil expression, and generates tuned parallel
implementations in Fortan, C, or CUDA. The system uses an IR to
explore auto-tuning transformations, and has an SMP backend code
generator. They demonstrate their system on 4 architectures using 3
benchmarks, with speedups of up to x22 over serial. The CUDA code
generator only uses global memory (!!). Also, there's no real search
engine. They randomly enumerate a subset of the optimisation space,
and then record only a single execution time (!!!), reporting the
fastest. Cited by 127.~\cite{Kamil2010}

\paragraph{PATUS} In~\cite{Christen2011}, \citeauthor{Christen2011}
presents a DSL for expressing stencil codes, a C code generator, and
an autotuner for exploring the optimisation space, using blocking and
vectorisation strategies. Pro: Supports arbitrarily high dimensional
grids. They introduce *2* new DSLs without comment on why they were
needed. This is a huge price of entry for anyone who actually wants to
*use* PATUS to solve problems, and a decision that I think they could
have justified better. They provide nice and concise explanations of
stencil codes and the types of optimisation methods used by other
autotuners (ATLAS, FLAME, FFTW, SPIRAL). However, they *barely*
explain their own, saying only that they perform *either* an
exhaustive, multi-run Powell, Nelder Mead, or evolutionary algorithms.
Their evaluation uses 6 benchmarks, and 3 architectures. Only 1 of
those is a GPU. It would have been nice to shown performance across
GPU architectures. From the perspective of autotuning, the paper comes
as across quite weak: They do not present an "oracle" performance, so
we can't compare the quality of their autotuner compared to it. They
don't show how the optimal tunable parameter values vary across
results, so they don't demonstrate how autotuning is *necessary*
(maybe there's one single value which works well for all
results?). They do not give any indication of convergence time of
their autotuner. They do not report the number of different
combinations that their autotuner tries.

% T. Lutz, C. Fensch, and M. Cole, “PARTANS: An Autotuning Framework
% for Stencil Computation on Multi-GPU Systems,” ACM
% Trans. Archit. Code Optim., vol. 9, no. 4, pp. 1–24, 2013.
\paragraph{PARTANS} \citeauthor{Lutz2013} explored the effect of
varying the size of the halo regions for multi-GPU stencil
computations in~\cite{Lutz2013}. A large halo increases redundant
computation across GPUs but decreases communication costs. They found
that the optimal halo size depends on the problem size, number of
partitions, and the connection mechanism (i.e.\ PCI express). They
developed an autotuner which determines problem decomposition and
swapping strategy offline, and performs an online search for the
optimal halo size. The study is limited to only three optimisation
parameters of a single class of program, and the results of tuning are
not shared across programs or across multiple runs of the same
program.
