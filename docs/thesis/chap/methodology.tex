\section{Introduction}

This chapter describes an experiment to explore the performance
influence of implementation-independent factors (architecture,
programs, and datasets) on the performance of Stencil programs using
different workgroup sizes. It contains the methodology used to achieve
statistically sound comparisons of competing programs using different
configurations, and the experimental setup.


\section{Defining the Optimisation Space}

This section describes the benchmarks, architectures, and datasets
used to gather performance data.


\subsection{Benchmark Stencil Applications}

To explore the effect of workgroup size across different stencils,
reference implementations of standard stencil applications from the
fields of image processing, cellular automata, and partial
differential equation solvers are used.

% \begin{table}
% \footnotesize
% \centering
% \begin{tabular}{| l | l | l | l | l |}
% \hline
% \textbf{Name} & \textbf{Application} & \textbf{Skeletons used} & \textbf{Iterative?} & \textbf{LOC}\\
% \hline
% CannyEdgeDetection & Image processing & Stencil & - & 225 / 61\\
% DotProduct & Linear algebra & Zip, Reduce & - & 143 / 2\\
% FDTD & Scientific simulation & Map, Stencil & Y & 375 / 127\\
% GameOfLife & Cellular automata & Stencil & Y & 92 / 12\\
% GaussianBlur & Image processing & Stencil & - & 262 / 47\\
% HeatSimulation & Scientific simulation & Stencil & Y & 180 / 13\\
% MandelbrotSet & Fractal computation & Map & Y & 133 / 78\\
% MatrixMultiply & Linear algebra & AllPairs & - & 267 / 8\\
% SAXPY & Linear algebra & Zip & - & 149 / 3\\
% \hline
% \end{tabular}
% \caption{Benchmark applications. The LOC column shows lines of code, split between host (C++) and device (OpenCL).}
% \label{tab:benchmarks}
% \end{table}

\TODO{Descriptions (with diagrams) for each application:}

\subsubsection{Finite Difference Time Domain}

\subsubsection{Heat Equation}

\subsubsection{Gaussian Blur}

\TODO{Reword this to be more clear}

The Gaussian blur is a common image processing algorithm, used to
reduce noise and detail in an image. A two dimensional Gaussian blur
defines a function to compute a pixel value $f(x,y)$ using a standard
deviation $\sigma$ using:

\begin{equation}
f(x,y) = \frac{1}{2\pi\sigma^2}e^{-\frac{x^2 + y^2}{2\sigma^2}}
\end{equation}

Gaussian blurs are parameterised by a radius $r$ which define
symmetric, square stencil regions about the centre
pixel. \TODO{Describe the range of blur radius' used.} Unlike the
previous two applications, the Gaussian blur is not an iterative
stencil.

\subsubsection{Game of Life}

Conway's Game of Life~\cite{Conway1970} is a cellular automaton which
models the evolution of a regular grid of cells over discrete time
steps. At each time step, each cell value is updated to be either
\emph{live} or \emph{dead} based on it's current value and the value
of the nearest neighbouring elements (Algorithm~\ref{alg:gol}).

\begin{algorithm}[b]
\caption{Conway's Game of Life}
\label{alg:gol}
\begin{algorithmic}[1]
\Require current cell value $d_t$, neighbouring cell values $D_t$.
\Ensure new cell value $d_{t+1}$
\State $n \leftarrow $sum$(D_t)$
\If{$n = 3$ OR $(n = 2$ AND $d_t = 1)$}
  \State \textbf{return} 1
\Else
  \State \textbf{return} 0
\EndIf
\end{algorithmic}

\end{algorithm}

The stencil shape for Game of Life is always 1 in each direction.

\subsubsection{Canny Edge Detection}

The Canny edge detection algorithm consists of four distinct stages:
\TODO{4 separate stencils, with different behaviours.}


\subsection{Selection of Parameter Values}

The parameter space is defined as the number of rows and columns in a
workgroup. To explore this 2D space, all workgroup sizes over the set
$\{1 \times 1, 2 \times 2, 100 \times 100\}$ were evaluated, subject
to satisfying two hard constraints: the maximum workgroup size imposed
by the architecture, and the maximum workgroup size imposed by a
kernel. Exceeding either of these constraints will cause an
\texttt{CL\_OUT\_OF\_RESOURCES} error when the OpenCL kernel is
enqueued.

\subsubsection{Architecture constraints}

Each OpenCL device imposes a maximum workgroup size which can be
statically checked by querying the \texttt{clGetDeviceInfo()}
API. These are defined by the limits of the hardware, typically the
number of execution units which have access to a shared memory pool.

\subsubsection{Kernel constraints}

At runtime, once an OpenCL program has been compiled to a kernel,
users can query the maximum workgroup size supported by that kernel
using the \texttt{clGetKernelInfo()} API. Crucially, this value cannot
easily be obtained statically, as there is no mechanism to determine
the maximum workgroup size for a given source code and device without
first compiling it, which in OpenCL does not occur until
runtime. Factors which affect a kernel's maximum workgroup size
include the number registers required for a kernel, and the available
number of SIMD execution units for each type of instructions in a
kernel.


\section{Experimental Setup}

This section describes the environment under which performance data
was collected. Tables~\ref{tab:hw},~\ref{tab:kernels},
and~\ref{tab:datasets} list the range of execution devices, kernels,
and datasets used. For each unique combination of architecture,
kernel, and dataset (hereby referred to as a \emph{scenario}),
training data was collected by iteratively sampling the space of
workgroup sizes.

\TODO{Description of test systems. E.g. mobo, memory, \& GPUs.}

\begin{table}
\scriptsize
\centering
\rowcolors{2}{white}{gray!25}
\scriptsize
\centering
\rowcolors{2}{white}{gray!25}
\scriptsize
\centering
\rowcolors{2}{white}{gray!25}
\input{gen/tab/hosts}



\caption{%
  Specification of experimental platforms.%
}
\label{tab:hosts}
\end{table}

\begin{table}
\scriptsize
\centering
\rowcolors{2}{white}{gray!25}
\begin{tabular}{ L{4.5cm} L{1.5cm} L{1.5cm} L{1.5cm} L{1.5cm} L{1.5cm} }
\hline
\scriptsize
\centering
\rowcolors{2}{white}{gray!25}
\begin{tabular}{ L{4.5cm} L{1.5cm} L{1.5cm} L{1.5cm} L{1.5cm} L{1.5cm} }
\hline
\scriptsize
\centering
\rowcolors{2}{white}{gray!25}
\begin{tabular}{ L{4.5cm} L{1.5cm} L{1.5cm} L{1.5cm} L{1.5cm} L{1.5cm} }
\hline
\input{gen/tables/devices}
\hline
\end{tabular}

\hline
\end{tabular}

\hline
\end{tabular}

\caption{%
  Specification of experimental OpenCL devices.%
}
\label{tab:hw}
\end{table}

\begin{table}
\scriptsize
\centering
\rowcolors{2}{white}{gray!25}
\begin{tabular}{lrrrrr}
\toprule
      Name &  North &  South &  East &  West &  Instruction Count \\
\midrule
   % complex &     30 &     30 &    30 &    30 &                161 \\
   % complex &      1 &     10 &    30 &    30 &                681 \\
   % complex &     20 &     10 &    20 &    10 &                161 \\
   % complex &      5 &      5 &     5 &     5 &                734 \\
   % complex &      5 &      5 &     5 &     5 &                161 \\
   % complex &      1 &     10 &    30 &    30 &                154 \\
   % complex &     10 &     10 &    10 &    10 &                161 \\
   % complex &     20 &     20 &    20 &    20 &                706 \\
   % complex &      1 &      1 &     1 &     1 &                137 \\
   % complex &     20 &     10 &    20 &    10 &                706 \\
   % complex &      1 &      1 &     1 &     1 &                661 \\
   % complex &     10 &     10 &    10 &    10 &                794 \\
   % complex &     30 &     30 &    30 &    30 &                706 \\
   % complex &     20 &     20 &    20 &    20 &                161 \\
   %  simple &      5 &      5 &     5 &     5 &                 67 \\
   %  simple &     20 &     10 &    20 &    10 &                612 \\
   %  simple &     20 &     20 &    20 &    20 &                612 \\
   %  simple &      1 &     10 &    30 &    30 &                592 \\
   %  simple &     10 &     10 &    10 &    10 &                700 \\
   %  simple &     30 &     30 &    30 &    30 &                612 \\
   %  simple &      1 &      1 &     1 &     1 &                 93 \\
   %  simple &     30 &     30 &    30 &    30 &                 67 \\
   %  simple &     20 &     10 &    20 &    10 &                 67 \\
   %  simple &      5 &      5 &     5 &     5 &                640 \\
   %  simple &     20 &     20 &    20 &    20 &                 67 \\
   %  simple &     10 &     10 &    10 &    10 &                 67 \\
   %  simple &      0 &      0 &     0 &     0 &                 40 \\
   %  simple &      1 &     10 &    30 &    30 &                 65 \\
   %  simple &      1 &      1 &     1 &     1 &                617 \\
   synthetic & 1--30 & 1--30 & 1--30 & 1--30 & 67--137\\
   synthetic & 1--30 & 1--30 & 1--30 & 1--30 & 592--706\\
   gaussian &      1--10 &      1--10 &     1--10 &     1--10 & 82--83 \\
  % gaussian &      5 &      5 &     5 &     5 &                655 \\
  % gaussian &      5 &      5 &     5 &     5 &                657 \\
  % gaussian &      0 &      0 &     0 &     0 &                 46 \\
  % gaussian &      5 &      5 &     5 &     5 &                 82 \\
  % gol &      1 &      1 &     1 &     1 &                714 \\
   gol &      1 &      1 &     1 &     1 &                190 \\
   he  &      1 &      1 &     1 &     1 &                113 \\
% he &      1 &      1 &     1 &     1 &                637 \\
%nms &      1 &      1 &     1 &     1 &                748 \\
   nms &      1 &      1 &     1 &     1 &                224 \\
%     sobel &      1 &      1 &     1 &     1 &               1008 \\
   sobel &      1 &      1 &     1 &     1 &                246 \\
% threshold &      0 &      0 &     0 &     0 &                 16 \\
   threshold &      0 &      0 &     0 &     0 &                 46 \\
\bottomrule
\end{tabular}

\caption{%
  Benchmark applications, border sizes, and static instruction counts.
  The ``simple'' and ``complex'' kernels are synthetic training
  programs. \FIXME{There's some suspicious duplicate entries
    here\ldots} \TODO{Add a ``stencil shape'' label above north
    south east west columns.}%
  % \FIXME{I also have a FDTD benchmark which I have yet to
  % collect results for.}
}
\label{tab:kernels}
\end{table}

\begin{table}
\footnotesize
\centering
\rowcolors{2}{white}{gray!25}
\begin{tabular}{ l l l l }
\hline
\footnotesize
\centering
\rowcolors{2}{white}{gray!25}
\begin{tabular}{ l l l l }
\hline
\footnotesize
\centering
\rowcolors{2}{white}{gray!25}
\begin{tabular}{ l l l l }
\hline
\input{gen/tables/datasets}
\hline
\end{tabular}

\hline
\end{tabular}

\hline
\end{tabular}

\caption{%
  Datasets used.%
}
\label{tab:datasets}
\end{table}


\subsection{Sampling Runtimes}

The number of ``moving parts'' in the modern software stack provides
multiple sources of noise when measuring program execution times. As
such, evaluating the relative performance of different versions of
programs requires a judicious approach to isolate the appropriate
performance metrics and to take a statistically rigorous approach to
collecting data.


\subsubsection{Isolating the Impact of Workgroup Size}

The execution of a SkelCL stencil application can be divided into 6
distinct phases, shown in Table~\ref{tab:stencil-runtime-components}.

\TODO{Include a flow diagram}

\begin{table}
\footnotesize
\centering
\rowcolors{2}{white}{gray!25}
\begin{tabular}{ l l l l }
  \hline
  & \textbf{Description} & \textbf{Type} & \textbf{Cost}\\
  \hline
  $\bm{c}$ & Kernel compilation times & Host & Fixed \\
  $\bm{p}$ & Skeleton prepare times & Host & Fixed \\
  $\bm{u}$ & Host $\rightarrow$ Device transfers & Device & Fixed \\
  $\bm{k}$ & Kernel execution times & Device & Iterative \\
  $\bm{d}$ & Device $\rightarrow$ Host transfers & Device & Fixed \\
  $\bm{s}$ & Devices $\leftrightarrow$ Host (sync) transfers & Host & Iterative \\
  \hline
\end{tabular}

\caption{Execution phases of a SkelCL stencil skeleton. ``Fixed''
  costs are those which occur up to once per stencil
  invocation. ``Iterative'' costs are those which scale with the
  number of iterations of a stencil.}
\label{tab:stencil-runtime-components}
\end{table}

\paragraph{Kernel compilation times} Upon invocation, template
substitution is performed of the user code into the stencil skeleton
implementation, then compiled into an OpenCL program. Once compiled,
the program binary is cached for the lifetime of the host program.

\paragraph{Skeleton prepare times} Before a kernel is executed, a
preparation phase is required to allocate buffers for the input and
output data on each execution device.

\paragraph{Host $\rightarrow$ Device and Device $\rightarrow$ Host
  transfers} Data must be copied to and from the execution devices
before and after execution of the stencils, respectively. Note that
this is performed lazily, so iterative stencils do not require
repeated transfers between host and device memory.

\paragraph{Kernel execution times} This is the time elapsed executing
the stencil kernel, and is representative of ``work done''.

\paragraph{Devices $\leftrightarrow$ Host (sync) transfers} For
iterative stencils on multiple execution devices, an overlapping halo
region is shared at the border between the devices' grids. This must
be synchronised between iterations, requiring an intermediate transfer
to host memory, since device to device memory is not currently
supported by OpenCL.

For each of the six distinct phases of execution, accurate runtime
information can be gathered either through timers embedded in the host
code, or using the OpenCL \texttt{clGetEventProfilingInfo()} API for
operations on the execution devices. For single-device stencils, the
total time $t$ of a SkelCL stencil application is simply the sum of
all times recorded for each distinct phase:

\begin{equation}
t = \bm{1c} + \bm{1p} + \bm{1u} + \bm{1k} + \bm{1d}
\end{equation}

Note that there are no synchronisation costs $s$. For applications
with $n$ execution devices, the runtime can be approximate as the sum
of the sequential host-side phases, and the sum of the device-side
phases divided by the number of devices:

\begin{equation}
t \approx \sum_{i=1}^n{(\bm{1c}_{i})} + \bm{1p} + \bm{1s} +
  \frac{\sum_{i=1}^n{\bm{1u}_{i} + \bm{1k}_{i} + \bm{1d}_{i}}}{n}
\end{equation}

\TODO{Compare against empirical runtime information, showing how
  accurate this estimation is.}

The purpose of tuning workgroup size is to maximise the throughput of
stencil kernels. For this reason, isolating the kernel execution times
$\bm{k}$ produces the most accurate performance comparisons, as it
removes the impact of constant overheads introduced by memory
transfers between host and device memory, for which the selection of
workgroup size has no influence. Note that as demonstrated
in~\cite{Gregg2011}, care must be taken to ensure that isolating
device compute time does not cause misleading comparisons to be made
between devices. For example, if using an autotuner to determine
whether execution of a given stencil is faster on a CPU or GPU, the
device transfer times $\bm{u}$, $\bm{d}$, and $\bm{s}$ would need to
be considered. For our purposes, we do not need to consider the
location of the data in the system's memory as it is has no bearing on
the execution time of a stencil kernel.


\subsubsection{Sampling plan and technique}

All runtimes were recorded with millisecond precision using either the
system clock or OpenCL's Profiling API. Measurement noise was
minimised by reducing system load through disabling all unwanted
services and graphical environments, and exclusive single-user access
was ensured for each machine. Frequency governors for each CPU were
disabled, and the benchmark processes were set to the highest priority
available to the task scheduler. Datasets and programs were stored in
an in-memory file system. Samples were collected during a training
phase in which OmniTune uniformly repeatedly samples the space of
legal workgroup sizes.


\section{Summary}

This section describes the methodology for collecting performance data
of \input{gen/num_runtime_stats} combinations of architectures,
programs, datasets, and workgroup sizes.
