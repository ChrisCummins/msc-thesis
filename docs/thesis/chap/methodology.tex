% METHODOLOGY
% ===========
%
% Description of the work undertaken: this may be divided into
% chapters describing the conceptual design work and the actual
% implementation separately. Any problems or difficulties and the
% suggested solutions should be mentioned. Alternative solutions and
% their evaluation should also be included.

\begin{table}
\footnotesize
\centering
\begin{tabular}{| L{7.5cm} | L{1.5cm} | L{1.5cm} | L{1.5cm} | L{1.5cm} | L{1.5cm} |}
\hline
\scriptsize
\centering
\rowcolors{2}{white}{gray!25}
\begin{tabular}{ L{4.5cm} L{1.5cm} L{1.5cm} L{1.5cm} L{1.5cm} L{1.5cm} }
\hline
\scriptsize
\centering
\rowcolors{2}{white}{gray!25}
\begin{tabular}{ L{4.5cm} L{1.5cm} L{1.5cm} L{1.5cm} L{1.5cm} L{1.5cm} }
\hline
\scriptsize
\centering
\rowcolors{2}{white}{gray!25}
\begin{tabular}{ L{4.5cm} L{1.5cm} L{1.5cm} L{1.5cm} L{1.5cm} L{1.5cm} }
\hline
\input{gen/tables/devices}
\hline
\end{tabular}

\hline
\end{tabular}

\hline
\end{tabular}

\hline
\end{tabular}
\caption{%
  Execution devices. \FIXME{Missing data from CPUs!}%
}
\label{tab:hw}
\end{table}

\begin{table}
\footnotesize
\centering
\begin{tabular}{| l | l | l | l | l | l |}
\hline
\scriptsize
\centering
\rowcolors{2}{white}{gray!25}
\begin{tabular}{lrrrrr}
\toprule
      Name &  North &  South &  East &  West &  Instruction Count \\
\midrule
   % complex &     30 &     30 &    30 &    30 &                161 \\
   % complex &      1 &     10 &    30 &    30 &                681 \\
   % complex &     20 &     10 &    20 &    10 &                161 \\
   % complex &      5 &      5 &     5 &     5 &                734 \\
   % complex &      5 &      5 &     5 &     5 &                161 \\
   % complex &      1 &     10 &    30 &    30 &                154 \\
   % complex &     10 &     10 &    10 &    10 &                161 \\
   % complex &     20 &     20 &    20 &    20 &                706 \\
   % complex &      1 &      1 &     1 &     1 &                137 \\
   % complex &     20 &     10 &    20 &    10 &                706 \\
   % complex &      1 &      1 &     1 &     1 &                661 \\
   % complex &     10 &     10 &    10 &    10 &                794 \\
   % complex &     30 &     30 &    30 &    30 &                706 \\
   % complex &     20 &     20 &    20 &    20 &                161 \\
   %  simple &      5 &      5 &     5 &     5 &                 67 \\
   %  simple &     20 &     10 &    20 &    10 &                612 \\
   %  simple &     20 &     20 &    20 &    20 &                612 \\
   %  simple &      1 &     10 &    30 &    30 &                592 \\
   %  simple &     10 &     10 &    10 &    10 &                700 \\
   %  simple &     30 &     30 &    30 &    30 &                612 \\
   %  simple &      1 &      1 &     1 &     1 &                 93 \\
   %  simple &     30 &     30 &    30 &    30 &                 67 \\
   %  simple &     20 &     10 &    20 &    10 &                 67 \\
   %  simple &      5 &      5 &     5 &     5 &                640 \\
   %  simple &     20 &     20 &    20 &    20 &                 67 \\
   %  simple &     10 &     10 &    10 &    10 &                 67 \\
   %  simple &      0 &      0 &     0 &     0 &                 40 \\
   %  simple &      1 &     10 &    30 &    30 &                 65 \\
   %  simple &      1 &      1 &     1 &     1 &                617 \\
   synthetic & 1--30 & 1--30 & 1--30 & 1--30 & 67--137\\
   synthetic & 1--30 & 1--30 & 1--30 & 1--30 & 592--706\\
   gaussian &      1--10 &      1--10 &     1--10 &     1--10 & 82--83 \\
  % gaussian &      5 &      5 &     5 &     5 &                655 \\
  % gaussian &      5 &      5 &     5 &     5 &                657 \\
  % gaussian &      0 &      0 &     0 &     0 &                 46 \\
  % gaussian &      5 &      5 &     5 &     5 &                 82 \\
  % gol &      1 &      1 &     1 &     1 &                714 \\
   gol &      1 &      1 &     1 &     1 &                190 \\
   he  &      1 &      1 &     1 &     1 &                113 \\
% he &      1 &      1 &     1 &     1 &                637 \\
%nms &      1 &      1 &     1 &     1 &                748 \\
   nms &      1 &      1 &     1 &     1 &                224 \\
%     sobel &      1 &      1 &     1 &     1 &               1008 \\
   sobel &      1 &      1 &     1 &     1 &                246 \\
% threshold &      0 &      0 &     0 &     0 &                 16 \\
   threshold &      0 &      0 &     0 &     0 &                 46 \\
\bottomrule
\end{tabular}

\hline
\end{tabular}
\caption{%
  Benchmark applications, border sizes, and static instruction counts.
  The ``simple'' and ``complex'' kernels are synthetic training
  programs. \FIXME{I also have a FDTD benchmark which I have yet to
    collect results for.}%
}
\label{tab:kernels}
\end{table}

\begin{table}
\footnotesize
\centering
\begin{tabular}{| l | l | l | l |}
\hline
\footnotesize
\centering
\rowcolors{2}{white}{gray!25}
\begin{tabular}{ l l l l }
\hline
\footnotesize
\centering
\rowcolors{2}{white}{gray!25}
\begin{tabular}{ l l l l }
\hline
\footnotesize
\centering
\rowcolors{2}{white}{gray!25}
\begin{tabular}{ l l l l }
\hline
\input{gen/tables/datasets}
\hline
\end{tabular}

\hline
\end{tabular}

\hline
\end{tabular}

\hline
\end{tabular}
\caption{%
  Datasets used.%
}
\label{tab:datasets}
\end{table}

Tables~\ref{tab:hw},~\ref{tab:kernels}, and~\ref{tab:datasets} list
the range of execution devices, kernels, and datasets used. For each
unique combination of architecture, kernel, and dataset (hereby
referred to as a \emph{scenario}), training data was collected by
randomly sampling the space of legal workgroup sizes, until multiple
samples have been collected for each combination of scenario and
workgroup size.

% Leather, H., O’Boyle, M., & Worton, B. (2009). Raced Profiles:
% Efficient Selection of Competing Compiler Optimizations. In LCTES
% ’09: Proceedings of the ACM SIGPLAN/SIGBED 2009 Conference on
% Languages, Compilers, and Tools for Embedded Systems
% (pp. 1–10). Dublin.
\TODO{%
  Using adapting sampling plans to reduce the number of samples
  required to distinguish good from bad compiler
  configurations~\cite{Leather2009}.%
}

\begin{table}
\footnotesize
\centering
\begin{tabular}{| l | l |}
  \hline
  & \textbf{Descriptionr}\\
  \hline
  $\bm{c}$ & Kernel compilation times \\
  $\bm{p}$ & Skeleton prepare times \\
  $\bm{u}$ & Host $\rightarrow$ Device transfers \\
  $\bm{k}$ & Kernel execution times \\
  $\bm{d}$ & Device $\rightarrow$ Host transfers \\
  $\bm{s}$ & Devices $\leftrightarrow$ Host (sync) transfers \\
  \hline
\end{tabular}
\caption{Measurable performance values.}
\label{tab:metric}
\end{table}

\TODO{%
  Approximating the total time of skeletons based on SkelCL profiling
  times:
\[t \approx \sum_{i=1}^n{\bm{1c}_{i}} + \bm{1p} + \bm{1s} +
  \frac{\sum_{i=1}^n{\bm{1u}_{i} + \bm{1k}_{i} + \bm{1d}_{i}}}{n} \]
}

\TODO{I'm only tuning GPU-side runtimes, which could be
  misleading~\cite{Gregg2011}}.

\begin{table}
\footnotesize
\centering
\begin{tabular}{| l | l | l | l | l |}
\hline
\textbf{Name} & \textbf{Application} & \textbf{Skeletons used} & \textbf{Iterative?} & \textbf{LOC}\\
\hline
CannyEdgeDetection & Image processing & Stencil & - & 225 / 61\\
DotProduct & Linear algebra & Zip, Reduce & - & 143 / 2\\
FDTD & Scientific simulation & Map, Stencil & Y & 375 / 127\\
GameOfLife & Cellular automata & Stencil & Y & 92 / 12\\
GaussianBlur & Image processing & Stencil & - & 262 / 47\\
HeatSimulation & Scientific simulation & Stencil & Y & 180 / 13\\
MandelbrotSet & Fractal computation & Map & Y & 133 / 78\\
MatrixMultiply & Linear algebra & AllPairs & - & 267 / 8\\
SAXPY & Linear algebra & Zip & - & 149 / 3\\
\hline
\end{tabular}
\caption{Benchmark applications. The LOC column shows lines of code, split between host (C++) and device (OpenCL).}
\label{tab:benchmarks}
\end{table}

\begin{table}
\footnotesize
\centering
\begin{tabular}{| l | l | l |}
\hline
\textbf{Parameter} & \textbf{Values} & \textbf{Skeleton}\\
\hline
Number of columns, rows, segments & \{8, 16, 32, 64, 128\} & AllPairs\\
Global size & \{256, 512, 1024, 2048, \ldots, 2097152\} & Reduce\\
Work group size & [32\ldots$n$] & *\\
Work group size (2D) & \{8, 16, 32, 64, 128, 256\} & Stencil\\
Device type & \{CPU, GPU\} & *\\
Device count & [1,4] & *\\
Halo region size & [1\ldots$n$] & Stencil\\
Implementation of stencil operation & {MapOverlap, Stencil} & Stencil\\
Border loading strategy & TODO & Stencil\\
Thread coarsening factor & TODO & *\\
\hline
\end{tabular}
\caption{Tunable parameters.}
\label{tab:knobs}
\end{table}

\begin{table}
\footnotesize
\centering
\begin{tabular}{| l | l | l | l |}
\hline
\textbf{CPU} & \textbf{Memory} & \textbf{GPU}\\
\hline
Intel i7-4770 & 16GiB & NVIDIA GTX TITAN\\
Intel i7-2600K & 16GiB & NVIDIA GTX 690\\
Intel i7-2600K & 8GiB & 2$\times$ NVIDIA GTX 590\\
Intel i7-3820 & 8GiB & 2$\times$ AMD Tahiti 7970\\
Intel i5-4570 & 8GiB & -\\
\hline
\end{tabular}
\caption{Testing hardware.}
\label{tab:hw}
\end{table}

\TODO{Table of derived dependent variables (e.g. FLOPS, GPU utilisation)}

\newpage

\begin{algorithm}
\caption{Request workgroup size}\label{bar}
\begin{algorithmic}[1]
\Require kernel features $k$, hardware features $h$, dataset features $d$.
\Ensure workgroup size $c$

\State $C \leftarrow \{ c_1, c_2,\ldots, c_n \}$
\Comment Set of all possible workgroup sizes

\If{no control flow in kernel}
    \State \textbf{return} $\underset{c}{\argmin} f(k,h,d,c) = r$
\Else
   \State converged $\leftarrow$ false
   \State $c_b \leftarrow$ baseline values
   \State $r_b \leftarrow$ measure runtime of runtime of program with $c_b$
   \While{not converged}
     \State return $\underset{c}{\argmax} g(k,h,d,c) = s$
     \State evaluate $c$, measuring runtime $r$\;
     \If{measured speedup $\frac{r_b}{r} \approx$ predicted speedup $s$}
       \State converged = true
     \Else
       \State $C = C - \{c\}$
     \EndIf
   \EndWhile
   \State \textbf{return} $c$
\EndIf
\end{algorithmic}
\end{algorithm}


\subsection{Oracle performance}
For a given scenario $s$ and workgroup size $w$, the arithmetic mean
of measured runtimes if $t(s,w)$. From the set of all scenarios $S$
and workgroup sizes $W$, the subset of workgroup scenarios
$W_{legal}(S) \subset W$ which are legal for a given scenario can be
found using:

\[ W_{legal}(s) = \left\{w | w \in W, w < W_{max}(s) \right\} \]

The arithmetic mean of the measured runtimes for a given scenario $s$
and workgroup size $w$ is represented by $t(s,w)$. The oracle
workgroup size $\Omega(s) \in W_{legal}(s)$ is the $w$ value which
minimises the value of $t(s,w)$:

\[\Omega(s) = \argmin_{w \in W} t(s,w) \]

This allows relative comparisons of performance of $w$ values against
the oracle:

\[ p(s,w) = \frac{t(s,\Omega(s))}{t(s,w)} \]

Where the performance is within the range $0 \le p(s,w) \le 1$. For a
given workgroup size, the average performance $\bar{p}(w)$ can be
found using the geometric mean of performance relative to the oracle
across all scenarios:

\[ \bar{p}(w) = \left(\prod_{s \in S} t(s,\Omega(s)) \cdot t(s,w)^{-1} \right)^{1/|S|} \]

\subsection{Autotuning using classification}
The simplest autotuner is one which selects the workgroup size which
is most commonly optimal (i.e. the mode of all optimal workgroup sizes
$\left\{ \Omega(s) | s \in S \right\}$):

\[ \text{ZeroR} = \text{mode}( \left\{ \Omega(s) | s \in S \right\} ) \]

However, this is not satisfactory for the purpose of selecting a
workgroup size to use, due to the constraint $W_{max}(s)$ enforced for
each scenario. As a result, we propose a so called ``OneR''
classifier, which first defines $W_{safe} \in W$ as the intersection
of legal workgroup sizes across all scenarios, then selects the
workgroup size $w$ which maximises the average performance
$\bar{p}(w)$ across this reduced parameter space.

\[W_{safe} = \cap \left\{ f(s) | s \in S \right\} \]
\[ \text{OneR} = \argmax_{w \in W_{safe}} \bar{p}(w) \]

This provides a baseline for comparing against a more sophisticated
autotuning approaches using machine learning.

\subsection{Feature Extraction}

For each scenario, a feature vector is extracted to capture properties
of the architecture, device, and dataset:

\begin{itemize}
\item \emph{Architectural features} --- size of local memory, maximum
  work group size, number of compute units, etc. Accessed using the
  OpenCL \texttt{clGetDeviceInfo()} API.
\item \emph{Kernel features} --- total static instruction count, ratio
  of instructions per type, ratio of basic blocks per instruction,
  etc. Accessed by compiling the OpenCL kernel to LLVM IR bitcode, and
  using the \texttt{opt} \texttt{InstCount} statistics pass.
\item \emph{Dataset features} --- size and type of the
  dataset. Accessed from the SkelCL Matrix container type.
\end{itemize}

See Appendix~\ref{app:features} for a full list of features and
types. For training, feature vectors are labelled with the oracle
workgroup size, and a classifier is trained on a subset of this
labelled training data. The performance of the classifier is evaluated
by comparing the performance of the workgroup size predicted for an
unseen feature vector against the oracle workgroup size for that
feature.

\subsubsection{Satisfying the maximum workgroup size constraint}

Since classifiers are probabilistic systems, it is possible that a
classifier will predict a workgroup size that is invalid for the given
scenario, $w \not\in W_{legal}(s)$. In these cases, one of three
fallback strategies is used to select a safe workgroup size:

\begin{enumerate}
\item \emph{OneR} --- select the workgroup size which is known to be
  safe and provides the highest average case performance.
\item \emph{Random} --- select a random workgroup size uniformly from
  the set of legal values $w \in W_{legal}$.
\item \emph{Reshape} --- attempt to scale predicted the predicted
  workgroup size proportionally so that it fits within the space of
  legal workgroup sizes.
\end{enumerate}

The evaluation compares the average performance achieved using each
fallback strategy, along with the percentage of cases for which these
fallback strategies were required.
