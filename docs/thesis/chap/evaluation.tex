\section{Introduction}

This chapter contains an evaluation of the performance of OmniTune
when tasked with selecting workgroup size of SkelCL stencil codes. The
effectiveness of each of the autotuning techniques described in the
previous chapter are scrutinised and evaluated using multiple
different machine learning algorithms, and the prediction quality is
evaluated for portability across programs, devices, and datasets.


\subsubsection{Questions to ask of each approach to autotuning}

\begin{enumerate}
\item How successful is it at selecting optimal workgroup sizes?
\item How successful is it at selecting \emph{good} workgroup sizes?
\item What is the performance when training on synthetic benchmarks
  and testing on real?
\item What is the performance when testing on an unseen: architecture,
  kernel, dataset? (leave one out evaluation)
\item How well does it perform as a function of the number of training
  points?
\end{enumerate}

\section{Classification Performance}

\begin{enumerate}
\item How often does it select \emph{invalid} workgroup sizes?
\item How does the three different techniques of handling invalid
  workgroup sizes perform?
\end{enumerate}

\begin{figure}
\centering
\includegraphics[width=.75\textwidth]{../../experiments/205-06-04-train/img/classification/xval.png}
\caption{%
  Classification performance using cross-validation.%
}
\end{figure}

\begin{figure}
\centering
\includegraphics[width=.8\textwidth]{../../experiments/205-06-04-train/img/classification/xval/err_fns/random_fn.png}
\caption{%
  Classification performance for all test cases.%
}
\end{figure}


% \TODO{Rank eigenvectors of PCA on features.}

% \TODO{Evaluate the effectiveness of training with synthetic
%   benchmarks.}


\section{Predicting Stencil Runtime}

\begin{enumerate}
\item How accurately does it predict the runtime of a stencil?
\item What is the relationship between \emph{classification} accuracy,
  and the accuracy of predicted runtimes? Does it really matter if the
  predicted runtime is inaccurate?
\end{enumerate}

\begin{figure}
\centering
\includegraphics[width=.75\textwidth]{../../experiments/205-06-04-train/img/runtime_regression/xval.png}
\caption{%
  Performance of regressors at predicting program runtime.%
}
\end{figure}


\section{Predicting Relative Performance}


\section{Summary}

\begin{table}
\scriptsize
\input{../../experiments/205-06-04-train/xval.tex}
\caption{Results of 10 cross-validation.}
\end{table}

\begin{table}
\scriptsize
\input{../../experiments/205-06-04-train/synthetic_real.tex}
\caption{Results of training using synthetic benchmarks and testing on
  real.}
\end{table}
