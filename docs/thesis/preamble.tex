% Path to external resources:
%
\newcommand{\DataDir}{/home/chris/data/msc-thesis/2015-06-23}

%%%%%%%%%%%%%%%%%%%%%%%%%
%% Document and Layout %%
%%%%%%%%%%%%%%%%%%%%%%%%%

% Fix for multiple "No room for a new \dimen" errors.
%
% See: http://tex.stackexchange.com/questions/38607/no-room-for-a-new-dimen
%
\usepackage{etex}

\usepackage[utf8]{inputenc}

% Fix for "'babel/polyglossia' detected but 'csquotes' missing"
% warning. NOTE: Include after inputenc.
%
\usepackage{csquotes}

% Make internal macro definitions accessible,
% e.g. \@title, \@date \@author.
\makeatletter

% Multi-column support.
\usepackage{multicol}

% A useful package which includes macros like \ifdef{}{}{}:
%
\usepackage{etoolbox}

% Uncomment the following line to remove column separation:
%
%\setlength{\columnsep}{5mm}

% Allow user-defined warning and error filters.
%
\usepackage{silence}


%%%%%%%%%%%%%%%%%%%%%
% Table of Contents %
%%%%%%%%%%%%%%%%%%%%%

% Set chapter and section numbering depth:
%
\setcounter{secnumdepth}{2}


%%%%%%%%%%%%%%%%
% Bibliography %
%%%%%%%%%%%%%%%%
\usepackage[%
    backend=biber,
    style=numeric-comp,  % numerical-compressed
    sorting=none,        % nty,nyt,nyvt,anyt,anyvt,ynt,ydnt,none
    sortcites=true,      % sort \cite{b a d c}: true,false
    block=none,          % space between blocks: none,space,par,nbpar,ragged
    indexing=false,      % indexing options: true,false,cite,bib
    citereset=none,      % don't reset cites
    isbn=false,          % print ISBN?
    url=false,           % print URL?
    doi=false,           % print DOI?
    natbib=true,         % natbib compatability
  ]{biblatex}

% Filter annoying and unavoidable biblatex warning:
\WarningFilter{biblatex}{Patching footnotes failed}

% Reduce the font size of the bibliography:
% \renewcommand{\bibfont}{\normalfont\scriptsize}

% Determine which BibTeX file to use:
%
% If available, use my Mendeley BibTex library, located in the home
% directory. Note that this is a relative path and will break if
% either this file or the BibTex library are moved. If the library is
% not present, use the local refs.bib file.
\newcommand{\BibResourceGlobal}{../../../../library.bib}
\newcommand{\BibResourceLocal}{refs.bib}

\IfFileExists{\BibResourceGlobal}
  {\newcommand{\BibResource}{\BibResourceGlobal}}
  {\newcommand{\BibResource}{\BibResourceLocal}}

\addbibresource{\BibResource}


%%%%%%%%%%%%%%
% Appendices %
%%%%%%%%%%%%%%

% Appendix package. Documentation:
%
%  http://mirror.ox.ac.uk/sites/ctan.org/macros/latex/contrib/appendix/appendix.pdf
%
% Package options:
%
% toc      - Put a header (e.g., `Appendices') into the Table of Contents
%            (the ToC) before listing the appendices. (This is done by
%            calling the \addappheadtotoc command.)
% page     - Puts a title (e.g., `Appendices') into the document at the
%            point where the appendices environment is begun. (This is
%            done by calling the \appendixpage command.)
% title    - Adds a name (e.g., `Appendix') before each appendix title in
%            the body of the document. The name is given by the value
%            of \appendixname. Note that this is the default behaviour
%            for classes that have chapters.
% titletoc - Adds a name (e.g., `Appendix') before each appendix listed
%            in the ToC. The name is given by the value
%            of \appendixname.
% header   - Adds a name (e.g., `Appendix') before each appendix in page
%            headers.  The name is given by the value
%            of \appendixname. Note that this is the default behaviour
%            for classes that have chapters.
\usepackage[title, titletoc]{appendix}


%%%%%%%%%%%%%%%%%%%%%%%%%%%%%%%%%%%%%
%% Figures, footnotes and listings %%
%%%%%%%%%%%%%%%%%%%%%%%%%%%%%%%%%%%%%

%\usepackage{float}
%\restylefloat{figure}

% Use bold ``(Figure|Table|Listing)'' caption text.
%\usepackage[margin=1cm]{caption}

% Set the font for captions.
%\renewcommand{\captionfont}{\footnotesize}
% Set the font for caption labels.
%\renewcommand{\captionlabelfont}{\footnotesize\bf}

% Use arabic numbers for footnote.
%\renewcommand{\thefootnote}{\arabic{footnote}}

% Ensure that footnotes always appear at the bottom of pages.
%\usepackage[bottom]{footmisc}

% Reset the footnote counter on every page.
%\usepackage{perpage}
%\MakePerPage{footnote}

% Pre-requisites for rendering upquotes in listings package.
\usepackage[T1]{fontenc}
\usepackage{lmodern}
\usepackage{textcomp}

% Pseudo-code listings.
\usepackage{algorithm}
\usepackage{algpseudocode}
\newcommand{\Break}{\State \textbf{break} }
\algblockdefx[Loop]{Loop}{EndLoop}[1][]{\textbf{Loop} #1}{\textbf{End
    Loop}}

\algrenewcommand\ALG@beginalgorithmic{\footnotesize}

% Code listings.
\usepackage{listings}

% Set \ttfamily to use courier fonts.
%
% See: http://tex.stackexchange.com/a/33686
%
\usepackage{courier}

\lstset{frame=bt,                    % Add top and bottom frame lines
        breaklines=true,             % Force line wrapping
        captionpos=b,                % Place caption below listing
        numbers=left,                % Add left-side line numbers
        basicstyle=\scriptsize\ttfamily, % Set font size and type
        showstringspaces=false,      % Don't show visible whitespace
        numberstyle=\tiny,
        upquote=true,                % Use upright quotes, not curly
        commentstyle=\bfseries}      % Embolden comments

% Use (*@ @*) to escape LaTeX commands within listings.
\lstset{escapeinside={(*@}{@*)}}

% Add 10pt space between chapters in TOC listings entries:
%\let\Chapter\chapter
%\def\chapter{\addtocontents{lol}{\protect\addvspace{10pt}}\Chapter}

% Add JavaScript support to listings. See:
%
%     http://tex.stackexchange.com/a/89576
%
\lstdefinelanguage{JavaScript}{
  keywords={
    break,
    case,
    catch,
    catch,
    do,
    else,
    false,
    function,
    if,
    in,
    new,
    null,
    return,
    switch,
    true,
    typeof,
    var,
    while}
  keywordstyle=\bfseries,
  ndkeywords={
    boolean,
    class,
    export,
    implements,
    import,
    this,
    throw}
  ndkeywordstyle=\bfseries,
  sensitive=false,
  comment=[l]{//},
  morecomment=[s]{/*}{*/},
  morestring=[b]',
  morestring=[b]"
}

% Add Clojure support to listings. See:
%
%     http://alexott.blogspot.co.uk/2010/01/clojure-latex.html
%
\lstdefinelanguage{Clojure}{morekeywords={
    *,
    *1,
    *2,
    *3,
    *agent*,
    *allow-unresolved-vars*,
    *assert*,
    *clojure-version*,
    *command-line-args*,
    *compile-files*,
    *compile-path*,
    *e,
    *err*,
    *file*,
    *flush-on-newline*,
    *in*,
    *macro-meta*,
    *math-context*,
    *ns*,
    *out*,
    *print-dup*,
    *print-length*,
    *print-level*,
    *print-meta*,
    *print-readably*,
    *read-eval*,
    *source-path*,
    *use-context-classloader*,
    *warn-on-reflection*,
    +,
    -,
    ->,
    ->>,
    ..,
    /,
    :else,
    <,
    <=,
    =,
    ==,
    >,
    >=,
    @,
    accessor,
    aclone,
    add-classpath,
    add-watch,
    agent,
    agent-errors,
    aget,
    alength,
    alias,
    all-ns,
    alter,
    alter-meta!,
    alter-var-root,
    amap,
    ancestors,
    and,
    apply,
    areduce,
    array-map,
    aset,
    aset-boolean,
    aset-byte,
    aset-char,
    aset-double,
    aset-float,
    aset-int,
    aset-long,
    aset-short,
    assert,
    assoc,
    assoc!,
    assoc-in,
    associative?,
    atom,
    await,
    await-for,
    await1,
    bases,
    bean,
    bigdec,
    bigint,
    binding,
    bit-and,
    bit-and-not,
    bit-clear,
    bit-flip,
    bit-not,
    bit-or,
    bit-set,
    bit-shift-left,
    bit-shift-right,
    bit-test,
    bit-xor,
    boolean,
    boolean-array,
    booleans,
    bound-fn,
    bound-fn*,
    butlast,
    byte,
    byte-array,
    bytes,
    cast,
    char,
    char-array,
    char-escape-string,
    char-name-string,
    char?,
    chars,
    chunk,
    chunk-append,
    chunk-buffer,
    chunk-cons,
    chunk-first,
    chunk-next,
    chunk-rest,
    chunked-seq?,
    class,
    class?,
    clear-agent-errors,
    clojure-version,
    coll?,
    comment,
    commute,
    comp,
    comparator,
    compare,
    compare-and-set!,
    compile,
    complement,
    concat,
    cond,
    condp,
    conj,
    conj!,
    cons,
    constantly,
    construct-proxy,
    contains?,
    count,
    counted?,
    create-ns,
    create-struct,
    cycle,
    dec,
    decimal?,
    declare,
    def,
    definline,
    defmacro,
    defmethod,
    defmulti,
    defn,
    defn-,
    defonce,
    defprotocol,
    defstruct,
    deftype,
    delay,
    delay?,
    deliver,
    deref,
    derive,
    descendants,
    destructure,
    disj,
    disj!,
    dissoc,
    dissoc!,
    distinct,
    distinct?,
    do,
    do-template,
    doall,
    doc,
    dorun,
    doseq,
    dosync,
    dotimes,
    doto,
    double,
    double-array,
    doubles,
    drop,
    drop-last,
    drop-while,
    empty,
    empty?,
    ensure,
    enumeration-seq,
    eval,
    even?,
    every?,
    false,
    false?,
    ffirst,
    file-seq,
    filter,
    finally,
    find,
    find-doc,
    find-ns,
    find-var,
    first,
    float,
    float-array,
    float?,
    floats,
    flush,
    fn,
    fn?,
    fnext,
    for,
    force,
    format,
    future,
    future-call,
    future-cancel,
    future-cancelled?,
    future-done?,
    future?,
    gen-class,
    gen-interface,
    gensym,
    get,
    get-in,
    get-method,
    get-proxy-class,
    get-thread-bindings,
    get-validator,
    hash,
    hash-map,
    hash-set,
    identical?,
    identity,
    if,
    if-let,
    if-not,
    ifn?,
    import,
    in-ns,
    inc,
    init-proxy,
    instance?,
    int,
    int-array,
    integer?,
    interleave,
    intern,
    interpose,
    into,
    into-array,
    ints,
    io!,
    isa?,
    iterate,
    iterator-seq,
    juxt,
    key,
    keys,
    keyword,
    keyword?,
    last,
    lazy-cat,
    lazy-seq,
    let,
    letfn,
    line-seq,
    list,
    list*,
    list?,
    load,
    load-file,
    load-reader,
    load-string,
    loaded-libs,
    locking,
    long,
    long-array,
    longs,
    loop,
    macroexpand,
    macroexpand-1,
    make-array,
    make-hierarchy,
    map,
    map?,
    mapcat,
    max,
    max-key,
    memfn,
    memoize,
    merge,
    merge-with,
    meta,
    method-sig,
    methods,
    min,
    min-key,
    mod,
    monitor-enter,
    monitor-exit,
    name,
    namespace,
    neg?,
    new,
    newline,
    next,
    nfirst,
    nil,
    nil?,
    nnext,
    not,
    not-any?,
    not-empty,
    not-every?,
    not=,
    ns,
    ns-aliases,
    ns-imports,
    ns-interns,
    ns-map,
    ns-name,
    ns-publics,
    ns-refers,
    ns-resolve,
    ns-unalias,
    ns-unmap,
    nth,
    nthnext,
    num,
    number?,
    odd?,
    or,
    parents,
    partial,
    partition,
    pcalls,
    peek,
    persistent!,
    pmap,
    pop,
    pop!,
    pop-thread-bindings,
    pos?,
    pr,
    pr-str,
    prefer-method,
    prefers,
    primitives-classnames,
    print,
    print-ctor,
    print-doc,
    print-dup,
    print-method,
    print-namespace-doc,
    print-simple,
    print-special-doc,
    print-str,
    printf,
    println,
    println-str,
    prn,
    prn-str,
    promise,
    proxy,
    proxy-call-with-super,
    proxy-mappings,
    proxy-name,
    proxy-super,
    push-thread-bindings,
    pvalues,
    quot,
    rand,
    rand-int,
    range,
    ratio?,
    rational?,
    rationalize,
    re-find,
    re-groups,
    re-matcher,
    re-matches,
    re-pattern,
    re-seq,
    read,
    read-line,
    read-string,
    recur,
    reduce,
    ref,
    ref-history-count,
    ref-max-history,
    ref-min-history,
    ref-set,
    refer,
    refer-clojure,
    reify,
    release-pending-sends,
    rem,
    remove,
    remove-method,
    remove-ns,
    remove-watch,
    repeat,
    repeatedly,
    replace,
    replicate,
    require,
    reset!,
    reset-meta!,
    resolve,
    rest,
    resultset-seq,
    reverse,
    reversible?,
    rseq,
    rsubseq,
    second,
    select-keys,
    send,
    send-off,
    seq,
    seq?,
    seque,
    sequence,
    sequential?,
    set,
    set!,
    set-validator!,
    set?,
    short,
    short-array,
    shorts,
    shutdown-agents,
    slurp,
    some,
    sort,
    sort-by,
    sorted-map,
    sorted-map-by,
    sorted-set,
    sorted-set-by,
    sorted?,
    special-form-anchor,
    special-symbol?,
    split-at,
    split-with,
    str,
    stream?,
    string?,
    struct,
    struct-map,
    subs,
    subseq,
    subvec,
    supers,
    swap!,
    symbol,
    symbol?,
    sync,
    syntax-symbol-anchor,
    take,
    take-last,
    take-nth,
    take-while,
    test,
    the-ns,
    throw,
    time,
    to-array,
    to-array-2d,
    trampoline,
    transient,
    tree-seq,
    true,
    true?,
    try,
    type,
    unchecked-add,
    unchecked-dec,
    unchecked-divide,
    unchecked-inc,
    unchecked-multiply,
    unchecked-negate,
    unchecked-remainder,
    unchecked-subtract,
    underive,
    unquote,
    unquote-splicing,
    update-in,
    update-proxy,
    use,
    val,
    vals,
    var,
    var-get,
    var-set,
    var?,
    vary-meta,
    vec,
    vector,
    vector?,
    when,
    when-first,
    when-let,
    when-not,
    while,
    with-bindings,
    with-bindings*,
    with-in-str,
    with-loading-context,
    with-local-vars,
    with-meta,
    with-open,
    with-out-str,
    with-precision,
    xml-seq,
    zero?,
    zipmap
  },
  sensitive,
  alsodigit=-,
  morecomment=[l];,
  morestring=[b]"
}[keywords,comments,strings]


%%%%%%%%%%%%%%%%%%%%%%%%
%% Graphics and maths %%
%%%%%%%%%%%%%%%%%%%%%%%%
\usepackage{amsmath}

% Additional amsmath symbols, see:
%
% http://texblog.org/2007/08/27/number-sets-prime-natural-integer-rational-real-and-complex-in-latex/
%
\usepackage{amssymb}

\usepackage{graphicx}
\usepackage{mathtools}
\usepackage{tikz}
\usepackage{tikz-qtree}

% Provide bold font face in maths.
\usepackage{bm}

\usepackage{subcaption}
\expandafter\def\csname ver@subfig.sty\endcsname{}

% Define an 'myalignat' command which behave as 'alignat' without the
% vertical top and bottom padding. See:
%     http://www.latex-community.org/forum/viewtopic.php?f=5&t=1890
\newenvironment{myalignat}[1]{%
  \setlength{\abovedisplayskip}{-.7\baselineskip}%
  \setlength{\abovedisplayshortskip}{\abovedisplayskip}%
  \start@align\z@\st@rredtrue#1
}%
{\endalign}

% Define additional operators:
\DeclareMathOperator*{\argmin}{arg\,min}
\DeclareMathOperator*{\argmax}{arg\,max}

% Skeleton operators.
\DeclareMathOperator*{\map}{Map}
\DeclareMathOperator*{\reduce}{Reduce}
\DeclareMathOperator*{\scan}{Scan}
\DeclareMathOperator*{\stencil}{Stencil}
\DeclareMathOperator*{\zip}{Zip}
\DeclareMathOperator*{\allpairs}{All\,Pairs}

% Maths plots using pgfplots, see:
%
%     http://pgfplots.sourceforge.net/pgfplots.pdf
%
\usepackage{pgfplots}

% Disable compatability mode.
%
\pgfplotsset{compat=1.12}

% Gantt charts using pgfgantt, see:
%
%     http://www.ctan.org/pkg/pgfgantt
%
\usepackage{pgfgantt}

% Fix milestone aspect ratio by defining a custom element.
\newganttchartelement*{mymilestone}{
  mymilestone/.style={
    shape=diamond,
    inner sep=2pt,
    draw=black,
    top color=black,
    bottom color=black,
  }
}

% Tikz flowchart configuration.
\usetikzlibrary{shapes,arrows,shadows,fit,backgrounds}
\tikzstyle{decision} = [diamond,
                        draw,
                        text width=4.5em,
                        text badly centered,
                        node distance=3cm,
                        inner sep=0pt]
\tikzstyle{block}    = [rectangle,
                        draw,
                        text width=5em,
                        text centered,
                        node distance=3cm,
                        minimum height=4em,
                        inner sep=.2cm]
\tikzstyle{line}     = [draw, -latex']

% Add dirtree picture style, see:
%
%     http://tex.stackexchange.com/a/34268
%
\newcount\dirtree@lvl
\newcount\dirtree@plvl
\newcount\dirtree@clvl
\def\dirtree@growth{%
  \ifnum\tikznumberofcurrentchild=1\relax
    \global\advance\dirtree@plvl by 1
    \expandafter\xdef\csname dirtree@p@\the\dirtree@plvl\endcsname{\the\dirtree@lvl}
  \fi
  \global\advance\dirtree@lvl by 1\relax
  \dirtree@clvl=\dirtree@lvl
  \advance\dirtree@clvl by -\csname dirtree@p@\the\dirtree@plvl\endcsname
  \pgf@xa=0.33cm\relax
  \pgf@ya=-\baselineskip\relax
  \pgf@ya=\dirtree@clvl\pgf@ya
  \pgftransformshift{\pgfqpoint{\the\pgf@xa}{\the\pgf@ya}}%
  \ifnum\tikznumberofcurrentchild=\tikznumberofchildren
    \global\advance\dirtree@plvl by -1
  \fi
}
\tikzset{
  dirtree/.style={
    growth function=\dirtree@growth,
    every node/.style={anchor=north},
    every child node/.style={anchor=west},
    edge from parent path={(\tikzparentnode\tikzparentanchor) |- (\tikzchildnode\tikzchildanchor)}
  }
}

% UML sequence diagram macros, see:
%
%     https://code.google.com/p/pgf-umlsd/
%
% Options:
%
%     underline - Underline object names
%
\usepackage[underline=false]{pgf-umlsd}

% Support for SVG graphics.
%
% NOTE that you must pass the "--shell-escape" argument to pdflatex to
% compile. NOTE also that images *MUST* be placed within the graphics
% path.
\usepackage{svg}
\graphicspath{{img/}}

%%%%%%%%%%%%%%%%%%%%%%
%% Tables and lists %%
%%%%%%%%%%%%%%%%%%%%%%

% Required to use labm8 exported tables.
%
\usepackage{booktabs}

%\usepackage{enumitem}
%\setenumerate{itemsep=0pt}

% Use no left margin for lists:
%\setlist{leftmargin=*}

\usepackage{longtable}

% Define column types L, C, R with known text justification and fixed
% widths:
\usepackage{array}
\newcolumntype{L}[1]{>{\raggedright\let\newline\\\arraybackslash\hspace{0pt}}m{#1}}
\newcolumntype{C}[1]{>{\centering\let\newline\\\arraybackslash\hspace{0pt}}m{#1}}
\newcolumntype{R}[1]{>{\raggedleft\let\newline\\\arraybackslash\hspace{0pt}}m{#1}}


%%%%%%%%%%%%%%%%%%%%%%%%%%%%%
%% Typesetting and symbols %%
%%%%%%%%%%%%%%%%%%%%%%%%%%%%%

% Adjustable font sizes in \Verbatim{}
\usepackage{fancyvrb}

%\usepackage{titlesec}
% Set section and paragraph heading fonts:
%\titleformat*{\section}{\Large\bfseries}
%\titleformat*{\subsection}{\normalsize\bfseries}
%\titleformat*{\subsubsection}{\normalsize}
%\titleformat*{\paragraph}{\large\bfseries}
%\titleformat*{\subparagraph}{\large\bfseries}

% Set section heading margins. Usage:
% \titlespacing*{<command>}{<left>}{<before>}{<after>}
%\titlespacing*{\section}{0pt}{.6em}{.3em}
%\titlespacing*{\subsection}{0pt}{.6em}{.2em}

% Set paragraph indentation size. Default is 15pt.
%\setlength{\parindent}{10pt}

% The line spacing can be globally set using \linespread:
%
% \linespread{1.2}

% Add a command \hr{} which will draw a horizontal rule the width of
% the text.
%
\newcommand{\hr}{\noindent\makebox[\linewidth]{\rule{\textwidth}{0.2pt}}}

% Add a command \br{} which will create a horizontal space of exactly
% one line height.
%
\newcommand{\br}{\hspace{\baselineskip}}

% Define a command to allow word breaking.
\newcommand*\wrapletters[1]{\wr@pletters#1\@nil}
\def\wr@pletters#1#2\@nil{#1\allowbreak\if&#2&\else\wr@pletters#2\@nil\fi}

% Define a command to create centred page titles.
\newcommand{\centredtitle}[1]{
  \begin{center}
    \large
    \vspace{0.9cm}
    \textbf{#1}
  \end{center}}

% Support hyperlinks using the \hyperref, \url and \href
% macros. Usage:
%
%    \hyperref[label_name]{''link text''}
%
%    \url{<my_url>}
%
%    \href{<my_url>}{<description>}
%
\usepackage{hyperref}

% Disable colored borders of links, cross-references etc in PDF output
\hypersetup{pdfborder={0 0 0}}

% Provide generic commands \degree, \celsius, \perthousand, \micro
% and \ohm which work both in text and maths mode.
\usepackage{gensymb}

%%%%%%%%%%%%%%%%%%%%%%%%%%%%%%%%%
%% Placeholder text generation %%
%%%%%%%%%%%%%%%%%%%%%%%%%%%%%%%%%

% Use either \blindtext or \libpsum to generate placeholder text. Also
% note the macros \blinditemize, \blindenumerate, \blinddescription.
\usepackage[english]{babel}
\usepackage{blindtext}
\usepackage{lipsum}
