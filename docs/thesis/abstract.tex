The physical limitations of microprocessor design have forced the
industry towards increasingly heterogeneous architectures to extract
performance. This trend has not been matched with software tools to
cope with such parallelism, leading to a growing disparity between the
levels of available performance and the ability for application
developers to exploit it.

Algorithmic Skeletons simplify parallel programming by providing
high-level, reusable patterns of computation. Achieving performant
skeleton implementations is a difficult task; developers must attempt
to anticipate and tune for a wide range of architectures and use
cases. This results in implementations that target the general case
and cannot provide the performance advantages that are gained from
tuning low level optimisation parameters.

To address this, I present OmniTune - an extensible and distributed
autotuner for stencil skeletons on CPUs and GPUs. Targeting the
workgroup size of OpenCL kernels, I demonstrate in a comprehensive
evaluation of \input{gen/num_runtime_stats} test cases that simple
heuristics cannot provide portable performance across the range of
architectures, kernels, and datasets which Algorithmic Skeletons must
target.

OmniTune uses procedurally generated synthetic benchmarks and machine
learning to predict workgroup sizes for unseen programs. In an
evaluation of \input{gen/num_scenarios} combinations of programs,
architectures, and datasets, OmniTune is able to achieve
$\input{gen/best_avg_classification_performance}\%$ of the available
performance, providing an average speedup of
$\input{gen/best_avg_classification_speedup}\times$ (max
$\input{gen/best_max_classification_speedup}\times$) over the best
possible performance which can be achieved without autotuning.

% TODO: Contrast with human expert.


\ifdraft{
  \newpage
  \textcolor{red}{\section*{Notes for draft version \version{}}}
  \begin{itemize}
  \item \textcolor{blue}{Blue} figures and all graphs are auto
    generated and may change before submission.
  \item Addressed all comments from first draft review on 20/7/2015.
  \item Chapter restructuring: after related work, explain the idea,
    the implementation, the methodology, then the results.
  \item There is significantly more data in the evaluation now.
  \item Stuff I haven't yet done (and target deadlines):
    \begin{enumerate}
    \item Conclusions chapter, and most of the background. Deadline:
      two weeks.
    \item I haven't evaluated the \emph{time} cost of autotuning using
      the different approaches (e.g.\ using decision trees vs linear
      regression). For example, if the cost of picking the right
      workgroup size using technique $x$ is $y$ --- how many
      iterations of a stencil do we need to perform before the
      benefits of the autotuning outweigh $y$? Deadline: one week.
    \item I talk about ``procedurally generated synthetic
      benchmarks'', but the synthetic benchmarks I use have been
      written by hand. Either implement a stencil generator or
      reword. Deadline: two weeks.
    \end{enumerate}
  \end{itemize}
}
