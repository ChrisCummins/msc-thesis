The physical limitations of microprocessor design have forced the
industry towards increasingly heterogeneous architectures to extract
performance. This trend has not been matched with software tools to
cope with such parallelism, leading to a growing disparity between the
levels of available performance and the ability for application
developers to exploit it.

Algorithmic Skeletons simplify parallel programming by providing
high-level, reusable patterns of computation. Achieving performant
skeleton implementations is a difficult task; developers must attempt
to anticipate and tune for a wide range of architectures and use
cases. This results in implementations that target the general case
and cannot provide the performance advantages that are gained from
tuning low level optimisation parameters.

To address this, I present OmniTune --- an extensible and distributed
framework for runtime autotuning of optimisation parameters. Targeting
the workgroup size of OpenCL kernels, I demonstrate an implementation
of OmniTune for stencil codes on CPUs and multi-GPU systems. I show in
a comprehensive evaluation of \input{gen/num_runtime_stats} test cases
that simple heuristics cannot provide portable performance across the
range of architectures, kernels, and datasets which Algorithmic
Skeletons must target.

OmniTune uses procedurally generated synthetic benchmarks and machine
learning to predict workgroup sizes for unseen programs. In an
evaluation of \input{gen/num_scenarios} combinations of programs,
architectures, and datasets, with up to \input{gen/max_num_params}
parameter values for each, OmniTune is able to achieve
$\input{gen/best_avg_classification_performance}\%$ of the available
performance, an improvement of
$\input{gen/best_avg_classification_speedup_he_perc}\%$ over the
values selected by human experts, without requiring any user
intervention. This adaptive tuning provides an average speedup of
$\input{gen/best_avg_classification_speedup}\times$ (max
$\input{gen/best_max_classification_speedup}\times$) over the best
possible performance which can be achieved without adaptive tuning.
