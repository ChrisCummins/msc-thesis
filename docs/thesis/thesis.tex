% DEADLINE: 12 NOON, FRIDAY, 21ST AUGUST 2015
%
% The project is only assessed on the basis of a final written
% thesis. Additional material, such as the code you submit, may be taken
% into account in case of doubt, but you should make sure that all the
% work you have done is carefully described in the thesis. Theses will
% typically conform to the following format:
%
% * The length should be 40 -- 70 pages in total and no shorter than 35
%   pages.
%
% * Title page with abstract.
%
% * Introduction: an introduction to the document, clearing stating
%   the hypothesis or objective of the project, motivation for the work
%   and the results achieved. The structure of the remainder of the
%   document should also be outlined.
%
% * Background: background to the project, previous work, exposition of
%   relevant literature, setting of the work in the proper context. This
%   should contain sufficient information to allow the reader to
%   appreciate the contribution you have made.
%
% * Description of the work undertaken: this may be divided into
%   chapters describing the conceptual design work and the actual
%   implementation separately. Any problems or difficulties and the
%   suggested solutions should be mentioned. Alternative solutions and
%   their evaluation should also be included.
%
% * Analysis or Evaluation: results and their critical analysis should
%   be reported, whether the results conform to expectations or
%   otherwise and how they compare with other related work. Where
%   appropriate evaluation of the work against the original objectives
%   should be presented.
%
% * Conclusion: concluding remarks and observations, unsolved problems,
%   suggestions for further work.  Bibliography.
%
% In addition, the thesis must be accompanied by a statement declaring
% that the student has read and understood the University's plagiarism
% guidelines. Students should budget at least six weeks for the final
% thesis writing-up phase. Where appropriate the thesis may additionally
% contain appendices in which relevant program listings, experimental
% data, circuit diagrams, formal proofs, etc. may be included. However,
% students should keep in mind that they are marked on the quality of
% the thesis, not its length. The thesis must be word-processed using
% either LaTeX or a system with similar capabilities. The LaTeX thesis
% template can be found via the local packages web page. You don't have
% to use these packages, but your thesis must match the style (i.e.,
% font size, text width etc) shown in the sample output for an
% Informatics thesis. Technical problems during project work are only
% considered for resources we provide; no technical support,
% compensation for lost data, extensions for time lost due to technical
% problems with external hard- and software as provided will be given,
% except where this is explicitly stated as part of a project
% specification and adequately resourced at the start of the project.


%%%%%%%%%%%%%%%%%%%%%%
%% Document details %%
%%%%%%%%%%%%%%%%%%%%%%
%%%%
%% Load the class. Put any options that you want here (see the documentation
%% for the list of options). The following are samples for each type of
%% thesis:
%%
%% Note: you can also specify any of the following options:
%%  logo: put a University of Edinburgh logo onto the title page
%%  frontabs: put the abstract onto the title page
%%  deptreport: produce a title page that fits into a Computer Science
%%      departmental cover [not sure if this actually works]
%%  singlespacing, fullspacing, doublespacing: choose line spacing
%%  oneside, twoside: specify a one-sided or two-sided thesis
%%  10pt, 11pt, 12pt: choose a font size
%%  centrechapter, leftchapter, rightchapter: alignment of chapter headings
%%  sansheadings, normalheadings: headings and captions in sans-serif
%%      (default) or in the same font as the rest of the thesis
%%  [no]listsintoc: put list of figures/tables in table of contents (default:
%%      not)
%%  romanprepages, plainprepages: number the preliminary pages with Roman
%%      numerals (default) or consecutively with the rest of the thesis
%%  parskip: don't indent paragraphs, put a blank line between instead
%%  abbrevs: define a list of useful abbreviations (see documentation)
%%  draft: produce a single-spaced, double-sided thesis with narrow margins
%%
%% MSc by Research, which also needs an institute
\documentclass[mscres,icsa,twopage]{infthesis}


%%%%%%%%%%%%%%%%%%%%%%%%%%%%%%%
% Draft Version Specification %
%%%%%%%%%%%%%%%%%%%%%%%%%%%%%%%
% Comment out the following line to disable all draft features:
\def\draft{}
\def\version{2015-07-02}

%%%%%%%%%%%%%%%%%%%%%
%% Draft versions  %%
%%%%%%%%%%%%%%%%%%%%%

% A set of macros for adding "todo" and rough notes to
% documents. Usage: \todo{Todo note}.

% Pretty colours plz.
\usepackage{color}

% \ifdraft{} and \ifndef{} commands.
\newcommand{\ifdraft}[1]{\ifdef{\draft}{#1}{}}
\newcommand{\ifndraft}[1]{\ifdef{\draft}{}{#1}}

% \todo{} command.
\newcommand{\todo}[1]{\ifdraft{\noindent\textcolor{red}{\em\footnotesize#1}}}

% \TODO{} command.
\newcommand{\TODO}[1]{\ifdraft{\todo{\textbf{TODO:} #1}}}

% \note{} command.
\newcommand{\note}[1]{\ifdraft{\noindent\textcolor{blue}{\em\footnotesize#1}}}

% \fixme{} command.
\newcommand{\fixme}[1]{\noindent\textcolor{red}{\em\footnotesize#1}}

% \FIXME{} command.
\newcommand{\FIXME}[1]{\todo{\textbf{FIXME:} #1}}


% Paper title
\title{Dynamic Autotuning\\of Algorithmic Skeletons%
\ifdraft{\\\textcolor{red}{DRAFT VERSION \version{}}}}

% Author
\author{Chris Cummins}

\input{preamble}

\abstract{Algorithmic Skeletons simplify parallel programming by providing
high-level, reusable patterns of computation. This forces skeleton
authors to optimise for the general case, and users to forgo the
performance advantages that are gained from tuning low level
optimisation parameters. This thesis examines the effect of one such
implementation detail --- setting the workgroup size of OpenCL kernels
--- on the performance of stencil codes in SkelCL, a library for data
parallel patterns on GPUs and CPUs. Through an empirical evaluation of
a large optimisation space, we find that there is no one sensible
value which can provides portable performance across the range of
architectures, kernels, and data sets which Algorithmic Skeletons must
target. To address this, we present an extensible and distributed
autotuner that performs runtime adaptive tuning of workgroup size. The
presented autotuner uses offline training with synthetic stencils to
rapidly evaluate a subset of the optimisation space. On a suite of 5
stencil benchmarks on 4 different GPUs, the autotuner is able to
achieve 97\% of the oracle performance, providing an average speedup
of $1.35\times$ (max $4.00\times$) over the best possible performance
which can be achieved without autotuning.

\ifdraft{
  \textcolor{red}{\section*{Notes for draft version \version{}}}
  \begin{itemize}
  \item Abstract must be submitted for inclusion in industrial
    newlsetter, deadline \emph{20th July}. No hard limits, but aim for
    100-150 words.

  \item Missing some significant chunks of both text \emph{and}
    data. The evaluation is not yet complete. Notable missing stuff:
    data from CPU results, evaluation of linear regression.
  \end{itemize}
}
}

%%%%%%%%%%
%% Body %%
%%%%%%%%%%
\begin{document}

%% Preliminary pages.
\begin{preliminary}

\maketitle

\ifndraft{%
  \begin{acknowledgements}%
  \input{acknowledgements}%
  \end{acknowledgements}%

  \standarddeclaration

  \dedication{}
}

%% Contents, figures, tables, etc:
\tableofcontents
\ifndraft{%
  \listoffigures
  \listoftables
}

\end{preliminary}


\chapter{Introduction}\label{chap:introduction}
Parallelism is increasingly being seen as the only viable approach to
maintaining continued performance improvements in a multicore
world. Despite this, the adoption of parallel programming practises
has been slow and awkward, due to the prohibitive complexity and low
level of abstractions available to programmers.

Algorithmic Skeletons address this issue by providing reusable
patterns for parallel programming, offering higher level abstractions
and reducing programmer effort~\cite{Cole1989, Cole2004}. Tuning the
performance of these Algorithmic Skeletons is a manual process which
requires exhaustively searching the optimisation space to select
optimal parameters.
% TODO: how big are these optimisation spaces?

The aim of this project is to demonstrate that the tuning of
optimisation parameters can be successfully performed at runtime. This
will enable self-tuning programs which adapt to their execution
environment by selecting optimal parameters dynamically. Such
configurations of parameters can be learned over time, allowing each
successive iteration of a program to benefit from its predecessors.
% TODO: this last sentence is weak ^^

The case for dynamically autotuning applications is strong. There are
many factors which contribute to the performance of programs which
cannot be determined by program developers.
% TODO: the case of self-tuning systems: TCP/IP.
As a result, performance optimisation requires the programmer to
either overfit the choice of parameters to optimise for a specific
task and environment, or laboriously create heuristics which segment
the optimisation space into regions of identical configurations. Both
approaches have significant drawbacks: optimising for a specific task
and environment creates brittle and non-portable optimisations that do
not generalise to other architectures and inputs, and the task of
creating heuristics which cover every possible combination of factors
is prohibitively time consuming for developers.

\subsection{Hypotheses}
This project proposes two hypotheses about the performance of
Algorithmic Skeletons:
\begin{itemize}
\item a dynamic autotuner will improve the performance of Algorithmic
  Skeletons in the general case, by selecting optimisations which
  target specific dynamic features;
\item a dynamic autotuner will provide better average performance than
  a statically tuned equivalent implementation across different
  inputs, by adapting to features which can only be determined
  dynamically.
\end{itemize}

These hypotheses can be referred to respectively as the claims
\emph{specialisation} and \emph{generalisation}. It can be inferred
from these that a dynamic autotuner cannot provide better performance
than a statically tuned equivalent implementation for a \emph{fixed}
input, since the extra instructions that implement the dynamic
autotuning behaviour present a performance overhead. The reduction of
this overhead is one of the greatest challenges facing the development
of dynamic autotuners. The novelty of my solution is to reduce
parameter convergence time by implementing a dynamic autotuner for
Algorithmic Skeletons which can store the results of successive
iterations persistently, across program runs.
% TODO: parameter convergence time? This is a totally new topic
An evaluation of a successful implementation will contribute empirical
evidence supporting the two hypotheses.
% TODO: reword. This is too passive ^^

% The rest of the document is structured as follows:
% Section~\ref{sec:motivation} contains the motivation for this
% research; Section~\ref{sec:background} briefly outlines related work;
% Sections~\ref{sec:methodology} and~\ref{sec:evaluation} describe the
% methodology and evaluation plans for this research;
% Section~\ref{sec:work-plan} contains the work plan; followed by the
% conclusion in Section~\ref{sec:conclusions}.


\chapter{Background}\label{chap:background}
This section briefly outlines some of the most closely related pieces
of work that address the issue of improving software performance
through the selection of optimal parameters. These can broadly be
categorised as either offline tuning, or dynamic optimisation.

\subsection{Offline tuning}\label{subsec:offline-tuning}
Offline tuning involves selecting the set of parameters that provides
the best performance for a given input, based on some model of
performance that is generated offline. Performance models can either
be predictive models which attempt to characterise performance as a
function of the optimisation parameters and input, or based on
empirical data gathered by evaluating many different parameter
configurations. In both cases, the performance model maps the
relationship between a set of parameters $p$ and performance $f(p,x)$,
given a specific problem $x$. The purpose of the offline tuning phase
is then to select the set of parameters $p_{optimal}$ which produces
the greatest performance:
\begin{align*}
  p_{optimal} = \argmax_{p}f(p,x)
\end{align*}
% predictive: quality limited by accuracy of f(p,x).
% empirical:  quality limited by amount of training data, or ability to
%             interpolate.
Iterative compilation is an approach to autotuning which uses an
offline training phase to perform an extensive search of the
optimisation space of a program.  Empirical data is gathered through
repeatedly compiling and evaluating different trial configurations,
before selecting the configuration which proved the most
profitable. Iterative compilation techniques has been successfully
applied to a range of optimisation challenges. Of particular relevance
to this work is MaSiF, a static autotuning tool which combines
iterative compilation techniques with machine learning.  It performs a
focused search of the optimisation space of FastFlow and Intel Thread
Building Blocks, two popular Algorithmic Skeleton libraries. While
sharing the same goal as MaSiF, the approach of this project focuses
on performing optimisation space searching at runtime, without the
need for the expensive offline training phase, which is a prohibitive
drawback of iterative compilation.

% collective optimisation
~\cite{Fursin2010}

% JS JIT
~\cite{Auler2014}

% Using Machine Learning to Focus Iterative Optimization
~\cite{Agakov}

\subsection{Dynamic optimisation}\label{subsec:dynamic-optimisation}

% snippet
Many existing dynamic optimisation systems do not store the results of
their efforts persistently, allowing the work to die along with the
host process. This approach relies on the assumption that either that
the convergence time to reach an optimal set of parameters is short
enough to have negligible performance impact, or that the runtime of
the host process is sufficiently long to reach an optimal set of
parameters in good time. Neither assumption can be shown to fit the
general case.

% snippet PETABRICKS
PetaBricks is a language and runtime which supports dynamic
algorithmic choice determined by properties of the data input. This
provides a promising space for optimisation but has the drawback of
increasing programmer effort by requiring them to implement multiple
versions of an algorithm tailored to different optimisation
parameters. SkelCL has the advantage of being able to localise this
extra programmer effort into a single library implementation.

% snippet SIBLING RIVALRY
SiblingRivalry poses an interesting solution to the challenge of
providing sustained quality of service. Resources are divided in half,
and two copies of a target subroutine are executed simultaneously, one
using the current best known configuration, and the other using a
trial configuration which is to be evaluated. If the trial
configuration outperforms the current best configuration, then it
replaces it as the new best configuration. By doing this, the tuning
framework has the freedom to evaluate vastly suboptimal configurations
while still providing adequate performance for the user. However, a
large runtime penalty is incurred by dividing the available resources
in half.

% snippet WHY MAP REDUCE SUCKS
In such cases, the overhead introduced by these massively scaleable
high performance skeleton libraries would likely outweigh the
performance gains. If Algorithmic Skeletons are to achieve widespread
adoption, they must provide scalable performance benefits not only to
the upper-tier of high performance computers but also to modest
consumer hardware, which is increasingly reliant on software
parallelism in order to achieve performance improvements.

% snippet
MapReduce is a hugely successful framework for writing high
performance massively distributed applications at a fraction of the
effort required for a hand tuned implementation. The source code of
Hadoop is over 800,000 lines of code, but efficient user programs can
be written in well under 100 lines. While is an incredible technical
feat, the overhead the huge runtime would negate the performance
advantages for all but the largest computations. Users writing
programs for more modestly specced off the shelf hardware will not be
able to take advantage of the engineering achievement.

% snippet What are your claims or hypotheses?
The problem with attempting to model optimisation spaces is that they
are fundamentally stochastic. As a result, they can only properly be
built using empirical evidence, and so the problem becomes one of
trying to extrapolate complicated many-dimensional spaces using the
least amount of data points, since the time taken to acquire data is
prohibitively expensive. Previous static autotuners have taken as long
as three months to sample the optimisation space.

\subsection{Dynamic optimisation}
Whereas iterative compilation requires an expensive offline training
phase to search an optimisation space, dynamic optimisers perform this
optimisation space exploration at runtime, allowing programs to
respond to dynamic features ``online''. This is a challenging task, as
a random search of the optimisation space will result in many
configurations with vastly suboptimal performance. In a real world
system, evaluating many suboptimal configurations will cause a
significant slowdown of the program. Thus a requirement of dynamic
optimisers is that convergence time towards optimal parameters is
minimal.

Existing dynamic optimisation research has typically taken a low level
approach to performing optimisations. Dynamo is a dynamic optimiser
which performs binary level transformations of programs using
information gathered from runtime profiling and tracing. While this
provides the ability to respond to dynamic features, it restricts the
range of optimisations that can be applied to binary
transformations. These low level transformations cannot match the
performance gains that higher level parameter tuning produces.

One of the biggest challenges facing the implementation of dynamic
optimisers is to minimise the runtime overhead so that it does not
outweigh the performance advantages of the optimisations. A
significant contributor to this runtime overhead is the requirement to
compile code dynamically. Previous research has negated this cost by
compiling multiple versions of a target subroutine ahead of time. At
runtime, execution switches between the available versions, selecting
the version with the best performance. In practice, this technique
massively reduces the optimisation space which can be searched as it
is unfeasible to insert the thousands of different versions of a
subroutine that are tested using offline autotuning.


\chapter{Methodology}\label{chap:methodology}
The objective of this research is to develop a dynamic autotuner for
the SkelCL library. The novelty of the approach posed in this research
is to combine the advantages of offline training phases and online
parameter tuning by implementing maintaining persistent data between
individual program runs.

Michel Steuwer, a research associate at the University of Edinburgh,
developed SkelCL as an approach to high-level programming for
multi-GPU systems~\cite{Steuwer2011,
Steuwer2013a}. \citeauthor{Steuwer2012} demonstrated an $11\times$
reduction in programmer effort compared to implement equivalent
programs written in pure OpenCL, while suffering only a modest 5\%
overhead~\cite{Steuwer2012}.

The core of SkelCL comprises a set of parallel container data types
for vectors and matrices, and an automatic distribution mechanism
which performs implicit transfer of these data structures between the
host and device memory. Application programmers express computations
on these data structures using Algorithmic Skeletons that are
parameterised with small sections of OpenCL code. At runtime, SkelCL
compiles the OpenCL code into compute kernels for execution on
GPUs. This makes SkelCL an excellent candidate for dynamic autotuning,
as it exposes both the optimisation space of the OpenCL compiler, and
the high level tunable parameters provided by the structure of
Algorithmic Skeletons.
% TODO: is this next sentence necessary?
SkelCL offers the unique advantage of being able to amortise many of
the costs associated with dynamic compilation due to its JIT-like
nature of compiling OpenCL kernels immediately before execution.

Implementing a dynamic optimiser poses a number of difficult
challenges which must be overcome.
% TODO: What are these challenges? ^^
There is a risk that the runtime overhead of the dynamic optimiser
will exceed the performance gained by the optimisations
themselves. The proposed approach to dynamically autotune SkelCL will
overcome one of the most significant overheads associated with dynamic
optimising: that of instrumenting the code the purposes of profiling
and tracing. Since Algorithmic Skeletons coordinate muscle functions,
it is possible to forgo many of the profiling counters that dynamic
optimisers require by making assumptions about the execution frequency
of certain code paths, given the nature of the skeleton. Additionally,
the placement of profiling counters can be optimised manually.

% snippet
Contributors to this overhead include: time spent evaluating dynamic
features and deciding on which optimisations to select (extra
instructions to execute), and either increased code size from having
multiple copies of procedures (bad for branch predictors / instruction
prefetch), or decreased ability for optimisations (because of setting
parameters at runtime instead of at compiled).

% snippet
Whereas current approaches to Algorithmic Skeleton autotuning have
largely relied on huge offline training periods and optimising for
static features, this proposed research will develop an online
autotuner which is capable of adapting to dynamic features at runtime.


\chapter{Evaluation}\label{chap:evaluation}
% EVALUATION
% ==========
%
% Results and their critical analysis should be reported, whether the
% results conform to expectations or otherwise and how they compare
% with other related work. Where appropriate evaluation of the work
% against the original objectives should be presented.


\TODO{Evaluate the effectiveness of training with synthetic
  benchmarks.}

% P. J. Fleming and J. J. Wallace, “How not to lie with statistics:
% the correct way to summarize benchmark results,” Commun. ACM,
% vol. 29, no. 3, pp. 218–221, 1986.
\TODO{%
  How to properly report benchmark results~\cite{Fleming1986}.%
}

% A. Georges, D. Buytaert, and L. Eeckhout, “Statistically Rigorous
% Java Performance Evaluation,” in Proceedings of the 22Nd Annual ACM
% SIGPLAN Conference on Object-oriented Programming Systems and
% Applications, 2007, vol. 42, no. 10, p. 57.
\TODO{%
  Statistically rigorous performance evaluations.~\cite{Georges2007}.
}


\chapter{Conclusions}\label{chap:conclusions}
% CONCLUSION
% ==========
%
% Concluding remarks and observations, unsolved problems, suggestions
% for further work.


\clearpage
\begin{appendices}

\section{Features}\label{app:features}
\begin{multicols}{2}
\begin{Verbatim}[fontsize=\scriptsize]
data_width                         numeric
data_height                        numeric
data_tin                           nominal
data_tout                          nominal
kern_north                         numeric
kern_south                         numeric
kern_east                          numeric
kern_west                          numeric
kern_max_wg_size                   numeric
kern_instruction_count             numeric
kern_ratio_AShr_insts              numeric
kern_ratio_Add_insts               numeric
kern_ratio_Alloca_insts            numeric
kern_ratio_And_insts               numeric
kern_ratio_Br_insts                numeric
kern_ratio_Call_insts              numeric
kern_ratio_FAdd_insts              numeric
kern_ratio_FCmp_insts              numeric
kern_ratio_FDiv_insts              numeric
kern_ratio_FMul_insts              numeric
kern_ratio_FPExt_insts             numeric
kern_ratio_FPToSI_insts            numeric
kern_ratio_FSub_insts              numeric
kern_ratio_GetElementPtr_insts     numeric
kern_ratio_ICmp_insts              numeric
kern_ratio_InsertValue_insts       numeric
kern_ratio_Load_insts              numeric
kern_ratio_Mul_insts               numeric
kern_ratio_Or_insts                numeric
kern_ratio_PHI_insts               numeric
kern_ratio_Ret_insts               numeric
kern_ratio_SDiv_insts              numeric
kern_ratio_SExt_insts              numeric
kern_ratio_SIToFP_insts            numeric
kern_ratio_SRem_insts              numeric
kern_ratio_Select_insts            numeric
kern_ratio_Shl_insts               numeric
kern_ratio_Store_insts             numeric
kern_ratio_Sub_insts               numeric
kern_ratio_Trunc_insts             numeric
kern_ratio_UDiv_insts              numeric
kern_ratio_Xor_insts               numeric
kern_ratio_ZExt_insts              numeric
kern_ratio_basic_blocks            numeric
kern_ratio_memory_instructions     numeric
kern_ratio_non_external_functions  numeric
dev_count                          numeric
dev_address_bits                   numeric
dev_double_fp_config               numeric
dev_endian_little                  numeric
dev_execution_capabilities         numeric
dev_extensions                     nominal
dev_global_mem_cache_size          numeric
dev_global_mem_cache_type          numeric
dev_global_mem_cacheline_size      numeric
dev_global_mem_size                numeric
dev_host_unified_memory            numeric
dev_image2d_max_height             numeric
dev_image2d_max_width              numeric
dev_image3d_max_depth              numeric
dev_image3d_max_height             numeric
dev_image3d_max_width              numeric
dev_image_support                  numeric
dev_local_mem_size                 numeric
dev_local_mem_type                 numeric
dev_max_clock_frequency            numeric
dev_max_compute_units              numeric
dev_max_constant_args              numeric
dev_max_constant_buffer_size       numeric
dev_max_mem_alloc_size             numeric
dev_max_parameter_size             numeric
dev_max_read_image_args            numeric
dev_max_samplers                   numeric
dev_max_work_group_size            numeric
dev_max_work_item_dimensions       numeric
dev_max_work_item_sizes_0          numeric
dev_max_work_item_sizes_1          numeric
dev_max_work_item_sizes_2          numeric
dev_max_write_image_args           numeric
dev_mem_base_addr_align            numeric
dev_min_data_type_align_size       numeric
dev_native_vector_width_char       numeric
dev_native_vector_width_double     numeric
dev_native_vector_width_float      numeric
dev_native_vector_width_half       numeric
dev_native_vector_width_int        numeric
dev_native_vector_width_long       numeric
dev_native_vector_width_short      numeric
dev_preferred_vector_width_char    numeric
dev_preferred_vector_width_double  numeric
dev_preferred_vector_width_float   numeric
dev_preferred_vector_width_half    numeric
dev_preferred_vector_width_int     numeric
dev_preferred_vector_width_long    numeric
dev_preferred_vector_width_short   numeric
dev_queue_properties               numeric
dev_single_fp_config               numeric
dev_type                           numeric
dev_vendor                         nominal
dev_vendor_id                      nominal
dev_version                        nominal
\end{Verbatim}
\end{multicols}


\end{appendices}

\label{bibliography}
\printbibliography

\end{document}
