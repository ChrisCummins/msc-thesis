%%%%%%%%%%%%%%%%%%%%%%
%% Document details %%
%%%%%%%%%%%%%%%%%%%%%%

% Paper title
\title{Progress Report}

% Author
\author{Chris Cummins}

\input{preamble}

%%%%%%%%%%
%% Body %%
%%%%%%%%%%
\begin{document}

\maketitle

\begin{abstract}
  \noindent
  GPUs enable massive performance through heterogeneous
  parallelism. However, developing software for these devices is
  challenging, as the programming models provided by OpenCL and CUDA
  require a low level knowledge of the underlying architecture to
  properly exploit the potential performance. SkelCL addresses this
  programmability challenge by providing high level skeleton patterns
  for common data parallel operations. This project demonstrates that
  the parameters which are necessarily abstracted by such skeletons
  can have a huge impact on performance. To demonstrate this, I
  present an autotuner for selecting the \emph{workgroup size} of
  Stencil pattern kernels. Using offline training and runtime feature
  extraction, I demonstrate that a machine-learning enabled autotuner
  can achieve \todo{XXX}\% of the oracle performance, providing an
  average speedup of \todo{XXX}$\times$ (max \todo{XXX}$\times$) over
  the best statically chosen workgroup size.
\end{abstract}

\section{Introduction}

\section{Methodology}


\subsection{Experimental Setup}

Tables~\ref{tab:hw},\ref{tab:kernels},\ref{tab:datasets}

\begin{table}
\footnotesize
\centering
\begin{tabular}{| L{1.2cm} | L{6.2cm} | L{1.5cm} | L{1.5cm} | L{1.5cm} | L{1.5cm} | L{1.5cm} |}
\hline
\scriptsize
\centering
\rowcolors{2}{white}{gray!25}
\begin{tabular}{ L{4.5cm} L{1.5cm} L{1.5cm} L{1.5cm} L{1.5cm} L{1.5cm} }
\hline
\scriptsize
\centering
\rowcolors{2}{white}{gray!25}
\begin{tabular}{ L{4.5cm} L{1.5cm} L{1.5cm} L{1.5cm} L{1.5cm} L{1.5cm} }
\hline
\scriptsize
\centering
\rowcolors{2}{white}{gray!25}
\begin{tabular}{ L{4.5cm} L{1.5cm} L{1.5cm} L{1.5cm} L{1.5cm} L{1.5cm} }
\hline
\input{gen/tab/devices}
\hline
\end{tabular}

\hline
\end{tabular}

\hline
\end{tabular}

\hline
\end{tabular}
\caption{%
  Execution devices. \todo{I also have access to a machine with 4x GTX 590.}%
}
\label{tab:hw}
\end{table}

\begin{table}
\footnotesize
\centering
\begin{tabular}{| l | l | l | l | l | l |}
\hline
\scriptsize
\centering
\rowcolors{2}{white}{gray!25}
\begin{tabular}{ l l l l l l }
\hline
\scriptsize
\centering
\rowcolors{2}{white}{gray!25}
\begin{tabular}{ l l l l l l }
\hline
\scriptsize
\centering
\rowcolors{2}{white}{gray!25}
\begin{tabular}{ l l l l l l }
\hline
\input{gen/tables/kernels}
\hline
\end{tabular}

\hline
\end{tabular}

\hline
\end{tabular}

\hline
\end{tabular}
\caption{%
  Benchmark applications, border sizes, and static instruction counts.
  The ``simple'' and ``complex'' kernels are synthetic training
  programs. \todo{I also have a FDTD benchmark.}%
}
\label{tab:kernels}
\end{table}

\begin{table}
\footnotesize
\centering
\begin{tabular}{| l | l | l | l |}
\hline
\footnotesize
\centering
\begin{tabular}{| l | l | l | l |}
\hline
\footnotesize
\centering
\begin{tabular}{| l | l | l | l |}
\hline
\footnotesize
\centering
\begin{tabular}{| l | l | l | l |}
\hline
\input{gen/tables/datasets}
\hline
\end{tabular}

\hline
\end{tabular}

\hline
\end{tabular}

\hline
\end{tabular}
\caption{%
  Datasets used.%
}
\label{tab:datasets}
\end{table}

\section{Results}

\begin{figure}
\centering
\includegraphics{gen/img/oracle_param_space.png}
\caption{%
  Distribution of optimal workgroup sizes.%
}
\end{figure}

\begin{figure}
\centering
\includegraphics{gen/img/num_param_oracle.png}
\caption{%
  Accuracy compared to the oracle as a function of the number of
  workgroup sizes used. The best accuracy that is achievable using a
  single statically chosen value is 10\%.%
}
\end{figure}

\section{Evaluation}


\section{Conclusions}


\clearpage
\begin{appendices}
\end{appendices}

\end{document}
