\begin{figure*}[t!]
\centering
\begin{subfigure}{\textwidth}
\centering
\section{Introduction}

SkelCL\footnote{\url{http://skelcl.uni-muenster.de}} is an object
oriented C++ library that provides OpenCL implementations of data
parallel algorithmic skeletons for heterogeneous
parallelism. Skeletons are parameterised with muscle functions by the
user, which are compiled into OpenCL kernels for execution on device
hardware. The Vector and Matrix container types transparently handle
communication between the host and device memory, and support
partitioning for multi-GPU execution.

% Steuwer, M., Kegel, P., & Gorlatch, S. (2011). SkelCL - A Portable
% Skeleton Library for High-Level GPU Programming. In Parallel and
% Distributed Processing Workshops and Phd Forum (IPDPSW), 2011 IEEE
% International Symposium on
% (pp. 1176–1182). IEEE. doi:10.1109/IPDPS.2011.269
\TODO{The introductory SkelCL paper. Most highly cited:
\cite{Steuwer2011}}

% Steuwer, M., & Gorlatch, S. (2013). SkelCL: Enhancing OpenCL for
% High-Level Programming of Multi-GPU Systems. Parallel Computing
% Technologies, 7979, 258–272. doi:10.1007/978-3-642-39958-9_24
\TODO{Support for multi-GPU systems: \cite{Steuwer2013a}}

% S. Breuer, M. Steuwer, and S. Gorlatch, “Extending the SkelCL
% Skeleton Library for Stencil Computations on Multi-GPU Systems,”
% HiStencils 2014, pp. 23–30, 2014.
\TODO{Stencil computations: \cite{Breuer2014}}

% M. Steuwer, M. Friese, S. Albers, and S. Gorlatch, “Introducing and
% implementing the allpairs skeleton for programming multi-GPU
% Systems,” Int. J. Parallel Program., vol. 42, pp. 601–618, 2014.
\TODO{Allpairs skeleton: \cite{Steuwer2014}}


% Steuwer, M., & Gorlatch, S. (2013). High-level Programming for
% Medical Imaging on Multi-GPU Systems Using the SkelCL
% Library. Procedia Computer Science, 18,
% 749–758. doi:10.1016/j.procs.2013.05.239
\TODO{An example of reduced programmer effort for real world
application using SkelCL: \cite{Steuwer2013}}

% Steuwer, M., Kegel, P., & Gorlatch, S. (2012). Towards High-Level
% Programming of Multi-GPU Systems Using the SkelCL Library. In
% Parallel and Distributed Processing Symposium Workshops & PhD Forum
% (IPDPSW), 2012 IEEE 26th International
% (pp. 1858–1865). Ieee. doi:10.1109/IPDPSW.2012.229
\TODO{\cite{Steuwer2012}}

\section{Pattern definitions}

\paragraph{Map}

\begin{equation}
\map\left(f, [x_1,x_2,\ldots,x_n]\right) \to [f(x_1),f(x_2),\ldots,f(x_n)]
\end{equation}

When applied to an $n \times m$ matrix:

\begin{equation}
\map\left(f,
\begin{bmatrix}
  x_{11} & \cdots & x_{1m} \\
  \vdots & \ddots & \vdots \\
  x_{n1} & \cdots & x_{nm}
\end{bmatrix}\right)
\to
\begin{bmatrix}
  f(x_{11}) & \cdots & f(x_{1m}) \\
  \vdots & \ddots & \vdots \\
  f(x_{n1}) & \cdots & f(x_{nm})
\end{bmatrix}
\end{equation}

\paragraph{Zip}

\begin{equation}
\zip\left( \oplus, [x_1,x_2,\ldots,x_n], [y_1,y_2,\ldots,y_n] \right)
\to
\left[ x_1 \oplus y_1, x_2 \oplus y_2, \ldots, x_n \oplus y_n \right]
\end{equation}

\begin{equation}
\begin{split}
\zip \left( \oplus,
\begin{bmatrix}
  x_{11} & \cdots & x_{1m} \\
  \vdots & \ddots & \vdots \\
  x_{n1} & \cdots & x_{nm}
\end{bmatrix},
\begin{bmatrix}
  y_{11} & \cdots & y_{1m} \\
  \vdots & \ddots & \vdots \\
  y_{n1} & \cdots & y_{nm}
\end{bmatrix} \right) \\
\to
\begin{bmatrix}
  x_{11} \oplus y_{11} & \cdots & x_{1m} \oplus y_{1m} \\
  \vdots & \ddots & \vdots \\
  x_{n1} \oplus y_{n1} & \cdots & x_{nm} \oplus y_{nm}
\end{bmatrix}
\end{split}
\end{equation}

\paragraph{Reduce}

\begin{equation}
\reduce \left( \oplus, i, [x_1,x_2,\ldots,x_n] \right)
\to
x_1 \oplus x_2 \oplus \ldots \oplus x_n
\end{equation}

\begin{equation}
\reduce \left( \oplus, i,
\begin{bmatrix}
  x_{11} & \cdots & x_{1m} \\
  \vdots & \ddots & \vdots \\
  x_{n1} & \cdots & x_{nm}
\end{bmatrix} \right)
\to
x_{11} \oplus x_{12} \oplus \ldots \oplus x_{nm}
\end{equation}

\paragraph{Scan}

\begin{equation}
\scan \left( \oplus, i, [x_1,x_2,\ldots,x_n] \right)
\to
\left[ i, x_1, x_1 \oplus x_2, \ldots, x_1 \oplus x_2 \oplus \ldots \oplus x_n \right]
\end{equation}

\paragraph{AllPairs}

\begin{equation}
\allpairs \left( \oplus,
\begin{bmatrix}
  x_{11} & \cdots & x_{1d} \\
  \vdots & \ddots & \vdots \\
  x_{n1} & \cdots & x_{nd}
\end{bmatrix},
\begin{bmatrix}
  y_{11} & \cdots & y_{1m} \\
  \vdots & \ddots & \vdots \\
  y_{n1} & \cdots & y_{nm}
\end{bmatrix} \right)
\to
\begin{bmatrix}
  z_{11} & \cdots & z_{1m} \\
  \vdots & \ddots & \vdots \\
  z_{n1} & \cdots & z_{nm}
\end{bmatrix}
\end{equation}

where:

\begin{equation}
z_{ij} =
\left[ x_{i1}, x_{i2}, \ldots, x_{id} \right] \oplus
\left[ y_{j1}, y_{j2}, \ldots, y_{jd} \right]
\end{equation}

an additional implementation is provided for when the $\oplus$
operator is known to match that of a zip pattern:

\begin{equation}
z_{ij} =
\left[
  x_{i1}, \oplus y_{j1}, x_{i2} \oplus y_{j2}, \ldots, x_{id} \oplus y_{jd}
\right]
\end{equation}


\paragraph{Stencil}

Given a customising function $f$, a \emph{stencil shape} $S$

\begin{equation}
\stencil \left( f, S,
\begin{bmatrix}
  x_{11} & \cdots & x_{1m} \\
  \vdots & \ddots & \vdots \\
  x_{n1} & \cdots & x_{nm}
\end{bmatrix} \right)
\to
\begin{bmatrix}
  z_{11} & \cdots & z_{1m} \\
  \vdots & \ddots & \vdots \\
  z_{n1} & \cdots & z_{nm}
\end{bmatrix}
\end{equation}

where:

\begin{equation}
z_{ij} = f \left(
\begin{bmatrix}
  z_{i-S_n,j-S_w} & \cdots & z_{i-S_n,j+S_e} \\
  \vdots & \ddots & \vdots \\
  z_{i+S_s,j-S_w} & \cdots & z_{i+S_s,j+S_e}
\end{bmatrix} \right)
\end{equation}

A popular application of Stencil codes is for iterative problems, in
which \todo{\ldots} discrete time steps $0 <= t <= t_{max}$, and
$t \in \mathbb{Z}$

\begin{equation}
g(f, S, M, t) =
\begin{cases}
  \stencil \left( f, S, g(f, S, M, t-1) \right),& \text{if } t \geq 1\\
  M_{init}, & \text{otherwise}
\end{cases}
\end{equation}


\section{Implementation details}

Each skeleton is represented by a template class, declared in a header
file detailing the public API. A private header file contains the
template definition. E.g. \texttt{SkelCL/Map.h} contains the Map
class, and \texttt{SkelCL/detail/MapDef.h} contains the
implementation. Non-trivial kernels are stored in separate source
files, e.g. \texttt{SkelCL/detail/MapKernel.cl}.

\subsection{Container Types}

\TODO{Description of vector and matrices, supported data types, lazy
  data transfer \ldots}

\subsection{Skeleton implementations}

\TODO{Description of OpenCL skeleton templates, and the compilation
  process - i.e. substitution of user functions, handling additional
  arguments  \ldots}

\section{Example applications}

\TODO{%
  Provide definition of three simple example programs, then code
  listings for SkelCL implementations and a comparison of runtimes
  using a decent GPU vs. sequential CPU. Preferably also hand coded
  OpenCL?%
}

\subsection{Application 1: Dot Product}

\TODO{Definition, and performance results.}

\lstset{language=C++}
\begin{lstlisting}[
  basicstyle=\scriptsize,
  caption={Example program to calculate dot product using SkelCL.}
]
#include <SkelCL/SkelCL.h>
#include <SkelCL/Vector.h>
#include <SkelCL/Zip.h>
#include <SkelCL/Reduce.h>

int main(int argc, char* argv[]) {
  // Initialise SkelCL to use any device.
  skelcl::init(skelcl::nDevices(1).deviceType(skelcl::device_type::ANY));

  // Define the skeleton objects.
  skelcl::Zip<int(int, int)> mult("int func(int x, int y) { return x * y; }");
  skelcl::Reduce<int(int)> sum("int func(int x, int y) { return x + y; }", "0");

  // Create two vectors A and B of length "n".
  const int n = 1024; skelcl::Vector<int> A(n), B(n);
  skelcl::Vector<int>::iterator a = A.begin(), b = B.begin();
  while (a != A.end()) { *a = rand() % n; ++a; *b = rand() % n; ++b; }

  // Invoke skeleton: x = A . B
  int x = sum(mult(A, B)).first();

  return 0;
}
\end{lstlisting}

\subsection{Application 2: Mandelbrot Set}

\TODO{Definition, listing, and performance results.}

\subsection{Application 3: Gaussian Blur}

\TODO{Definition, listing, and performance results.}


\section{Summary}

\caption{}
\label{subfig:skelcl}
\end{subfigure}

\vspace{1em}

\begin{subfigure}{\textwidth}
\centering
\begin{tikzpicture}[%
    thick,
    scale=.65,
    every node/.style={scale=0.65},
    node distance = 3cm,
    % Database shape:
    database/.style={%
      draw,
      cylinder,
      cylinder uses custom fill,
      shape border rotate=90,
      aspect=0.25
    }
  ]

% Nodes:
\node (start) [block, fill=red!10, text width=3.5cm]
  {\textbf{\begin{tabular}{c}Invoke\\skeleton object\end{tabular}}};
\node (arch) [block, below of=start, yshift=.8cm, text width=3.5cm] {Architecture};
\node (features) [block, right of=start, xshift=.5cm] {Extract dynamic features};
\node (params) [block, right of=features, xshift=.3cm] {Set parameters};
\node (compile) [block, right of=params] {OpenCL Compiler};
\node (kernel) [block, fill=green!10, right of=compile] {\textbf{OpenCL kenerl}};
\node (exec) [block, right of=kernel] {Execute};

\node (training-set) [database, fill=orange!10, below of=params]
  {\textbf{\begin{tabular}{c}Training\\data\end{tabular}}};

\node (persistent) [database, fill=orange!10, below of=training-set, yshift=-.5cm]
  {\textbf{\begin{tabular}{c}Persistent\\data\end{tabular}}};


\node (runtime) [draw, dashed, color=gray, yshift=-1.3cm,
                 inner ysep=2.7cm, inner xsep=4cm,
                 label={[yshift=-.7cm]\textbf{Runtime}},
                 fit=(start) (exec)] {};

% Connectors:
\draw[->] (start) -- (features);
\draw[->] (arch) -- (features);
\draw[->] (features) -- (params);
\draw[->] (params) -- (compile);
\draw[->] (compile) -- (kernel);
\draw[->] (kernel) -- (exec);

\draw[->,dotted] (features) -- (training-set);
\draw[->,dotted] (exec.south) -- (training-set);
\draw[->] (training-set) -- (params);

\draw[<->] (training-set) -- (persistent);

\end{tikzpicture}

\caption{}
\label{subfig:skelcl-autotune}
\end{subfigure}

\caption{The skeleton invocation behaviour of current SkelCL
  % TODO: neaten up these subfigure references.
  (\ref{subfig:skelcl}), and with dynamic autotuning
  (\ref{subfig:skelcl-autotune}). When invoked, the dynamic features
  of a skeleton object are extracted and an online machine learning
  model recommends optimal parameters. The OpenCL compiler is invoked
  on this parameterised skeleton to generate an OpenCL kernel for
  execution on device. Profiling information is gathered during
  execution and added to the training dataset.}
\label{fig:method}
\end{figure*}

The work required to complete this research has been broadly divided
into three stages:

\begin{enumerate}
\item Modify SkelCL to enable the runtime configuration of
  optimisation parameters, and the extraction of dynamic features.
\item Evaluate the significance of optimisation parameters and dynamic
  features to select the parameters which provide the most profitable
  optimisation space.
\item Implement a dynamic autotuner which uses online machine learning
  to searches and builds a model of this optimisation space at
  runtime.
\end{enumerate}

This section outlines the work required for each stage, listing some
of the possible challenges and approaches to overcoming them.

\subsection{Model features and parameters}
In the first stage, I will replace compile-time constant parameters in
the SkelCL library with variable parameters, and add an API to support
dynamically setting these parameters. Examples of parameters which can
be set dynamically include the mapping of work items to threads and
the OpenCL compiler configuration. I will also modify the container
types of SkelCL so that dynamic features of input data structures can
be extracted at runtime. Examples of dynamic features include the
distribution of elements within a container and the data type of
elements.

\subsection{Feature and parameter evaluation}
Pilot experiments will then be used to evaluate the effect of
different parameters and features on performance by varying test
stimuli across a range of different inputs and measuring their impact
on performance. Statistical methods will be used to analyse these
results in order to isolate the parameters and features with the
greatest performance impact. Principle Component Analysis can be used
to reduce the dimensionality of this optimisation space by orientating
the space along the directions of greatest variance.

This exploratory phase provides opportunities for the novel use of
dynamic features for the purpose of autotuning Algorithmic Skeletons,
since previous research has focused on offline tuning and so has been
restricted to the set of features which can be either statically
determined or approximated. In addition to dynamic features, SkelCL
compiles OpenCL kernels at runtime immediately before execution. This
enables the setting of arbitrary optimisation parameters without
having to use the multi-versioning techniques of many state of the art
dynamic optimisers.

\subsection{Dynamic autotuner implementation}
In the final stage, I will construct a dynamic autotuner by
implementing an online machine learning model that uses the features
and parameters selected in the exploratory phase. To the best of our
knowledge, this will be the first attempt to develop a dynamic
autotuner using online machine learning for Algorithmic Skeletons. The
goal of the implementation will be to exploit the advantages of
dynamic features to provide improved performance over existing static
Algorithmic Skeleton autotuners, and to exploit the high-level
abstractions of Algorithmic Skeletons to provide improved performance
over existing dynamic optimisers.

Implementing a dynamic autotuner poses a number of difficult
challenges. The primary challenge is in developing an online machine
learning model which balances the potentially conflicting requirements
to:

\begin{itemize}
\item offer the best performance configurations to maximise
  performance;
\item search the large optimisation space to avoid becoming trapped in
  local minima;
\item build statistical confidence in training data through repeated
  invocations of identical configurations.
\end{itemize}

\noindent
\textbf{\textit{TODO: reinforcement learning!}}

% \begin{enumerate}
% \item Receive an instance
% \item Predict the label of the instance
% \item Receive the true label of the instance
% \end{enumerate}

% A unique challenge of online autotuning is in minimising the runtime
% overhead so that it does not outweigh the performance gains of the
% optimisations themselves. The proposed approach to dynamically
% autotune SkelCL will overcome one of the most significant overheads
% associated with dynamic optimising: that of instrumenting the code
% for the purpose of profiling and tracing. Since Algorithmic
% Skeletons coordinate muscle functions, it is possible to forgo many
% of the profiling counters that dynamic optimisers require by making
% assumptions about the execution frequency of certain code paths,
% given the nature of the skeleton. Profiling counters will be placed
% by hand at critical points in the code, allowing the frequency of
% counter increments to be minimised.

The convergence time of autotuning can be improved by saving the
results of trial configurations persistently in a central
database. This provides two advantages: first, it allows the results
of autotuning to be used by future program runs; second, it allows the
result of autotuning to be shared amongst any program which uses the
SkelCL library. The challenge of implementing this persistent data
storage is that results must be stored efficiently and compactly, to
allow for indefinite scaling of the dataset as future results are
added. Increasing the size of the training dataset also increases the
time required to compute new results, and there is additional
latencies associated with reading and writing data to and from disk.
