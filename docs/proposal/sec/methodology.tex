The novelty of the approach posed in this research is to combine the
advantages of offline training phases and online parameter tuning by
implementing a dynamic autotuner which maintains persistent data
in-between program executions using SkelCL.

Michel Steuwer, a research associate at the University of Edinburgh,
developed SkelCL as an approach to high-level programming for
multi-GPU systems. Steuwer demonstrated an $11\times$ reduction in
programmer effort compared to implement equivalent programs written in
pure OpenCL, while suffering only a modest 5\% overhead. The core of
SkelCL comprises a set of parallel container data types for vectors
and matrices, and an automatic distribution mechanism which performs
implicit transfer of these data structures between the host and device
memory. Application programmers express computations on these data
structures using Algorithmic Skeletons that are parameterised with
small sections of OpenCL code. At runtime, SkelCL compiles the OpenCL
code into compute kernels for execution on GPUs. This makes SkelCL an
excellent candidate for dynamic autotuning, as it exposes both the
optimisation space of the OpenCL compiler, and the high level tunable
parameters provided by the structure of Algorithmic Skeletons. SkelCL
offers the unique advantage of being able to amortise many of the
costs associated with dynamic compilation due to its JIT-like nature
of compiling OpenCL kernels immediately before execution.

Implementing a dynamic optimiser poses a number of difficult
challenges which must be overcome.
% TODO: What are these challenges? ^^
There is a risk that the runtime overhead of the dynamic optimiser
will exceed the performance gained by the optimisations
themselves. The proposed approach to dynamically autotune SkelCL will
overcome one of the most significant overheads associated with dynamic
optimising: that of instrumenting the code the purposes of profiling
and tracing. Since Algorithmic Skeletons coordinate muscle functions,
it is possible to forgo many of the profiling counters that dynamic
optimisers require by making assumptions about the execution frequency
of certain code paths, given the nature of the skeleton. Additionally,
the placement of profiling counters can be optimised manually.

% skelcl
~\cite{Steuwer2011, Steuwer2013a}

% skelcl 11x 5%
~\cite{Steuwer2012}.

% snippet
Contributors to this overhead include: time spent evaluating dynamic
features and deciding on which optimisations to select (extra
instructions to execute), and either increased code size from having
multiple copies of procedures (bad for branch predictors / instruction
prefetch), or decreased ability for optimisations (because of setting
parameters at runtime instead of at compiled).

% snippet
Whereas current approaches to Algorithmic Skeleton autotuning have
largely relied on huge offline training periods and optimising for
static features, this proposed research will develop an online
autotuner which is capable of adapting to dynamic features at runtime.
