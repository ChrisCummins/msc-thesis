Existing research has shown that Algorithmic Skeletons improve
programmer effectiveness for a range of tasks from general purpose
computing, to bioinformatics, and complex simulations. For example,
the SkelCL library has been used to implement high performance medical
imaging applications. A dynamic autotuner for SkelCL will improve the
performance of these applications, and provide a starting point for
future research into the online autotuning of Algorithmic Skeletons.

% snippet
We are ideally suited for tackling this difficult problem at
University of Edinburgh. Not only have academic members been
responsible for introducing and developing Algorithmic
Skeletons~\cite{Cole1989, Cole2004, Benoit2005a}, but there is a large
and active research interest in iterative compilation and machine
learning based optimisation~\cite{Fursin2008, Agakov,
Fursin2005}. Previous research at the University of Edinburgh has also
approach the static autotuning of Algorithmic
Skeletons~\cite{Collins2012, Collins2013}, which will provide a solid
source of inspiration and an interesting counterpoint for evaluating
the performance of a dynamic autotuning approach.

% snippet
While iterative compilation is a very well studied field, fewer papers
have been published about dynamic optimisation. Therefore work in this
field has a greater chance of influencing future research, besides the
primary benefit of improving the performance of algorithmic skeletons.
