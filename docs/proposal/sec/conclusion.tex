Algorithmic Skeletons have been shown to improve programmer
effectiveness by providing the necessary high-level abstractions for
parallel programming. The SkelCL library has been used to implement
high performance medical imaging applications using shorter, better
structured programs that perform within 5\% of a hand tuned OpenCL
implementation~\cite{Steuwer2012}.

A dynamic autotuner for SkelCL will improve the performance of these
applications by providing optimisations which, while targeting
specific applications and inputs, are transferable across problem
domains. Through the novel combination of persistent training data
with the dynamic compilation of OpenCL kernels and runtime features,
it will be possible to implement a dynamic autotuner for SkelCL with
minimal runtime overhead.

Our approach will be to first modify SkelCL so that it enables runtime
configuration of optimisation parameters, then to evaluate the
significance of dynamic features and optimisation parameters, before
implementing a dynamic autotuner which searches and builds a
persistent model of this optimisation space at run.

We are ideally suited for tackling this difficult problem at
University of Edinburgh, with expert researchers in the fields of
Algorithmic Skeletons, iterative compilation, and machine learning
based optimisation. Previous research at the University of Edinburgh
has addressed the static autotuning of Algorithmic
Skeletons~\cite{Collins2012, Collins2013}, which will provide a point
of reference for comparing a dynamic autotuning approach.
