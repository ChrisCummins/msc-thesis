This paper proposes the development of a dynamic autotuner for SkelCL
which uses online machine learning techniques to explore the space of
optimisation parameters and recommend optimal configurations based on
dynamic features. This will be the first attempt to implement a
dynamic autotuner using online machine learning for Algorithmic
Skeletons, and will enable runtime performance tuning without the
requirement of long offline training periods associated with state of
the art Algorithmic Skeletons autotuners.

Our approach will be to modify SkelCL so that it enables runtime
configuration of optimisation parameters and dynamic extraction of
features. Then we will evaluate the significance of dynamic features
and optimisation parameters to develop an effective online machine
learning algorithm. We will implement this as a dynamic autotuner
which searches and builds a persistent model of this optimisation
space at runtime. We will compare the performance of this dynamic
autotuner across a number of benchmarks against a baseline unmodified
SkelCL and a gold standard hand-tuned OpenCL implementation.

Algorithmic Skeletons have been shown to improve programmer
effectiveness by providing the necessary high-level abstractions for
parallel programming. The SkelCL library has been used to implement
high performance medical imaging applications using shorter, better
structured programs that perform within 5\% of a hand tuned OpenCL
implementation~\cite{Steuwer2012}. As the trend towards increasingly
parallel hardware continues, the demand for high-performance parallel
programming abstractions will continue.

We are ideally suited for tackling this difficult problem at
University of Edinburgh, with expert researchers in the fields of
Algorithmic Skeletons, iterative compilation, and machine learning
based optimisation. Previous research at the University of Edinburgh
has addressed the static autotuning of Algorithmic
Skeletons~\cite{Collins2012, Collins2013}, which will provide a point
of reference for comparing a dynamic autotuning approach.
