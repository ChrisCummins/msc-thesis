Existing research has shown that Algorithmic Skeletons improve
programmer effectiveness for a range of tasks from general purpose
computing, to bioinformatics, and complex simulations. For example,
the SkelCL library has been used to implement high performance medical
imaging applications~\cite{Steuwer2012}.

A dynamic autotuner for SkelCL will improve the performance of these
applications, and provide a starting point for future research into
the online autotuning of Algorithmic Skeletons. By combining the novel
application of persistent training data with the dynamic compilation
of OpenCL kernels and runtime features, it will be possible to
implement a dynamic autotuner for SkelCL with minimal runtime
overhead.

We are ideally suited for tackling this difficult problem at
University of Edinburgh, with expert researchers in the field of
Algorithmic Skeletons, iterative compilation, and machine learning
based optimisation. Previous research at the University of Edinburgh
has addressed the static autotuning of Algorithmic
Skeletons~\cite{Collins2012, Collins2013}, which will provide a good
point of reference for comparing a dynamic autotuning approach.
