My hypothesis is that the performance of Algorithmic Skeletons will be
improved by using dynamic autotuning. After implementing a prototype
dynamic autotuner, experimental evidence will be collected to support
or refute this hypothesis: standard empirical methods will be used to
evaluate the performance of a range of representative
benchmarks. Careful selection of these benchmarks

The evaluation methodology must incorporate statistically rigorous
performance evaluation techniques \cite{Georges2007}.

As a result, evidence must be collected to support

To test this hypothesis, I will
collect empirical data from a suite of representative benchmarks,
comparing the performance of my implementation against: baseline
performance provided by an unmodified SkelCL implementation;\ and a
hand-tuned ``oracle'' implementation using an optimal configuration
discovered through an offline exhaustive search of the optimisation
space.

Other measurable success metrics include: the overhead introduced by
the runtime; the amount of time required to converge to a sufficiently
good configuration; and the ability of the dynamic optimiser to adapt
to changes in dynamic features (e.g.\ system load). All of these
metrics will be evaluated by profiling performance benchmarks.

% important factors: amount of training data, variance in inputs
