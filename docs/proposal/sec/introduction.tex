Parallelism is increasingly being seen as the only viable approach to
maintaining continued performance improvements in a multicore
world. Despite this, the adoption of parallel programming practises
has been slow and awkward, due to the prohibitive complexity and low
level of abstractions available to programmers.

Algorithmic Skeletons address this issue by providing reusable
patterns for parallel programming, offering higher-level abstractions
and reducing programmer effort~\cite{Cole1989, Cole2004}. Tuning the
performance of these Algorithmic Skeletons is a manual process which
requires exhaustively searching the optimisation space to select
optimal parameters.
% TODO: how big are these optimisation spaces?

The aim of this project is to demonstrate that the tuning of
optimisation parameters can be successfully performed at runtime. This
will enable self-tuning programs which adapt to their execution
environment by selecting optimal parameters dynamically. Such
configurations of parameters can be learned over time, allowing each
successive iteration of a program to benefit from its predecessors.
% TODO: this last sentence is weak ^^

The case for dynamically autotuning applications is strong. There are
many factors which contribute to the performance of programs which
cannot be determined by program developers.
% TODO: the case of self-tuning systems: TCP/IP.
As a result, performance optimisation requires the programmer to
either overfit the choice of parameters to optimise for a specific
task and environment, or laboriously create heuristics which segment
the optimisation space into regions of identical configurations. Both
approaches have significant drawbacks: optimising for a specific task
and environment creates brittle and non-portable optimisations that do
not generalise to other architectures and inputs, and the task of
creating heuristics which cover every possible combination of factors
is prohibitively time consuming for developers.

\subsection{Hypotheses}
This project proposes two hypotheses about the performance of
Algorithmic Skeletons:
\begin{itemize}
\item a dynamic autotuner will improve the performance of Algorithmic
  Skeletons in the general case, by selecting optimisations which
  target specific dynamic features;
\item a dynamic autotuner will provide better average performance than
  a statically tuned equivalent implementation across different
  inputs, by adapting to features which can only be determined
  dynamically.
\end{itemize}

These hypotheses can be referred to respectively as the claims
\emph{specialisation} and \emph{generalisation}. It can be inferred
from these that a dynamic autotuner cannot provide better performance
than a statically tuned equivalent implementation for a \emph{fixed}
input, since the extra instructions that implement the dynamic
autotuning behaviour present a performance overhead. The reduction of
this overhead is one of the greatest challenges facing the development
of dynamic autotuners. The novelty of my solution is to reduce
parameter convergence time by implementing a dynamic autotuner for
Algorithmic Skeletons which can store the results of successive
iterations persistently, across program runs.
% TODO: parameter convergence time? This is a totally new topic
An evaluation of a successful implementation will contribute empirical
evidence supporting the two hypotheses.
% TODO: reword. This is too passive ^^

% The rest of the document is structured as follows:
% Section~\ref{sec:motivation} contains the motivation for this
% research; Section~\ref{sec:background} briefly outlines related work;
% Sections~\ref{sec:methodology} and~\ref{sec:evaluation} describe the
% methodology and evaluation plans for this research;
% Section~\ref{sec:work-plan} contains the work plan; followed by the
% conclusion in Section~\ref{sec:conclusions}.
