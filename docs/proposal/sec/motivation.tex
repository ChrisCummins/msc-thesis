\textit{TODO. This section will contain a brief outline the results of
two tests on an example parallel merge sort implementation (DaC
skeleton). The first test will show the strong and uneven relationship
that TWO optimisation parameters have on performance; this is to
demonstrate that these optimisation spaces are complex and require
automated searching (i.e.\ it can't just be done by hand). The second
test will show the difference in the optimal value of a SINGLE
optimisation parameter as a function of different input types and
sizes; this is to demonstrate that these optimisation spaces are
influenced by dynamic features.}

% Consider a recursive merge sort. When called, the algorithm determines
% whether the input list is short enough to be solved directly using a
% linear sorting method, or whether it should split it into multiple
% sub-lists and sort them by recursing on each sub-list before combining
% the results. This computational pattern of repeatedly dividing a
% problem into smaller subproblems which are then recombined is
% abstracted by a Divide and Conquer pattern. This can be parallelised
% effectively by considering each recursion as a new task which can be
% executed concurrently.

% The parallel Divide and Conquer pattern is a common form of
% Algorithmic Skeleton, whereby the user provides muscle functions for
% the split, merge, and conquer logic, and the skeleton can coordinate
% the allocation of new tasks. When implementing such a Divide and
% Conquer skeleton, there are two immediate parameters which will
% greatly affect the performance: the maximum depth at which recursion
% should occur as a new task, as opposed to sequentially; and the
% threshold minimum size of the input problem before the problem is
% conquered directly rather than recursively.

% Existing iterative compilation techniques can perform an exhaustive
% search of the optimisation space generated by these two parameters,
% which would reveals a strong interaction between them: the optimum
% value for one parameter is strongly influenced by the value of the
% other parameter. Additionally, the optimisation space of both
% parameters is strongly influenced by an independent factor: the size
% and type of the input problem. This means that in the case of a
% parallel merge sort algorithm, the optimum values for the max
% recursion depth and minimum input threshold parameters will be very
% different when sorting lists of integers and lists of bytes, or lists
% of arbitrary user-chosen data structures. This cannot be modelled
% using iterative compilation techniques, as the size and type of the
% input problem a dynamic features, which can only be determined at
% runtime.

% Static approaches to this problem involve segmenting the dynamic
% feature space using heuristics in order to select optimum values for
% approximate ranges. The effectiveness of these heuristics is limited
% by their complexity and the thoroughness of the optimisation space
% search. In addition, the resulting optimisation heuristics would be
% very fragile and non-portable, so that the whole tedious process would
% need to be repeated for every target architecture, and with every new
% generation of hardware. Such an approach is clearly
% impractical. Compare this to the alternate approach of a Divide and
% Conquer skeleton which is capable of performing this empirical data
% gathering online and during normal execution, and which will use
% successive iterations to converge naturally upon an optimum
% configuration. Such a system would be capable of dealing with varying
% dynamic features which would destroy the capabilities of a static
% heuristic based system. This is the goal of my research.
\newpage
