\begin{figure*}[!b]
\centering
\begin{subfigure}[t]{0.45\textwidth}
\centering
\begin{tikzpicture}
\tikzstyle{every node}=[font=\small]
\begin{axis}[%
  width=\linewidth,
  height=\linewidth,
  xlabel={Parallelisation depth},
  ylabel={Speedup}
]table {
  \addplot+ table {dat/10.dat};   \addlegendentry{$n_s=10$}
  \addplot+ table {dat/1010.dat}; \addlegendentry{$n_s=1010$}
  \addplot+ table {dat/2010.dat}; \addlegendentry{$n_s=2010$}
}
\end{axis}
\end{tikzpicture}

\vspace{-1.5em} % Shrink vertical padding
\caption{}
\label{subfig:dac-pardepth}
\end{subfigure}
~
\begin{subfigure}[t]{0.45\textwidth}
\centering
\begin{tikzpicture}
\tikzstyle{every node}=[font=\small]
\begin{axis}[%
  width=\linewidth, height=\linewidth,
  xlabel={Parallelisation depth},
  ylabel={Speedup},
  cycle list name=black white
]table {
  \addplot+[mark=square] table {dat/in-int-n6.dat}; \addlegendentry{$10^6$ ints}
  \addplot+[mark=o]      table {dat/in-int-n4.dat}; \addlegendentry{$10^4$ ints}
  \addplot+[mark=x]      table {dat/in-char-n3.dat}; \addlegendentry{$10^3$ bytes}
}
\end{axis}
\end{tikzpicture}

\vspace{-1.5em} % Shrink vertical padding
\caption{}
\label{subfig:dac-in}
\end{subfigure}

\caption{The performance impact of dynamic features on the
  optimisation parameter ``parallelisation depth'': in
  \ref{subfig:dac-pardepth}, as a function of split size $n_s$; in
  \ref{subfig:dac-in}, as a function of input type and size. In both
  cases, no parallelisation depth value can provide optimal
  performance for all inputs.}
\label{fig:dac}
\end{figure*}

Consider a recursive merge sort algorithm. The algorithm takes an
input list, and returns a sorted permutation. It checks the length of
the input list to see if it is short enough to solve directly using a
linear sorting method, or whether it should split it into multiple
sub-lists and sort them recursively before combining the results. This
computational pattern is abstracted by the Divide and Conquer
skeleton, which can be effectively parallelised by executing each
recursion as a new parallel task. The Divide and Conquer skeleton
takes an input of type $T_i$ and returns an output of type $T_o$, and
 is parameterised with definitions for four muscle functions:

\begin{myalignat}{3}
should\_divide &: T_i & &\rightarrow boolean\\
divide &: T_i & &\rightarrow [T_i]\\
conquer &: T_i & &\rightarrow T_o\\
combine &: [T_o] & &\rightarrow T_o
\end{myalignat}

The degree of a Divide and Conquer skeleton is the number of
sub-problems that the $divide$ function splits a problem into. For a
given degree $k$, the number of tasks $n$ grows exponentially with
recursion depth $d$:

\[n = k^{d} - 1\]

On real hardware, the number of available processing units limits the
number of tasks which can be effectively executed in parallel. Since
the Divide and Conquer pattern does not constrain the maximum depth
that an algorithm may recurse to, the skeleton author must impose a
maximum ``parallelisation depth'' to prevent the task switching costs
skeletons which recurse deeply. Recursion above the parallelisation
depth causes the creation of parallel tasks, below this depth,
recursion occurs sequentially.

The remainder of this section describes experimental data that
consider the effect of varying input conditions on the optimal
parallelisation depth.

\subsection{Experimental setup}
I implemented a Divide and Conquer skeleton and parameterised it with
muscle functions to implement merge sort. I parallelised the skeleton
using the C++11 Thread Support Library, and a testbench recorded the
mean time to sort a vector of random unsorted data over 30
iterations. I varied the parallelisation depth over the range 0
(sequential) through 10.

\subsection{Results}
Figure~\ref{fig:dac} shows the mean performance speedup of different
parallelisation depths over sequential
execution. Figure~\ref{subfig:dac-pardepth} shows the effect of
varying the split size, which is a property of the $should\_divide$
muscle function that determines the maximum list size at which
recursive sort should bottom out and insertion sort is
used. Figure~\ref{subfig:dac-in} shows the effect of varying the size
and data type of the input vector.

We observe that changes in the input and muscle function definitions
can have a significant impact when determining the optimal
parallelisation depth parameter. Since the skeleton author cannot
determine the types and values of inputs \emph{a priori}, they must
resort to picking a value which they expect to provide best average
case performance, or devising a technique which sets this optimisation
parameter at runtime as a response to different inputs. This proposal
describes a solution using the latter approach.
