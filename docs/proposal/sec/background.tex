This section briefly outlines some of the most closely related pieces
of work that address the issue of improving software performance
through the selection of optimal parameters. These can broadly be
categorised as either offline tuning, or dynamic optimisation.

\subsection{Offline tuning}\label{subsec:offline-tuning}
Offline tuning involves selecting the set of parameters that provides
the best performance for a given input, based on some model of
performance that is generated offline. Performance models can either
be predictive, in that they attempt to characterise performance as a
function of the optimisation parameters and input, or empirical, in
that they predict performance based on empirical data gathered by
evaluating many different parameter configurations. In both cases, the
performance model $f(p,x)$ maps the relationship between a set of
parameters $p$, a specific problem $x$, and some metric of
profitability. The purpose of the offline tuning phase is to select
the set of parameters $p_{optimal}$ which maximises output of the
performance model:
\begin{align*}
  p_{optimal} = \argmax_{p}f(p,x)
\end{align*}
For predictive models, the quality of results is limited by the
ability of the prediction model to accurately capture the behaviour of
a real world system. Given the complexities of modern hardware and
software stacks, such models have become increasingly hard to develop.
% TODO: good for unknown inputs.
The quality of empirical models is limited by the amount of training
data available to it, or the ability to interpolate between training
data when faced with new unknown inputs.

An example of an empirical approach to offline tuning is iterative
compilation, which uses an offline training phase to perform an
extensive search of the optimisation space of a program by compiling
programs with different combinations of compiler transforms in order
to select the set which provides the best performance.

Iterative compilation techniques has been successfully applied to a
range of optimisation challenges, particularly when combined with
machine learning techniques to focus the number of evaluations of
training programs which are required~\cite{Agakov}. MILEPOST GCC is a
research compiler that uses iterative compilation to adjust compiler
heuristics for optimising programs on different
architectures~\cite{Fursin2008}. Machine learning techniques are
applied to model the large optimisation space, based on static
features extracted from the source code of training programs. A
classifier predicts optimisation parameters for new programs by
comparing the static program features against the training data.

The primary issue with iterative compilation techniques is the amount
of time required to evaluate a large range of training
programs. \citeauthor{Fursin2005} attempt to address this problem by
compiling multiple versions of a program's subroutines and switching
between them at runtime \cite{Fursin2005}. This has the disadvantage
of massively reducing the number of possible transformations that can
be searched.

An alternative approach to the problem of gathering sufficient
training data is to distribute the task of collecting it by enabling
the results of different program evaluations to be shared with a
central remote database. This has been successfully applied to offline
autotuners, in which the overhead of a round trip to a remote server
is performed offline~\cite{Fursin2010, Auler2014}. A typical 150ms
network round trip time in the performance critical path of a dynamic
autotuner will cause a serious degradation of performance, given
typical system performance in excess of 100 MIPS.

An offline tuning tool with particular relevance to this work is
MaSiF~\cite{Collins2013}, a static autotuner which uses iterative
compilation techniques to perform a focused search of the optimisation
space of FastFlow and Intel Thread Building Blocks, two popular
Algorithmic Skeleton libraries. While sharing the same goal as MaSiF,
the approach of this project focuses on performing optimisation space
searching at runtime.

\subsection{Dynamic optimisation}\label{subsec:dynamic-optimisation}
Whereas iterative compilation requires an expensive offline training
phase to search the space of possible optimisations, dynamic
optimisers perform this optimisation space exploration at runtime,
allowing programs to respond to dynamic features ``online''. This is a
challenging task, as a random search of an optimisation space will
typically result in many configurations with vastly suboptimal
performance. In a real world system, evaluating many suboptimal
configurations will cause a significant slowdown of the program. Thus
a requirement of dynamic optimisers is that convergence time towards
optimal parameters is minimised.

Existing dynamic optimisation research has typically taken a low level
approach to performing optimisations. Dynamo is a dynamic optimiser
which performs binary level transformations of programs using
information gathered from runtime profiling and tracing. While this
provides the ability to respond to dynamic features, it restricts the
range of optimisations that can be applied to binary
transformations. These low level transformations cannot match the
performance gains that higher level parameter tuning produces.

One of the biggest challenges facing the implementation of dynamic
optimisers is to minimise the runtime overhead so that it does not
outweigh the performance advantages of the optimisations. A
significant contributor to this runtime overhead is the requirement to
compile code dynamically. Previous research has negated this cost by
compiling multiple versions of a target subroutine ahead of time. At
runtime, execution switches between the available versions, selecting
the version with the best performance. In practice, this technique
massively reduces the optimisation space which can be searched as it
is unfeasible to insert the thousands of different versions of a
subroutine that are tested using offline autotuning.

% snippet
Many existing dynamic optimisation systems do not store the results of
their efforts persistently, allowing the work to die along with the
host process. This approach relies on the assumption that either that
the convergence time to reach an optimal set of parameters is short
enough to have negligible performance impact, or that the runtime of
the host process is sufficiently long to reach an optimal set of
parameters in good time. Neither assumption can be shown to fit the
general case.

% snippet PETABRICKS
PetaBricks is a language and runtime which supports dynamic
algorithmic choice determined by properties of the data input. This
provides a promising space for optimisation but has the drawback of
increasing programmer effort by requiring them to implement multiple
versions of an algorithm tailored to different optimisation
parameters. SkelCL has the advantage of being able to localise this
extra programmer effort into a single library implementation.

% snippet SIBLING RIVALRY
SiblingRivalry poses an interesting solution to the challenge of
providing sustained quality of service. Resources are divided in half,
and two copies of a target subroutine are executed simultaneously, one
using the current best known configuration, and the other using a
trial configuration which is to be evaluated. If the trial
configuration outperforms the current best configuration, then it
replaces it as the new best configuration. By doing this, the tuning
framework has the freedom to evaluate vastly suboptimal configurations
while still providing adequate performance for the user. However, a
large runtime penalty is incurred by dividing the available resources
in half.

% snippet WHY MAP REDUCE SUCKS
In such cases, the overhead introduced by these massively scaleable
high performance skeleton libraries would likely outweigh the
performance gains. If Algorithmic Skeletons are to achieve widespread
adoption, they must provide scalable performance benefits not only to
the upper-tier of high performance computers but also to modest
consumer hardware, which is increasingly reliant on software
parallelism in order to achieve performance improvements.

% snippet
MapReduce is a hugely successful framework for writing high
performance massively distributed applications at a fraction of the
effort required for a hand tuned implementation. The source code of
Hadoop is over 800,000 lines of code, but efficient user programs can
be written in well under 100 lines. While is an incredible technical
feat, the overhead the huge runtime would negate the performance
advantages for all but the largest computations. Users writing
programs for more modestly specced off the shelf hardware will not be
able to take advantage of the engineering achievement.

% snippet What are your claims or hypotheses?
The problem with attempting to model optimisation spaces is that they
are fundamentally stochastic. As a result, they can only properly be
built using empirical evidence, and so the problem becomes one of
trying to extrapolate complicated many-dimensional spaces using the
least amount of data points, since the time taken to acquire data is
prohibitively expensive. Previous static autotuners have taken as long
as three months to sample the optimisation space.
