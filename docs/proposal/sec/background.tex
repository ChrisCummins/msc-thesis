\begin{figure*}[t]
\begin{subfigure}[b]{\textwidth}
\centering
\begin{tikzpicture}[%
    auto,
    thick,
    scale=1,
    every node/.style={scale=0.65},
    node distance = 3cm,
    % Database shape:
    database/.style={%
      draw,
      cylinder,
      cylinder uses custom fill,
      shape border rotate=90,
      aspect=0.25
    }
  ]

% Nodes:
\node (t-start3) [block, fill=red!10, xshift=-.2cm, yshift=.2cm] {};
\node (t-start2) [block, fill=red!10, xshift=-.1cm, yshift=.1cm] {};
\node (t-start) [block, fill=red!10] {\textbf{Training programs}};
\node (t-features) [block, right of=t-start] {Extract static features};
\node (t-params) [block, right of=t-features, text width=3cm, xshift=.4cm] {Set parameter configuration};
\node (t-compile) [block, right of=t-params, xshift=.7cm] {Compile};
\node (t-binary) [block, fill=green!10, right of=t-compile] {\textbf{Binary}};
\node (t-execute) [block, right of=t-binary] {Execute};
\node (t-arch) [block, below of=t-start, yshift=.8cm] {Architecture};

\node (runtime) [draw, dashed, color=gray, yshift=.5cm,
                 inner ysep=2cm, inner xsep=4cm,
                 label={[yshift=-.7cm]\textbf{Training phase}},
                 fit=(t-start) (t-arch) (t-execute)] {};

\node (training-set) [database, fill=orange!10, below of=t-params, yshift=-1cm]
  {\textbf{\begin{tabular}{c}Training\\data\end{tabular}}};

\node (p-arch) [block, below of=t-arch] {Architecture};
\node (p-start) [block, fill=red!10, below of=p-arch, yshift=.8cm] {\textbf{User program}};
\node (p-features) [block, right of=p-start] {Extract static features};
\node (p-params) [block, right of=p-features, text width=3cm, xshift=.4cm] {Set parameter configuration};
\node (p-compile) [block, right of=p-params, xshift=1cm] {Compile};
\node (p-binary) [block, fill=green!10, right of=p-compile, xshift=1cm] {\textbf{Binary}};

% Connectors:
\draw[->] (t-start) -- (t-features);
\draw[->] (t-arch) -- (t-features);
\draw[->] (t-params) -- (t-compile);
\draw[->] (t-compile) -- (t-binary);
\draw[->] (t-binary) -- (t-execute);
\draw[->] (t-execute) |- (4.157cm,1.0cm) -- (t-params.north);

\draw[->,dotted] (t-features) -- (training-set);
\draw[->,dotted] (t-execute) -- (training-set);
\draw[->] (training-set) -- (t-params);
\draw[->] (training-set) -- (p-params);
\draw[->,dotted] (p-features) -- (training-set);

\draw[->] (p-start) -- (p-features);
\draw[->] (p-arch) -- (p-features);
\draw[->] (p-params) -- (p-compile);
\draw[->] (p-compile) -- (p-binary);
\end{tikzpicture}

\caption{}
\label{subfig:autotuner-offline}
\end{subfigure}

\vspace{1em}

\begin{subfigure}[b]{\textwidth}
\centering
\begin{tikzpicture}[%
    auto,
    thick,
    scale=1,
    every node/.style={scale=0.65},
    node distance = 3cm,
    % Database shape:
    database/.style={%
      draw,
      cylinder,
      cylinder uses custom fill,
      shape border rotate=90,
      aspect=0.25
    }
  ]

% Nodes:
\node (t-start) [block, fill=red!10] {\textbf{User program}};
\node (t-arch) [block, below of=t-start, yshift=.8cm] {Architecture};
\node (t-features) [block, right of=t-start] {Extract static features};
\node (t-procs) [block, right of=t-features, xshift=.4cm] {Clone procedures};
\node (t-compile) [block, right of=t-procs] {Compile};
\node (t-binary3) [block, fill=green!10, right of=t-compile, text width=3cm, xshift=.5cm, yshift=.2cm] {};
\node (t-binary2) [block, fill=green!10, right of=t-compile, text width=3cm, xshift=.6cm, yshift=.1cm] {};
\node (t-binary) [block, fill=green!10, right of=t-compile, text width=3cm, xshift=.7cm] {\textbf{Multiversioned binary}};

\node (training-set) [database, fill=orange!10, below of=t-procs]
  {\textbf{\begin{tabular}{c}Training\\data\end{tabular}}};

\node (t-dispatch) [block, right of=t-binary, xshift=.7cm] {Procedure dispatcher};
\node (t-exec) [block, right of=t-dispatch] {Execute procedure};

\node (runtime) [draw, dashed, color=gray, yshift=.5cm,
                 inner ysep=1.5cm, inner xsep=1.3cm,
                 label={[yshift=-.7cm]\textbf{Runtime}},
                 fit=(t-dispatch) (t-exec)] {};

% Connectors:
\draw[->] (t-start) -- (t-features);
\draw[->] (t-arch) -- (t-features);
\draw[->] (t-procs) -- (t-compile);
\draw[->] (t-compile) -- (t-binary);
\draw[->] (t-binary) -- (t-dispatch);
\draw[->] (t-dispatch) -- (t-exec);
\draw[->,dotted] (t-exec.north) |- (10.92cm,1.0cm) -- (t-dispatch.north);

\draw[->,dotted] (t-features) -- (training-set);
\draw[->,dotted] (t-exec.south) |- (training-set);
\draw[->] (training-set) -- (t-procs);

\end{tikzpicture}

\caption{}
\label{subfig:autotuner-online}
\end{subfigure}

\caption{Two approaches to static autotuning: in
  \ref{subfig:autotuner-offline}, offline autotuning using a separate
  training phase, as used in~\cite{Agakov, Fursin2011, Collins2013};
  in \ref{subfig:autotuner-online}, online autotuning using procedure
  multiversioning, as used in~\cite{Fursin2005}. In offline
  autotuning, training programs are used to populate the training
  dataset. In online autotuning, multiple versions of procedures and
  compiled and switched between using a procedure dispatcher at
  runtime.}
\label{fig:autotuners}
\end{figure*}

Relevant approaches to the problem of optimisation parameter tuning
can be broadly categorised as either offline tuning or dynamic
optimisation. This section outlines some of the most important works
in each category, followed by an introduction to the SkelCL library.

\subsection{Offline tuning}\label{subsec:offline-tuning}
Offline tuning involves selecting the set of parameters that provides
the best performance for a given input based on some model of
performance that has been generated beforehand. Performance models can
either be predictive, in that they attempt to characterise performance
as a function of the optimisation parameters and input, or empirical,
in that they select optimisation parameters based on empirical data
gathered from prior evaluation of many different parameter
configurations. In both cases, a performance function $f(c,p)$ models
the relationship between a parameter configuration $c$, a program $p$,
and some profitability goal. The purpose of the offline tuning phase
is to select the configuration $c_{optimal}$ which maximises the
output of the performance model:
\begin{align*}
  c_{optimal} = \argmax_{p}f(c,p)
\end{align*}
The quality of predictive models is limited by the ability of the
prediction function to accurately capture the behaviour of a real
world system. Given the complexities of modern architectures and
software stacks, such models have become increasingly hard to develop,
although \citeauthor{Yotov2003} demonstrated in~\cite{Yotov2003} that
under certain scenarios, the performance of accurately generated
hand-tuned models can approach that of empirical optimisations.

The quality of empirical models is limited by the amount of training
data available to it, and the ability to interpolate between training
data when faced with new unknown inputs. Offline machine learning
techniques have proven popular as an approach to reducing the number
of evaluations of training programs which are
required. In~\cite{Agakov}, \citeauthor{Agakov} use Markov Chains to
learn the most profitable areas of the optimisation space of source to
source transformations.

In~\cite{Fursin2011}, \citeauthor{Fursin2011} present Milepost GCC, a
self-tuning research compiler that selects optimisations based on
static program features. The approach proposed in this paper differs by
performing this search of the optimisation space during normal program
runs, instead of requiring costly offline training.

The task of collecting training data for offline autotuners has been
effectively distributed in~\cite{Fursin2014, Auler2014}. A remote
server contains a central store of training data which is retrieved
and contributed to by distributed clients; this allows multiple
clients to share the results of optimisations. The overhead of
communicating with a remote server would be too great to use
dynamically, a typical 150ms network round trip time in the critical
path of a program would cause a serious performance degradation.

\citeauthor{Collins2013} presented the offline autotuner MaSiF
in~\cite{Collins2013}. Principle Component Analysis was used to reduce
the search size of the optimisation space for FastFlow and Intel
Thread Building Blocks, two popular Algorithmic Skeleton
libraries. They achieved 89\% of the oracle performance by searching
0.05\% of the optimisation space. This paper differs by targeting the
feature space of heterogeneous parallelism and using online machine
learning instead of offline training.

A system-level overview of offline autotuning is shown in
Figure~\ref{subfig:autotuner-offline}.

\subsection{Dynamic optimisation}\label{subsec:dynamic-optimisation}
Dynamic optimisers improve the performance of programs by exploring
the optimisation space at runtime. Implementing an effective dynamic
optimiser is a challenging task, as the need to search the
optimisation space must be balanced against the need to provide
quality of service by avoiding suboptimal configurations. In a real
world system, evaluating many suboptimal configurations will cause a
significant slowdown of the program. Thus a requirement of dynamic
optimisers is that convergence time towards optimal parameters must be
minimised.

Dynamo is a dynamic optimiser which performs binary level
transformations of programs using information gathered from runtime
profiling and tracing~\cite{Bala2000}. While this provides the ability
to respond to dynamic features, it restricts the range of
optimisations that can be applied to binary transformations such as
function inlining, and cannot offer the performance gains that
higher-level parameter tuning such as setting the size of thread pools
provides.

\citeauthor{Fursin2005} negated the cost of dynamic compilation
in~\cite{Fursin2005} by compiling multiple versions of target
subroutines ahead of time. At runtime, execution is switched between
the available versions which are ranked by
performance. Figure~\ref{subfig:autotuner-online} shows a system-level
overview of this approach. In practice, this technique massively
reduces the size of the optimisation space which can be searched as it
is unfeasible to insert the thousands of different versions of a
subroutine that are tested using offline tuning. The approach proposed
in this paper enables online searching of the entire optimisation
space by compiling OpenCL kernels at runtime.

Many existing dynamic optimisation systems do not store the results of
their efforts persistently, allowing the training data to be lost when
the host process terminates. This approach relies on the assumption
that either the convergence time to reach an optimal set of parameters
is short enough to have negligible cost, or that the run time of the
process is sufficiently long to reach an optimal set of parameters in
good time. Neither assumption can be shown to fit the general
case. This has led to the development of collective compilation
techniques, which involve persistently storing the results of
successive optimisation runs in a persistent
database~\cite{Fursin2010}.

In~\cite{Ansel2009a}, \citeauthor{Ansel2009a} attempts to capture
high-level algorithmic choices using PetaBricks, a language and
compiler which allows programmers to express algorithms that target
specific dynamic features, and to select which algorithm to execute at
runtime. This has the disadvantage of increasing programmer effort by
requiring them to implement multiple versions of an algorithm tailored
to different optimisation parameters. A dynamic autotuner for
Algorithmic Skeletons will be able to exploit these high-level
optimisations without increasing programmer effort, by hiding the
complexity of optimisations within the SkelCL library.

SiblingRivalry~\cite{Ansel2012} is a dynamic optimiser that provides
sustained quality of service by dividing the available processing
units in half. When invoked, two copies of a target subroutine are
executed simultaneously, one using the current best known
configuration, and the other using a trial configuration which is to
be evaluated. If the trial configuration outperforms the current best
configuration then it replaces it as the new best configuration. This
allows for the low cost evaluation of suboptimal configurations, but
incurs a large runtime penalty by dividing the available resources in
half.

\subsection{SkelCL}
Michel Steuwer, a research associate at the University of Edinburgh,
developed SkelCL as an approach to high-level programming for
multi-GPU systems~\cite{Steuwer2011,
  Steuwer2013a}. \citeauthor{Steuwer2012} demonstrated an $11\times$
reduction in programmer effort compared to equivalent programs
implemented using pure OpenCL, while suffering only a modest 5\%
performance overhead~\cite{Steuwer2012}.

SkelCL comprises a set of parallel container data types for vectors
and matrices, and an automatic distribution mechanism that performs
implicit transfer of these data structures between the host and device
memory. Application programmers express computations on these data
structures by parameterising Algorithmic Skeletons with small sections
of OpenCL code. At runtime, SkelCL compiles the Algorithmic Skeletons
into compute kernels for execution on GPUs. This makes SkelCL an
excellent candidate for dynamic autotuning, as it exposes both the
optimisation space of the OpenCL compiler, and the high-level tunable
parameters provided by the structure of Algorithmic Skeletons.
