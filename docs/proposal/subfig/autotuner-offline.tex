\begin{tikzpicture}[%
    auto,
    thick,
    scale=1,
    every node/.style={scale=0.65},
    node distance = 3cm,
    % Database shape:
    database/.style={%
      draw,
      cylinder,
      cylinder uses custom fill,
      shape border rotate=90,
      aspect=0.25
    }
  ]

% Nodes:
\node (t-start3) [block, fill=gray!10, xshift=-.2cm, yshift=.2cm] {};
\node (t-start2) [block, fill=gray!10, xshift=-.1cm, yshift=.1cm] {};
\node (t-start) [block, fill=gray!10] {\textbf{Training programs}};
\node (t-features) [block, right of=t-start] {Extract static features};
\node (t-params) [block, right of=t-features, text width=3cm, xshift=.4cm] {Set parameter configuration};
\node (t-compile) [block, right of=t-params, xshift=.7cm] {Compile};
\node (t-binary) [block, fill=gray!10, right of=t-compile] {\textbf{Binary}};
\node (t-execute) [block, right of=t-binary] {Execute};
\node (t-arch) [block, below of=t-start, yshift=.8cm] {Architecture};

\node (runtime) [draw, dashed, color=gray, yshift=.5cm,
                 inner ysep=2cm, inner xsep=4cm,
                 label={[yshift=-.7cm]\textbf{Training phase}},
                 fit=(t-start) (t-arch) (t-execute)] {};

\node (training-set) [database, fill=gray!10, below of=t-params, yshift=-1cm]
  {\textbf{\begin{tabular}{c}Training\\data\end{tabular}}};

\node (p-arch) [block, below of=t-arch] {Architecture};
\node (p-start) [block, fill=gray!10, below of=p-arch, yshift=.8cm] {\textbf{User program}};
\node (p-features) [block, right of=p-start] {Extract static features};
\node (p-params) [block, right of=p-features, text width=3cm, xshift=.4cm] {Set parameter configuration};
\node (p-compile) [block, right of=p-params, xshift=1cm] {Compile};
\node (p-binary) [block, fill=gray!10, right of=p-compile, xshift=1cm] {\textbf{Binary}};

% Connectors:
\draw[->] (t-start) -- (t-features);
\draw[->] (t-arch) -- (t-features);
\draw[->] (t-params) -- (t-compile);
\draw[->] (t-compile) -- (t-binary);
\draw[->] (t-binary) -- (t-execute);
\draw[->] (t-execute) |- (4.157cm,1.0cm) -- (t-params.north);

\draw[->,dotted] (t-features) -- (training-set);
\draw[->,dotted] (t-execute) -- (training-set);
\draw[->] (training-set) -- (t-params);
\draw[->] (training-set) -- (p-params);
\draw[->,dotted] (p-features) -- (training-set);

\draw[->] (p-start) -- (p-features);
\draw[->] (p-arch) -- (p-features);
\draw[->] (p-params) -- (p-compile);
\draw[->] (p-compile) -- (p-binary);
\end{tikzpicture}
