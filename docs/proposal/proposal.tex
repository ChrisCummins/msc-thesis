%%%%%%%%%%%%%%%%%%%%%%
%% Document details %%
%%%%%%%%%%%%%%%%%%%%%%

% Author
\author{Chris Cummins}

% Date (Month Year)
\date{December 2014}

% Paper title
\title{Dynamic Autotuning\\of Algorithmic Skeletons}
%\newcommand{\multilinetitle}{}

% Subtitle
\newcommand{\subtitle}{MSc Research Proposal}

% Degree title
\newcommand{\degreeTitle}{MSc by Research\\ Pervasive Parallelism}

% Institution
\newcommand{\institution}{School of Informatics,\\
  The University of Edinburgh}

\input{_\jobname.tex}

%%%%%%%%%%
%% Body %%
%%%%%%%%%%
\section{Introduction}

Parallel computing is increasingly being seen as the only viable
approach for maintaining continued performance improvements in a
multicore world. Despite this, the adoption of parallel programming
practises has been slow and awkward, due to the prohibitive complexity
and low level of abstraction when writing parallel
software. Algorithmic Skeletons address this issue by equipping
programmers with reusable patterns for parallel programming, offering
higher level abstractions and reducing programmer effort. Tuning these
Algorithmic Skeletons is currently a manual process which requires
exhaustively searching the optimisation space to select optimum
parameters. The aim of this research is to demonstrate that this
parameter tuning can be performed at runtime without the need for
expensive offline training phases, a process called dynamic
autotuning.

It is my hypothesis that the performance of Algorithmic Skeletons can
be improved by developing an autotuner which considers dynamic
features which cannot be determined at compile time. The premise is
that the optimisation spaces of Algorithmic Skeletons are shaped
significantly by features which can only be determined at runtime, and
that effective searching of these spaces can only be performed by
collecting empirical data rather than building predictive models.

The objective of this research is improve the performance of
Algorithmic Skeletons by enabling them to adjust their behaviour at
runtime. This will be demonstrated by adding dynamic autotuning to
SkelCL, a C++ Algorithmic Skeleton Framework which targets
heterogeneous parallel programming using OpenCL.

% alg skel
~\cite{Cole1989, Cole2004}

% mapreduce
~\cite{Dean2008}
% intel tbb
~\cite{IntelTBB}

% snippet
Parallel computing, traditionally the remit of scientific programming,
is increasingly being seen as the only viable approach for maintaining
continued performance improvements in a multicore computing world.
Despite its growing popularity, writing robust parallel software is an
inherently challenging task, requiring the programmer to think in
unfamiliar paradigms, and with many associated problems raised by race
conditions, deadlock, and managing access to shared resources.

% snippet
Equivalent parallel programs require many more lines of code than
their sequential counterparts, and the additional programmer effort
that is required is dedicated to writing coordination logic -- the
logic responsible for allocating and coordinating access to shared
resources. Algorithmic Skeletons address the difficulties of parallel
programming by providing a higher level abstraction that encapsulates
this coordination logic, ridding the application programmer of these
responsibilities and allowing them to focus instead on the
problem-solving logic.

% snippet
Previous research has attempted to address this issue using iterative
compilation techniques, by tuning the performance of Algorithmic
Skeletons through searching the optimisation space offline and
selecting a set of parameter values that provide optimum performance
for a given program. Such tools perform static autotuning, that is,
they automatically tune optimisation parameters based on static
features which can be selected at compile time.

% snippet
Research interest in Algorithmic Skeletons is high, and while
frameworks of Algorithmic Skeletons abound, widespread adoption has so
far been restricted largely to established use cases that rely heavily
on high performance computing, for example, Google's MapReduce, and
Intel's Thread Building Blocks. While the demand for such frameworks
in the field of High Performance Computing (HPC) is self-evident, this
should by no means blinker the ambitions of skeletons research. The
benefits of Algorithmic Skeletons extend beyond that of HPC and cover
general purpose computing.

% snippet
The popularity of programming with parallel patterns is rapidly
increasing, as they have been demonstrated as a means of providing
re-usability to the thousands of man-hours that is required to write a
tuned and stable parallel application. For example, Google's
MapReduce, which allows their programmers to write data sorting
programs in 55 lines of code, while taking advantage of the over
800,000 lines of code present in a MapReduce implementation. Any
research forwarding the cause of parallel patterns will prove
extremely valuable to both application developers and future
researchers.

\subsection{Motivating Example}
% TODO: Static parameter tuning example

To give a concrete example, consider a recursive merge sorting
algorithm. When called, the algorithm determines whether the input
list is short enough to be solved directly using a linear sorting
method, or whether it should split it into multiple sub-lists and sort
them by recursing on each sub-list before combining the results. This
computational pattern of repeatedly dividing a problem into smaller
subproblems which are then recombined is abstracted by a Divide and
Conquer pattern. This can be parallelised effectively by considering
each recursion as a new task which can be executed concurrently.

The parallel Divide and Conquer pattern is a common form of
Algorithmic Skeleton, whereby the user provides muscle functions for
the split, merge, and conquer logic, and the skeleton can coordinate
the allocation of new tasks. When implementing such a Divide and
Conquer skeleton, there are two immediate parameters which will
greatly affect the performance: the maximum depth at which recursion
should occur as a new task, as opposed to sequentially; and the
threshold minimum size of the input problem before the problem is
conquered directly rather than recursively.

Existing iterative compilation techniques can perform an exhaustive
search of the optimisation space generated by these two parameters,
which would reveals a strong interaction between them: the optimum
value for one parameter is strongly influenced by the value of the
other parameter. Additionally, the optimisation space of both
parameters is strongly influenced by an independent factor: the size
and type of the input problem. This means that in the case of a
parallel merge sort algorithm, the optimum values for the max
recursion depth and minimum input threshold parameters will be very
different when sorting lists of integers and lists of bytes, or lists
of arbitrary user-chosen data structures. This cannot be modelled
using iterative compilation techniques, as the size and type of the
input problem a dynamic features, which can only be determined at
runtime.

Static approaches to this problem involve segmenting the dynamic
feature space using heuristics in order to select optimum values for
approximate ranges. The effectiveness of these heuristics is limited
by their complexity and the thoroughness of the optimisation space
search. In addition, the resulting optimisation heuristics would be
very fragile and non-portable, so that the whole tedious process would
need to be repeated for every target architecture, and with every new
generation of hardware. Such an approach is clearly
impractical. Compare this to the alternate approach of a Divide and
Conquer skeleton which is capable of performing this empirical data
gathering online and during normal execution, and which will use
successive iterations to converge naturally upon an optimum
configuration. Such a system would be capable of dealing with varying
dynamic features which would destroy the capabilities of a static
heuristic based system. This is the goal of my research.

\section{Background}

% What are the main pieces of related work?
% Has a similar solution to yours been applied to different problems?
In developing a dynamic autotuner for Algorithmic Skeletons, it is
helpful to draw on existing research from the fields of iterative
compilation and dynamic optimisation.

% snippet
Many existing dynamic optimisation systems do not store the results of
their efforts persistently, allowing the work to die along with the
host process. This approach relies on the assumption that either that
the convergence time to reach an optimal set of parameters is short
enough to have negligible performance impact, or that the runtime of
the host process is sufficiently long to reach an optimal set of
parameters in good time. Neither assumption can be shown to fit the
general case.

% snippet PETABRICKS
PetaBricks is a language and runtime which supports dynamic
algorithmic choice determined by properties of the data input. This
provides a promising space for optimisation but has the drawback of
increasing programmer effort by requiring them to implement multiple
versions of an algorithm tailored to different optimisation
parameters. SkelCL has the advantage of being able to localise this
extra programmer effort into a single library implementation.

% snippet SIBLING RIVALRY
SiblingRivalry poses an interesting solution to the challenge of
providing sustained quality of service. Resources are divided in half,
and two copies of a target subroutine are executed simultaneously, one
using the current best known configuration, and the other using a
trial configuration which is to be evaluated. If the trial
configuration outperforms the current best configuration, then it
replaces it as the new best configuration. By doing this, the tuning
framework has the freedom to evaluate vastly suboptimal configurations
while still providing adequate performance for the user. However, a
large runtime penalty is incurred by dividing the available resources
in half.

% snippet WHY MAP REDUCE SUCKS
In such cases, the overhead introduced by these massively scaleable
high performance skeleton libraries would likely outweigh the
performance gains. If Algorithmic Skeletons are to achieve widespread
adoption, they must provide scalable performance benefits not only to
the upper-tier of high performance computers but also to modest
consumer hardware, which is increasingly reliant on software
parallelism in order to achieve performance improvements.

% snippet
MapReduce is a hugely successful framework for writing high
performance massively distributed applications at a fraction of the
effort required for a hand tuned implementation. The source code of
Hadoop is over 800,000 lines of code, but efficient user programs can
be written in well under 100 lines. While is an incredible technical
feat, the overhead the huge runtime would negate the performance
advantages for all but the largest computations. Users writing
programs for more modestly specced off the shelf hardware will not be
able to take advantage of the engineering achievement.

% snippet What are your claims or hypotheses?
The problem with attempting to model optimisation spaces is that they
are fundamentally stochastic. As a result, they can only properly be
built using empirical evidence, and so the problem becomes one of
trying to extrapolate complicated many-dimensional spaces using the
least amount of data points, since the time taken to acquire data is
prohibitively expensive. Previous static autotuners have taken as long
as three months to sample the optimisation space.

\subsection{Iterative compilation}
Iterative compilation is an approach to autotuning which uses an
offline training phase to perform an extensive search of a program's
optimisation space by gathering empirical data through repeatedly
compiling and evaluating different trial configurations. Iterative
compilation techniques has been successfully applied to a range of
optimisation challenges. Of particular relevance to this work is
MaSiF, a static autotuning tool which combines iterative compilation
techniques with machine learning to perform a focused search of the
optimisation space of FastFlow and Intel Thread Building Blocks, two
popular Algorithmic Skeleton libraries. While sharing the same goal as
MaSiF, the approach of this project focuses on performing optimisation
space searching at runtime, without the need for the expensive offline
training phase, which is a prohibitive drawback of iterative
compilation.

\subsection{Dynamic optimisation}
Whereas iterative compilation requires an expensive offline training
phase to search an optimisation space, dynamic optimisers perform this
optimisation space exploration at runtime, allowing programs to
respond to dynamic features ``online''. This is a challenging task, as
a random search of the optimisation space will result in many
configurations with vastly suboptimal performance. In a real world
system, it is not satisfactory to be evaluating many suboptimal
configurations, as this will cause significant slowdown of the
program. Thus dynamic optimisers must be designed so that convergence
time towards optimal parameters is minimal.

Existing dynamic optimisation research has typically taken a low level
approach to performing optimisations. Dynamo is a dynamic optimiser
which performs binary level transformations of programs using
information gathered from runtime profiling and tracing. While this
provides the ability to respond to dynamic features, the range of
optimisations that can be applied is limited to binary
transformations, and so it cannot perform the high level parameter
tuning which can often have greatest impact on performance.

One of the biggest challenges facing the implementation of dynamic
optimisers is to minimise the runtime overhead so that it does not
outweigh the performance advantages of the optimisations. A
significant contributor to this runtime overhead is provided by the
requirement to compile code dynamically. Previous research has negated
this cost by compiling multiple versions of a target subroutine ahead
of time. At runtime, execution switches between the many versions,
selecting the version with the best performance. In practice, this
technique massively reduces the optimisation space which can be
searched as it is unfeasible to insert the thousands of different
versions of a subroutine that are tested using offline autotuning.

\section{Methodology}

The novelty of the approach posed in this research is to combine the
advantages of offline training phases and online parameter tuning by
implementing a dynamic autotuner which maintains persistent data
in-between program executions using SkelCL.

Michel Steuwer, a research associate at the University of Edinburgh,
developed SkelCL as an approach to high-level programming for
multi-GPU systems, demonstrating an $11\times$ reduction in programmer
effort compared to implement equivalent programs written in pure
OpenCL, while suffering only a modest 5\% overhead. The core of SkelCL
comprises a set of parallel container data types for vectors and
matrices, with an automatic distribution mechanism which performs
implicit transfer of these data structures between the host and device
memory. Computations on these data structures are expressed using
Algorithmic Skeletons, parameterised with small sections of OpenCL
code. At runtime, the OpenCL code is compiled into compute kernels for
execution on GPUs. This makes SkelCL an excellent candidate for
dynamic autotuning, as it exposes not only the optimisation space of
the OpenCL compiler, but also the high level tunable parameters
provided by the structure of Algorithmic Skeletons. SkelCL offers the
unique advantage of being able to amortise many of the costs
associated with dynamic compilation due to it's JIT-like nature of
compiling OpenCL kernels immediately before execution.

Implementing a dynamic optimiser poses many difficult challenges which
must be overcome. There is a risk that the runtime overhead of the
dynamic optimiser will exceed the performance gained by the
optimisations themselves. The proposed approach to dynamically
autotune SkelCL will overcome one of the most significant overheads
associated with dynamic optimising, which is that of profiling and
tracing. Since Algorithmic Skeletons perform the coordination of
muscle functions, it is possible to forgo the use of many profiling
counters by being to make assumptions about the execution frequency of
certain code paths, given the nature of the skeleton. Additionally,
the placement of profiling counters can be optimised manually.

% skelcl
~\cite{Steuwer2011, Steuwer2013a}

% skelcl 11x 5%
~\cite{Steuwer2012}.

% snippet
Contributors to this overhead include: time spent evaluating dynamic
features and deciding on which optimisations to select (extra
instructions to execute), and either increased code size from having
multiple copies of procedures (bad for branch predictors / instruction
prefetch), or decreased ability for optimisations (because of setting
parameters at runtime instead of at compiled).

% snippet
Whereas current approaches to Algorithmic Skeleton autotuning have
largely relied on huge offline training periods and optimising for
static features, this proposed research will develop an online
autotuner which is capable of adapting to dynamic features at runtime.

\section{Evaluation}

% What kind of evidence will be needed to support these claims or
% hypotheses? Is your evidence experimental or theoretical? Is it
% amenable to statistical analysis?
My hypothesis is that the performance of Algorithmic Skeletons can be
improved using dynamic autotuning. To test this hypothesis, empirical
data must be collected from several representative benchmarks by
comparing the performance of my implementation against: baseline
performance provided by an unmodified SkelCL implementation;\ and a
hand-tuned ``oracle'' implementation using an optimal configuration
discovered through an offline exhaustive search of the optimisation
space.

Other measurable success metrics include: the overhead introduced by
the runtime; the amount of time required to converge to a sufficiently
good configuration; and the ability of the dynamic optimiser to adapt
to changes in dynamic features (e.g.\ system load). All of these
metrics can be evaluated by profiling performance benchmarks.

% stat rigour
\cite{Georges2007}

\section{Work plan}

\begin{figure}[H]
\begin{ganttchart}{1}{26}
  \gantttitle{2015}{26} \\
  \gantttitlelist{1,...,26}{1} \\
  \ganttgroup{Group 1}{1}{7} \\
  \ganttbar{Task 1}{1}{2} \\
  \ganttlinkedbar{Task 2}{3}{7} \ganttnewline
  \ganttmilestone{Milestone}{7} \ganttnewline
  \ganttbar{Final Task}{26}{26}
  \ganttlink{elem2}{elem3}
  \ganttlink{elem3}{elem4}
\end{ganttchart}
\end{figure}

\section{Conclusion}

% Who would benefit from a solution to the problem you have set
% yourself?
Algorithmic Skeletons have been shown to improve programmer
effectiveness for a range of tasks, from general purpose computing to
bioinformatics and complex simulations. For example, the SkelCL
library has been used to implement high performance medical imaging
applications. A dynamic autotuner for SkelCL would improve the
performance of these applications, and provide a starting point for
future research into the online autotuning of Algorithmic Skeletons.

% snippet
We are ideally suited for tackling this difficult problem at
University of Edinburgh. Not only have academic members been
responsible for introducing and developing Algorithmic
Skeletons~\cite{Cole1989, Cole2004, Benoit2005a}, but there is a large
and active research interest in iterative compilation and machine
learning based optimisation~\cite{Fursin2008, Agakov,
Fursin2005}. Previous research at the University of Edinburgh has also
approach the static autotuning of Algorithmic
Skeletons~\cite{Collins2012, Collins2013}, which will provide a solid
source of inspiration and an interesting counterpoint for evaluating
the performance of a dynamic autotuning approach.

% snippet
While iterative compilation is a very well studied field, fewer papers
have been published about dynamic optimisation. Therefore work in this
field has a greater chance of influencing future research, besides the
primary benefit of improving the performance of algorithmic skeletons.

\input{_\jobname-post.tex}
