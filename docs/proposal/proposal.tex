%%%%%%%%%%%%%%%%%%%%%%
%% Document details %%
%%%%%%%%%%%%%%%%%%%%%%

% Author
\author{Chris Cummins}

% Date (Month Year)
\date{December 2014}

% Paper title
\title{Dynamic Autotuning\\of Algorithmic Skeletons}
%\newcommand{\multilinetitle}{}

% Subtitle
\newcommand{\subtitle}{MSc Research Proposal}

% Degree title
\newcommand{\degreeTitle}{MSc by Research\\ Pervasive Parallelism}

% Institution
\newcommand{\institution}{School of Informatics,\\
  The University of Edinburgh}

\input{_\jobname.tex}

%%%%%%%%%%
%% Body %%
%%%%%%%%%%
\section{Introduction}
Algorithmic Skeletons enable programmers to quickly write parallel
software by providing generic implementations of reusable
patterns~\cite{Cole1989, Cole2004}. By abstracting common patterns of
communication, frameworks of Higher Order Functions can be written
that provide tuned and robust implementations of parallel
patterns. Users of these patterns provide small sections of
problem-solving logic, called ``muscle functions'', while the
allocation and coordination of resources is managed automatically by
the pattern's implementation.

While frameworks of Algorithmic Skeletons abound, widespread adoption
has so far been restricted largely to established use cases that rely
heavily on high performance and distributed computation, for example,
Google's MapReduce~\cite{Dean2008}, and Intel's Thread Building
Blocks~\cite{IntelTBB}. While the demand for such frameworks in the
field of High Performance Computing (HPC) is self-evident, this should
by no means blinker the ambitions of skeletons research. The benefits
of Algorithmic Skeletons extend beyond that of HPC and cover general
purpose computing.

% snippet
In such cases, the overhead introduced by these massively scaleable
high performance skeleton libraries would likely outweigh the
performance gains. If Algorithmic Skeletons are to achieve widespread
adoption, they must provide scalable performance benefits not only to
the upper-tier of high performance computers but also to modest
consumer hardware, which is increasingly reliant on software
parallelism in order to achieve performance improvements.

\subsection{Motivating Example}
% TODO: Static parameter tuning example

\subsection{Hypothesis}

\section{Background}

% snippet
Many existing dynamic optimisation systems do not store the results of
their efforts persistently, allowing the work to die along with the
host process. This approach relies on the assumption that either that
the convergence time to reach an optimal set of parameters is short
enough to be amortized by the overhead of persistent storage, or that
the runtime of the host process is sufficiently long to reach an
optimal set of parameters in good time. Neither assumption can be
shown to fit the general case.

\section{Methodology}

% snippet
SkelCL is a C++ Algorithmic Skeleton Framework which targets
heterogeneous parallel programming using OpenCL~\cite{Steuwer2011,
Steuwer2013a}. Steuwer, a research associate at the University of
Edinburgh, developed SkelCL as an approach to high-level programming
of multi-GPU systems, demonstrating an $11\times$ reduction in
programmer effort for equivalent programs written in pure OpenCL,
while suffering only a modest 5\% overhead~\cite{Steuwer2012}.

\section{Evaluation}

% snippet
The primary goal of this research is to modify the behaviour of SkelCL
so that it can autotune its performance at runtime. The question which
must be answered when evaluating this goal is: has the performance of
SkelCL been improved? This is itself is not an easy answer to
quantify.

% stat rigour
\cite{Georges2007}


\section{Work plan}

\section{Conclusion}

% snippet
We are ideally suited for tackling this difficult problem at
University of Edinburgh. Not only have academic members been
responsible for introducing and developing Algorithmic
Skeletons~\cite{Cole1989, Cole2004, Benoit2005a}, but there is a large
and active research interest in iterative compilation and machine
learning based optimisation~\cite{Fursin2008, Agakov,
Fursin2005}. Previous research at the University of Edinburgh has also
approach the static autotuning of Algorithmic
Skeletons~\cite{Collins2012, Collins2013}, which will provide a solid
source of inspiration and an interesting counterpoint for evaluating
the performance of a dynamic autotuning approach.

\begin{figure}[H]
\begin{ganttchart}{1}{26}
  \gantttitle{2015}{26} \\
  \gantttitlelist{1,...,26}{1} \\
  \ganttgroup{Group 1}{1}{7} \\
  \ganttbar{Task 1}{1}{2} \\
  \ganttlinkedbar{Task 2}{3}{7} \ganttnewline
  \ganttmilestone{Milestone}{7} \ganttnewline
  \ganttbar{Final Task}{26}{26}
  \ganttlink{elem2}{elem3}
  \ganttlink{elem3}{elem4}
\end{ganttchart}
\end{figure}

\section{Snippets}

\subsection{Introduction}
Parallel computing, traditionally the remit of scientific programming,
is increasingly being seen as the only viable approach for maintaining
continued performance improvements in a multicore computing world.
Despite its growing popularity, writing robust parallel software is an
inherently challenging task, requiring the programmer to think in
unfamiliar paradigms, and with many associated problems raised by race
conditions, deadlock, and managing access to shared resources.

Equivalent parallel programs require many more lines of code than
their sequential counterparts, and the additional programmer effort
that is required is dedicated to writing coordination logic -- the
logic responsible for allocating and coordinating access to shared
resources. Algorithmic Skeletons address the difficulties of parallel
programming by providing a higher level abstraction that encapsulates
this coordination logic, ridding the application programmer of these
responsibilities and allowing them to focus instead on the
problem-solving logic.

\subsection{Motivating Example}

To give a concrete example, consider a recursive merge sorting
algorithm. When called, the algorithm determines whether the input
list is short enough to be solved directly using a linear sorting
method, or whether it should split it into multiple sub-lists and sort
them by recursing on each sub-list before combining the results. This
computational pattern of repeatedly dividing a problem into smaller
subproblems which are then recombined is abstracted by a Divide and
Conquer pattern. This can be parallelised effectively by considering
each recursion as a new task which can be executed concurrently.

The parallel Divide and Conquer pattern is a common form of
Algorithmic Skeleton, whereby the user provides muscle functions for
the split, merge, and conquer logic, and the skeleton can coordinate
the allocation of new tasks. When implementing such a Divide and
Conquer skeleton, there are two immediate parameters which will
greatly affect the performance: the maximum depth at which recursion
should occur as a new task, as opposed to sequentially; and the
threshold minimum size of the input problem before the problem is
conquered directly rather than recursively.

Existing iterative compilation techniques can perform an exhaustive
search of the optimisation space generated by these two parameters,
which would reveals a strong interaction between them: the optimum
value for one parameter is strongly influenced by the value of the
other parameter. Additionally, the optimisation space of both
parameters is strongly influenced by an independent factor: the size
and type of the input problem. This means that in the case of a
parallel merge sort algorithm, the optimum values for the max
recursion depth and minimum input threshold parameters will be very
different when sorting lists of integers and lists of bytes, or lists
of arbitrary user-chosen data structures. This cannot be modelled
using iterative compilation techniques, as the size and type of the
input problem a dynamic features, which can only be determined at
runtime.

Static approaches to this problem involve segmenting the dynamic
feature space using heuristics in order to select optimum values for
approximate ranges. The effectiveness of these heuristics is limited
by their complexity and the thoroughness of the optimisation space
search. In addition, the resulting optimisation heuristics would be
very fragile and non-portable, so that the whole tedious process would
need to be repeated for every target architecture, and with every new
generation of hardware. Such an approach is clearly
impractical. Compare this to the alternate approach of a Divide and
Conquer skeleton which is capable of performing this empirical data
gathering online and during normal execution, and which will use
successive iterations to converge naturally upon an optimum
configuration. Such a system would be capable of dealing with varying
dynamic features which would destroy the capabilities of a static
heuristic based system. This is the goal of my research.

\subsection{Background}
% TODO: reword
Many existing dynamic optimisation systems do not store the results of
their efforts persistently, allowing the work to die along with the
host process. This approach relies on the assumption that either that
the convergence time to reach an optimal set of parameters is short
enough to have negligible performance impact, or that the runtime of
the host process is sufficiently long to reach an optimal set of
parameters in good time. Neither assumption can be shown to fit the
general case.

% PETABRICKS
PetaBricks is a language and runtime which supports dynamic
algorithmic choice determined by properties of the data input. This
provides a promising space for optimisation but has the drawback of
increasing programmer effort by requiring them to implement multiple
versions of an algorithm tailored to different optimisation
parameters. SkelCL has the advantage of being able to localise this
extra programmer effort into a single library implementation.

% WHY MAP REDUCE SUCKS
In such cases, the overhead introduced by these massively scaleable
high performance skeleton libraries would likely outweigh the
performance gains. If Algorithmic Skeletons are to achieve widespread
adoption, they must provide scalable performance benefits not only to
the upper-tier of high performance computers but also to modest
consumer hardware, which is increasingly reliant on software
parallelism in order to achieve performance improvements.

MapReduce is a hugely successful framework for writing high
performance massively distributed applications at a fraction of the
effort required for a hand tuned implementation. The source code of
Hadoop is over 800,000 lines of code, but efficient user programs can
be written in well under 100 lines. While is an incredible technical
feat, the overhead the huge runtime would negate the performance
advantages for all but the largest computations. Users writing
programs for more modestly specced off the shelf hardware will not be
able to take advantage of the engineering achievement.

% What are your claims or hypotheses?
The problem with attempting to model optimisation spaces is that they
are fundamentally stochastic. As a result, they can only properly be
built using empirical evidence, and so the problem becomes one of
trying to extrapolate complicated many-dimensional spaces using the
least amount of data points, since the time taken to acquire data is
prohibitively expensive. Previous static autotuners have taken as long
as three months to sample the optimisation space.

\subsection{Objectives}
% What is your novel idea? What do you bring to this project that is
% new?
Whereas current approaches to Algorithmic Skeleton autotuning have
largely relied on huge offline training periods and optimising for
static features, this proposed research will develop an online
autotuner which is capable of adapting to dynamic features at runtime.

\subsection{Conclusion}

We are ideally suited for tackling this difficult problem at
University of Edinburgh. Not only have academic members been
responsible for introducing and developing Algorithmic Skeletons, but
there is a large and active research interest in iterative compilation
and machine learning based optimisation. Previous research at the
University of Edinburgh has also approach the static autotuning of
Algorithmic Skeletons, which will provide a solid source of
inspiration and an interesting counterpoint for evaluating the
performance of a dynamic autotuning approach.

\subsection*{--break--}

Contributors to this overhead include: time spent evaluating dynamic
features and deciding on which optimisations to select (extra
instructions to execute), and either increased code size from having
multiple copies of procedures (bad for branch predictors / instruction
prefetch), or decreased ability for optimisations (because of setting
parameters at runtime instead of at compiled).

They achieve this by abstracting common patterns of parallel
programming into specialised parallel higher order functions that
orchestrate the execution of ``muscle functions'' - small sections of
problem-solving logic provided by the user.
% TODO: example

Research interest in Algorithmic Skeletons is high, and while
frameworks of Algorithmic Skeletons abound, widespread adoption has so
far been restricted largely to established use cases that rely heavily
on high performance computing, for example, Google's MapReduce, and
Intel's Thread Building Blocks. While the demand for such frameworks
in the field of High Performance Computing (HPC) is self-evident, this
should by no means blinker the ambitions of skeletons research. The
benefits of Algorithmic Skeletons extend beyond that of HPC and cover
general purpose computing.

% Why do you think you can extend the state of the art? Can you solve
% any previously unsolved problems or overcome limitations of previous
% work? Why might you succeed where others have failed?

%%%%%%%%%%%%%%
% MOTIVATION %
%%%%%%%%%%%%%%

% What is the motivation of your project? That is, why do you think it
% will make a significant and original contribution to the scientific
% and/or engineering progress of Informatics?
While iterative compilation is a very well studied field, fewer papers
have been published about dynamic optimisation. Therefore work in this
field has a greater chance of influencing future research, besides the
primary benefit of improving the performance of algorithmic skeletons.

The project is motivating by empirical evidence showing that
optimising algorithms requires both static information about the
structure of the algorithm, and dynamic information about the type and
nature of algorithm inputs. Or rather, static optimisations cannot
provide a universally optimum configuration for all inputs. For
example, a sorting algorithm may display different asymptotic
performance based on the type of data being sorted, the size of the
list to be sorted, and the extent to which the list is already
sorted. Since these factors cannot be determined at compilation time,
programmers are reliant on crude static heuristics to switch been
different optimisations, or attempting to build models to predict
dynamic optimisation configurations. The modelling approach has been
know to fail due to the time required to build models - varying
architectures exhibit wildly different behaviours, etc.

% WHO CARES

% Why is it timely to tackle this project now and do you think it is
% feasible to achieve in the timescale available to you?

The popularity of programming with parallel patterns is rapidly
increasing, as they have been demonstrated as a means of providing
re-usability to the thousands of man-hours that is required to write a
tuned and stable parallel application. For example, Google's
MapReduce, which allows their programmers to write data sorting
programs in 55 lines of code, while taking advantage of the over
800,000 lines of code present in a MapReduce implementation. Any
research forwarding the cause of parallel patterns will prove
extremely valuable to both application developers and future
researchers.

\input{_\jobname-post.tex}
