%%%%%%%%%%%%%%%%%%%%%%
%% Document details %%
%%%%%%%%%%%%%%%%%%%%%%

% Author
\author{Chris Cummins}

% Date (Month Year)
\date{December 2014}

% Paper title
\title{Dynamic Autotuning\\of Algorithmic Skeletons}
%\newcommand{\multilinetitle}{}

% Subtitle
\newcommand{\subtitle}{MSc Research Proposal}

% Degree title
\newcommand{\degreeTitle}{MSc by Research\\ Pervasive Parallelism}

% Institution
\newcommand{\institution}{School of Informatics,\\
  The University of Edinburgh}

\input{_\jobname.tex}

%%%%%%%%%%
%% Body %%
%%%%%%%%%%
\section{Introduction}
Algorithmic Skeletons enable programmers to quickly write parallel
software by providing generic implementations of reusable
patterns~\cite{Cole1989, Cole2004}. By abstracting common patterns of
communication, frameworks of Higher Order Functions can be written
that provide tuned and robust implementations of parallel
patterns. Users of these patterns provide small sections of
problem-solving logic, called ``muscle functions'', while the
allocation and coordination of resources is managed automatically by
the pattern's implementation.

While frameworks of Algorithmic Skeletons abound, widespread adoption
has so far been restricted largely to established use cases that rely
heavily on high performance and distributed computation, for example,
Google's MapReduce~\cite{Dean2008}, and Intel's Thread Building
Blocks~\cite{IntelTBB}. While the demand for such frameworks in the
field of High Performance Computing (HPC) is self-evident, this should
by no means blinker the ambitions of skeletons research. The benefits
of Algorithmic Skeletons extend beyond that of HPC and cover general
purpose computing.

% snippet
In such cases, the overhead introduced by these massively scaleable
high performance skeleton libraries would likely outweigh the
performance gains. If Algorithmic Skeletons are to achieve widespread
adoption, they must provide scalable performance benefits not only to
the upper-tier of high performance computers but also to modest
consumer hardware, which is increasingly reliant on software
parallelism in order to achieve performance improvements.

\subsection{Motivating Example}
% TODO: Static parameter tuning example

\subsection{Hypothesis}

\section{Background}

% snippet
Many existing dynamic optimisation systems do not store the results of
their efforts persistently, allowing the work to die along with the
host process. This approach relies on the assumption that either that
the convergence time to reach an optimal set of parameters is short
enough to be amortized by the overhead of persistent storage, or that
the runtime of the host process is sufficiently long to reach an
optimal set of parameters in good time. Neither assumption can be
shown to fit the general case.

\section{Methodology}

% snippet
SkelCL is a C++ Algorithmic Skeleton Framework which targets
heterogeneous parallel programming using OpenCL~\cite{Steuwer2011,
Steuwer2013a}. Steuwer, a research associate at the University of
Edinburgh, developed SkelCL as an approach to high-level programming
of multi-GPU systems, demonstrating an $11\times$ reduction in
programmer effort for equivalent programs written in pure OpenCL,
while suffering only a modest 5\% overhead~\cite{Steuwer2012}.

\section{Evaluation}

% snippet
The primary goal of this research is to modify the behaviour of SkelCL
so that it can autotune its performance at runtime. The question which
must be answered when evaluating this goal is: has the performance of
SkelCL been improved? This is itself is not an easy answer to
quantify.

% stat rigour
\cite{Georges2007}


\section{Work plan}

\section{Conclusion}

% snippet
We are ideally suited for tackling this difficult problem at
University of Edinburgh. Not only have academic members been
responsible for introducing and developing Algorithmic
Skeletons~\cite{Cole1989, Cole2004, Benoit2005a}, but there is a large
and active research interest in iterative compilation and machine
learning based optimisation~\cite{Fursin2008, Agakov,
Fursin2005}. Previous research at the University of Edinburgh has also
approach the static autotuning of Algorithmic
Skeletons~\cite{Collins2012, Collins2013}, which will provide a solid
source of inspiration and an interesting counterpoint for evaluating
the performance of a dynamic autotuning approach.

\input{_\jobname-post.tex}
