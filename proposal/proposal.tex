%%%%%%%%%%%%%%%%%%%%%%
%% Document details %%
%%%%%%%%%%%%%%%%%%%%%%

% Author
\author{Chris Cummins}

% Date (Month Year)
\date{December 2014}

% Paper title
\title{Dynamic Autotuning\\of Algorithmic Skeletons}
%\newcommand{\multilinetitle}{}

% Subtitle
\newcommand{\subtitle}{MSc Research Proposal}

% Degree title
\newcommand{\degreeTitle}{MSc by Research\\ Pervasive Parallelism}

% Institution
\newcommand{\institution}{School of Informatics,\\
  The University of Edinburgh}

\input{_\jobname.tex}

%%%%%%%%%%
%% Body %%
%%%%%%%%%%
\section{Introduction}
Algorithmic Skeletons enable programmers to quickly write parallel
software by providing generic implementations of reusable
patterns~\cite{Cole1989, Cole2004}. By abstracting common patterns of
communication, frameworks of Higher Order Functions can be written
that provide tuned and robust implementations of parallel
patterns. Users of these patterns provide small sections of
problem-solving logic, called ``muscle functions'', while the
allocation and coordination of resources is managed automatically by
the pattern's implementation.

While frameworks of Algorithmic Skeletons abound, widespread adoption
has so far restricted largely to established use cases that rely
heavily on high performance and distributed computation, for example,
Google's MapReduce~\cite{Dean2008}. While the demand for such
frameworks in the field of High Performance Computing (HPC) is
self-evident, this should by no means blinker the ambitions of
skeleton research. The benefits of Algorithmic Skeletons extend beyond
that of HPC and cover general purpose computing.

% snippet
In such cases, the overhead introduced by these massively scaleable
high performance skeleton libraries would likely outweigh the
performance gains. If Algorithmic Skeletons are to achieve widespread
adoption, they must provide scalable performance benefits not only to
the upper-tier of high performance computers but also to modest
consumer hardware, which is increasingly reliant on software
parallelism in order to achieve performance improvements.
% snippet

\subsection{Motivating Example}
\subsection{Hypothesis}
\section{Background}
\section{Methodology}
\section{Evaluation}
\section{Work plan}

\section{Conclusion}
We are ideally suited for tackling this difficult problem at
University of Edinburgh. Not only have academic members been
responsible for introducing and developing Algorithmic
Skeletons~\cite{Cole1989, Cole2004, Benoit2005a}, but there is a large
and active research interest in iterative compilation and machine
learning based optimisation~\cite{Fursin2008, Agakov,
Fursin2005}. Previous research at the University of Edinburgh has also
approach the static autotuning of Algorithmic
Skeletons~\cite{Collins2012, Collins2013}, which will provide a solid
source of inspiration and an interesting counterpoint for evaluating
the performance of a dynamic autotuning approach.

\input{_\jobname-post.tex}
